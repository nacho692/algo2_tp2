\begin{Interfaz}
	\textbf{parámetros formales}\hangindent=2\parindent\\
	\parbox{1.7cm}{\textbf{géneros}} $\alpha$\\
	\parbox[t]{1.7cm}{\textbf{función}}\parbox[t]{\textwidth-2\parindent-1.7cm}{%
		\InterfazFuncion{Copiar}{\In{a}{$\alpha$}}{$\alpha$}
	    	{$res \igobs a$}
		[$\Theta(copy(a))$]
		[función de copia de $\alpha$'s]
	}
  
	\textbf{se explica con}: \tadNombre{Campus}
	
	\textbf{géneros}: \TipoVariable{campus}
	
	\tituloModulo{Operaciones de Campus}

	\InterfazFuncion{CrearCampus}{\In{ancho}{nat}, \In{alto}{nat}}{campus}
	{res $\igobs$ crearCampus(ancho,alto)}
	[$\Ogr$(alto x ancho)]
	[Crea un campus sin obstaculos]
	
	\InterfazFuncion{AgregarObstaculo}{\Inout{c}{campus}, \In{p}{posicion}}{}
	[posValida?(p,c) $\land$ $\neg$ocupada(p, c) $\land$ c = $c_{0}$]
	{c $\igobs$ agregarObstáculo(p, $c_{0}$)}
	[$\Ogr$(1)]
	[Agrega un obstáculo en la posición dada]
	
	\InterfazFuncion{Filas}{\In{c}{campus}}{nat}
	{res $\igobs$ filas(c)}
	[$\Ogr$(1)]
	[Indica la cantidad de filas del campus dado]
	
	\InterfazFuncion{Columnas}{\In{c}{campus}}{nat}
	{res $\igobs$ columnas(c)}
	[$\Ogr$(1)]
	[Indica la cantidad de columnas del campus dado]	
	
	\InterfazFuncion{Ocupada}{\In{c}{campus}, \In{p}{posicion}}{bool}
	[posValida?(p,c)]
	{res $\igobs$ ocupada(p,c)}
	[$\Ogr$(1)]
	[Indica si la posición dada tiene un obstáculo]
	
	\InterfazFuncion{PosValida}{\In{c}{campus}, \In{p}{posicion}}{bool}
	{res $\igobs$ posValida?(p,c)}
	[$\Ogr$(1)]
	
	\InterfazFuncion{IngresoSuperior}{\In{c}{campus}, \In{p}{posicion}}{bool}
	{res $\igobs$ ingresoSuperior?(p,c)}
	[$\Ogr$(1)]
	
	\InterfazFuncion{IngresoInferior}{\In{c}{campus}, \In{p}{posicion}}{bool}
	{res $\igobs$ ingresoInferior?(p,c)}
	[$\Ogr$(1)]
	
	\InterfazFuncion{EsIngreso}{\In{c}{campus}, \In{p}{posicion}}{bool}
	{res $\igobs$ esIngreso?(p,c)}
	[$\Ogr$(1)]
	
	\InterfazFuncion{Vecinos}{\In{c}{campus}, \In{p}{posicion}}{con(posicion)}
	[posValida?(p,c)]
	{res $\igobs$ vecinos?(p,c)}
	[$\Ogr$(1)]
	[Dada una posición, devuelve las 4 posiciones vecinas]
		
	\InterfazFuncion{VecinosComunes}{\In{c}{campus}, \In{p_1}{posicion}, \In{p_2}{posicion}}{conj(posicion)}
	[posValida?($p_1$,c) $\land$ posValida?($p_1$,c)]
	{res $\igobs$ vecinosComunes($p_1$,$p_2$,c)}
	[$\Ogr$(1)]
	
	\InterfazFuncion{VecinosValidos}{\In{c}{campus}, \Inout{cp}{conj(posicion)}}{}
	[c = $c_0$]
	{c $\igobs$ vecinosValidos}
	[$\Ogr$( |cp| )]
	[Dado un conjunto de posiciones devuelve el conjunto resultado de quitar las posiciones no validas]
	
	\InterfazFuncion{ProxPosicion}{\In{c}{campus}, \In{p}{posicion}, \In{d}{direccion}}{posicion}
	[posValida?(p,c)]
	{res $\igobs$ proxPosición(p,d,c)}
	[$\Ogr$(1)]
	[Devuelve el resultado de mover la posición indicada en la dirección dada]
		
	\InterfazFuncion{IngresosMasCercanos}{\In{c}{campus}, \In{p}{posicion}}{conj(posicion)}
	[posValida?(p,c)]
	{res $\igobs$ ingresosMásCercanos(p,c)}
	[$\Ogr$(1)]
	[Devuelve el conjunto de ingresos más cercanos a la posición dada]
				
\end{Interfaz}

\begin{Representacion}

	\tituloModulo{Representación del campus}
	
	\begin{Estructura}{campus}
	donde estr es Matriz(bool)
	\end{Estructura}
	
	\Rep{true}
	
~	
	
	\AbsFc{campus}{c $:$ campus $/$\\
	filas(c) $=$ Alto(e) $\land$ columnas(c) $=$ Ancho(e)\\
	$\land$ ($\forall$ p : posicion)( posValida?(p,c) $\impluego$ ocupada?(p,c) = e[p.X $-$ 1][p.Y $-$ 1] )
	}
	
	
\begin{Algoritmos}

	\begin{algorithm}[1]
		\caption{iCrearCampus}
		
		\begin{algorithmic}
			\Procedure{iCrearCampus}{\texttt{in} ancho : \texttt{nat}, \texttt{in} alto : \texttt{nat}} $\to res$ : $estr$
				\State $res \gets$ Crear(ancho,alto,false) \Comment $\Ogr$(ancho x alto)
			
			\EndProcedure
		\end{algorithmic}
		\underline{Complejidad:} $\Ogr$(ancho * alto)
				
	\end{algorithm}
	
	\begin{algorithm}
		\caption{iAgregar}
		\begin{algorithmic}
			\Procedure{iAgregar}{\texttt{in/out} e : \texttt{estr}, \texttt{in} p : \texttt{posicion}}
				\State e[p.X $-$ 1][p.Y $-$ 1] $\gets$ true \Comment $\Ogr$(1)
			
			\EndProcedure
		\end{algorithmic}
		\underline{Complejidad:} $\Ogr$(1)
	
	
	\end{algorithm}	
	
	\begin{algorithm}
		\caption{iFila}
		\begin{algorithmic}
			\Procedure{iFila}{\texttt{in} e : \texttt{estr}} $\to res$ : $nat$
				\State $res \gets$ Alto($e$)	\Comment $\Ogr$(1)
			\EndProcedure		
		\end{algorithmic}
		\underline{Complejidad:} $\Ogr$(1)
	\end{algorithm}
	
	\begin{algorithm}
		\caption{iColumna}
		\begin{algorithmic}
			\Procedure{iColumna}{\texttt{in} e : \texttt{estr}} $\to res$ : $nat$
				\State $res \gets$ Ancho($e$) \Comment $\Ogr$(1)
			\EndProcedure		
		\end{algorithmic}
		\underline{Complejidad:} $\Ogr$(1)
	\end{algorithm}
	
	\begin{algorithm}
		\caption{iOcupada}
		\begin{algorithmic}
			\Procedure{iOcupada}{\texttt{in} e : \texttt{estr}, \texttt{in} p: \texttt{posicion}} $\to res$ : $bool$	
				\State $res \gets$ e[p.X $-$ 1][e.Y $-$ 1] \Comment $\Ogr$(1)
			\EndProcedure		
		\end{algorithmic}
		\underline{Complejidad:} $\Ogr$(1)
	\end{algorithm}
	
	\begin{algorithm}
		\caption{iPosValida}
		\begin{algorithmic}
			\Procedure{iPosValida}{\texttt{in} e: \texttt{estr}, \texttt{in} p : \texttt{posicion}} $\to res$ : $bool$
				\State 0 $<$ p.X $\land$ p.X $\leq$ Ancho(e) $\land$ 0 $<$ p.Y $\land$ p.Y $\leq$ Alto(e) \Comment $\Ogr$(1)
			\EndProcedure		
		\end{algorithmic}
		\underline{Complejidad:} $\Ogr$(1)
	\end{algorithm}

	\begin{algorithm}
		\caption{iIngresoSuperior}
		\begin{algorithmic}
			\Procedure{iIngresoSuperior}{\texttt{in} e : \texttt{estr}, \texttt{in} p : \texttt{posicion}} $to res$ : $bool$
				\State $res \gets$ (p.Y $=$ 1) \Comment $\Ogr$(1)		
			\EndProcedure		
		\end{algorithmic}
		\underline{Complejidad:} $\Ogr$(1)
	\end{algorithm}
	
	\begin{algorithm}
		\caption{iIngresoInferior}
		\begin{algorithmic}
			\Procedure{iIngresoInferior}{\texttt{in} e : \texttt{estr}, \texttt{in} p : \texttt{posicion}} $to res$ : $bool$
				\State $res \gets$ (p.Y $=$ Alto()) \Comment $\Ogr$(1)		
			\EndProcedure		
		\end{algorithmic}
		\underline{Complejidad:} $\Ogr$(1)
	\end{algorithm}
	
	\begin{algorithm}
		\caption{iEsIngreso}
		\begin{algorithmic}
			\Procedure{iEsIngreso}{\texttt{in} e : \texttt{estr}, \texttt{in} p : \texttt{posicion}} $\to res$ : $bool$
				\State $res \gets$ (iEsIngresoSuperior(e,p) $\land$ iEsIngresoInferior(e,p)) \Comment $\Ogr$(1)
			\EndProcedure		
		\end{algorithmic}
		\underline{Complejidad:} $\Ogr$(1)
	\end{algorithm}
	
	\begin{algorithm}
		\caption{iVecinos}
		\begin{algorithmic}
			\Procedure{iVecinos}{\texttt{in} e : \texttt{estr}, \texttt{in} p : \texttt{posicion}} $\to res$ : $conj(posicion)$
				\State $res \gets$ Vacio() //vacio de conjunto \Comment $\Ogr$(1)
				\State AgregarRapido(res, <p.X $-$ 1, p.Y>) \Comment $\Ogr$(1)
				\State AgregarRapido(res, <p.X $+$ 1, p.Y>) \Comment $\Ogr$(1)
				\State AgregarRapido(res, <p.X, p.Y $-$ 1>) \Comment $\Ogr$(1)
				\State AgregarRapido(res, <p.X, p.Y $+$ 1>) \Comment $\Ogr$(1)
				\State $res \gets$ iVecinosValidos(e, res) \Comment $\Ogr$($\#$res) $=$ $\Ogr$(4) $=$ $\Ogr$(1)				
			\EndProcedure		
		\end{algorithmic}
		\underline{Complejidad:} $\Ogr$(1)
	\end{algorithm}
		
	\begin{algorithm}
		\caption{iVecinosComunes}
		\begin{algorithmic}
			\Procedure{iVecinosComunes}{\texttt{in} e : \texttt{estr}, \texttt{in} $p_1$ : \texttt{posicion}, \texttt{in} $p_2$ : \texttt{posicion}} $\to res$ : $conj(posicion)$
				\State $vec_1 \gets$ iVecinos(e, $p_1$) \Comment $\Ogr$(1)
				\State $vec_2 \gets$ iVecinos(e, $p_2$) \Comment $\Ogr$(1)
				\State $res \gets$ Vacio() //vacio de conjunto \Comment $\Ogr$(1)
				\State $it_1 \gets$ CrearIt($vec_1$) \Comment $\Ogr$(1)
				\While{HaySiguiente($it_1$)} \Comment Se repite 4 veces y lo de adentro es $\Ogr$(1) $\Rightarrow$ $\Ogr$(1)	
					\State $it_2 \gets$ CrearIt($vec_2$) \Comment $\Ogr$(1)
					\While{HaySiguiente($it_2$)} \Comment Se repite 4 veces y lo de adentro es $\Ogr$(1) $\Rightarrow$ $\Ogr$(1)
						\If{Siguiente($it_1$) $=$ Siguiente($it_2$)} \Comment $\Ogr$(1)
							\State AgregarRapido(res, Siguiente($it_1$))
						\EndIf
						\State Avanzar($it_2$) \Comment $\Ogr$(1)
					\EndWhile
					\State Avanzar($it_1$) \Comment $\Ogr$(1)
				\EndWhile		
			\EndProcedure		
		\end{algorithmic}
		\underline{Complejidad:} $\Ogr$(1)
	\end{algorithm}
			
	\begin{algorithm}
		\caption{iVecinosValidos}
		\begin{algorithmic}
			\Procedure{iVecinosValidos}{\texttt{in} e : \texttt{estr},\texttt{in/out} cp : $conj(posicion)$}
				\State $it \gets$ CrearIt(cp) \Comment $\Ogr$(1)
				\While{HaySiguiente(it)} \Comment Se repite $\#$cp veces y lo de adentro es $\Ogr$(1) $\Rightarrow$ $\Ogr$($\#$cp)
					\If{$\neg$ iPosValida(e, Siguiente(it))} \Comment $\Ogr$(1)
						\State EliminarSiguiente(it) \Comment $\Ogr$(1)
					\Else
						\State Avanzar(it) \Comment $\Ogr$(1)
					\EndIf
				\EndWhile
			\EndProcedure		
		\end{algorithmic}
		\underline{Complejidad:} $\Ogr$($\#$cp)
	\end{algorithm}
				
	\begin{algorithm}
		\caption{iProxPosicion}
		\begin{algorithmic}
			\Procedure{iProxPosicion}{\texttt{in} e : \texttt{estr}, \texttt{in} p : \texttt{posicion}, \texttt{in} dir : \texttt{direccion}} $\to res$ : $posicion$
				\State $res \gets$ <p.X $+$ $\beta$(dir $=$ der) $-$ $\beta$(dir $= $ izq), p.Y $+$ $\beta$(dir $=$ abajo) $-$ $\beta$(dir $=$ arriba)> \Comment $\Ogr$(1)
			\EndProcedure		
		\end{algorithmic}
		\underline{Complejidad:} $\Ogr$(1)
	\end{algorithm}
				
	\begin{algorithm}
		\caption{iIngresosMasCercanos}
		\begin{algorithmic}
			\Procedure{iIngresosMasCercanos}{\texttt{in} e : \texttt{estr}, \texttt{in} p : \texttt{posicion}} $\to res$ : $conj(posicion)$
				\State $res \gets$ Vacio() //vacio de conjunto \Comment $\Ogr$(1)
				\If{Dist(p, <p.X, 1>) $<$ Dist(p, <p.Y, iFilas(e)>)} \Comment $\Ogr$(1)
					\State AgregarRapido(res, <p.X, 1>) \Comment $\Ogr$(1)
				\Else
					\If{Dist(p, <p.X, 1>) $>$ Dist(p, <p.Y, iFilas(e)>)} \Comment $\Ogr$(1)
						\State AgregarRapido(res, <p.X, iFilas(e)>) \Comment $\Ogr$(1)
					\Else
						\State AgregarRapido(res, <p.X, 1>) \Comment $\Ogr$(1)
						\State AgregarRapido(res, <p.X, iFilas(e)>) \Comment $\Ogr$(1)
					\EndIf				
				\EndIf
			\EndProcedure		
		\end{algorithmic}
		\underline{Complejidad:} $\Ogr$(1)
	\end{algorithm}
	
\end{Algoritmos}
	    
    
\end{Representacion}