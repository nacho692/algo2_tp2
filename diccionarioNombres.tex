\begin{Interfaz}

	\textbf{parámetros formales}\hangindent=2\parindent\\
	\parbox{1.7cm}{\textbf{géneros}} $\sigma$\\
	\parbox[t]{1.7cm}{\textbf{función}}\parbox[t]{\textwidth-2\parindent-1.7cm}{%
		\InterfazFuncion{Copiar}{\In{s}{$\sigma$}}{$\sigma$}
	    	{$res \igobs a$}
		[$\Theta(copy(s))$]
		[función de copia de $\sigma$'s]
	}
	
	\textbf{se explica con}: \tadNombre{diccionario(string,$\sigma$)}
	
	\textbf{usa}: nat, bool
	
	\textbf{géneros}: \TipoVariable{diccNom(string, $\sigma$)}
	
	\tituloModulo{Operaciones de DiccionarioNombres(string,$\sigma$)}
	
	\InterfazFuncion{Vacio}{}{diccNom(string, $\sigma$)}
	{res $\igobs$ vacio}
	[$\Ogr$(1)]
	[Crea un diccionario vacio.]
	
	\InterfazFuncion{Definir}{\Inout{d}{diccNom(string,$\sigma$)}, \In{c}{string}, \In{s}{$\sigma$}}{itDiccNom(string,$\sigma$)}
	[$\neg$def?(c,d) $\land$ d=$d_0$]
	{d $\igobs$ Definir(c,s,$d_0$) $\land$ haySiguiente(res) $\yluego$ siguiente(res) $\igobs$ <k,s> $\land$ alias(esPermutacion(SecuSuby(res),d))}
	[$\Ogr$(|c| + copy(s))]
	[Agrega un elemento al diccionario y devuelve un iterador apuntando a ese elemento.]	
	
	\InterfazFuncion{Definido}{\In{d}{diccNom(string,$\sigma$)}, \In{c}{string}}{bool}
	{res $\igobs$ def?(c,d)}
	[$\Ogr$(|c|)]
	[Devuelve verdadero si y sólo si la clave de entrada esta definida en el diccionario.]	
	
	\InterfazFuncion{Borrar}{\Inout{d}{diccNom(string,$\sigma$)}, \In{c}{string}}{}
	[def?(c,d) $\land$ d=$d_0$]
	{d $\igobs$ borrar(c,$d_0$)}
	[$\Ogr$($\#$claves(d) $+$ |c|)]
	[Borra el elemento correspondiente a la clave c del diccionario.]
	
	\InterfazFuncion{BorrarRapido}{\Inout{d}{diccNom(string,$\sigma$)}, \In{it}{itDiccNom(string,$\sigma$)}}{}
	[def?(c,d) $\land$ d=$d_0$]
	{d $\igobs$ borrar(c,$d_0$)}
	[$\Ogr$(|c|)]
	[Borra el elemento correspondiente a la clave c del diccionario. c es la clave correspondiente al siguiente del iterador.]
	
	\InterfazFuncion{Obtener}{\Inout{d}{diccNom(string,$\sigma$)}, \In{c}{string}}{$\sigma$}
	[def?(c,d) $\land$ d=$d_0$]
	{alias(res $\igobs$ obtener(c,d)) $\land$ ($\forall$ c' : string)( c' $\in$ claves(s) $\impluego$ obtener(c,d) $\igobs$ obtener(c,$d_0$) )}
	[$\Ogr$(|c|)]
	[Devuelve el significado de la clave dada en el diccionario]
	[res es modificable si y solo si $\sigma$ es modificable]
	
	\InterfazFuncion{$\#$Claves}{\In{d}{diccNom(string,$\sigma$)}}{nat}
	{res $\igobs$ $\#$claves(d)}
	[$\Ogr$(1)]
	[Devuelve la cantidad de claves definidas en el diccionario]
	
\end{Interfaz}

\begin{Representacion}

	\tituloModulo{Representacion del DiccionaroNombres(string, $\sigma$)}
	
	\begin{Estructura}{diccNom(string,$\sigma$)}
		\begin{Tupla}
			\tupItem{lineal}{dicc(string,$\sigma$)}
			\tupItem{trie}{diccT(string,$\sigma$)}
		\end{Tupla}
		Donde diccT(string,$\sigma$) es Lista(tupla<$letra$ : char, $significado$ : puntero($\sigma$), $prox$ : diccT(string, $\sigma$)>)
	\end{Estructura}
	
	\Rep{Rep(e.lineal) $\land$ tamaño(e.trie) $<$ 256 $\land$ SinRepetidos(Letras(e.trie)) \\
		($\forall$ dT : diccT(string,$\sigma$)) ( Esta?(dT, e.trie.prox) $\impluego$ Rep(dT) )	
	}
	
	\tadOperacion{Letra}{li}{secu(char)}{}
	\tadAxioma{Letra(l)}{\IF vacio?(l)
							THEN \secuencia{}
							ELSE prim(l).letra $\bullet$ Letra(fin(l))
						FI}
	Donde li es Lista(tupla<$letra$ : char, $significado$ : puntero($\sigma$), $prox$ : diccT(string, $\sigma$)>
	
~	
	
	\tadOperacion{SinRepetidos}{secu($\alpha$)}{bool}{}
	\tadAxioma{SinRepetidos(s)}{\IF vacio?(s)
									THEN true
									ELSE $\neg$esta?(prim(s),fin(s)) $\land$ SinRepetidos(fin(s))
								FI}
	
~
~
	
	
	\AbsFc{diccNom(string, $\sigma$)}{dN : diccNom(string, $\sigma$) $/$ \\
		Abs(e.lineal) $\igobs$ dN \\
		$\land$ ($\forall$ c : string) (Def?(c,d) $\Leftrightarrow$ Definido(c,e.trie))\\
		$\land$ ($\forall$ c : string) (Def?(c,d) $\impluego$ Obtener(c,d) $\igobs$ Significado(c, e.trie))}
	
~
	\tadOperacion{Definido}{string,trie}{bool}{}
	\tadAxioma{Definido(s,t)}{estaLetra(s[0],t) $\yluego$ \IF Tamaño(s) == 1 
															THEN true
															ELSE Definido(c[1:Tamaño(c)-1],t)
														FI}		

~											
	\tadOperacion{estaLetra}{char,trie}{bool}{}
	\tadAxioma{estaLetra(c,t)}{	\IF vacio?(t)
									THEN false
									ELSE Primero(t).letra $\igobs$ c $\lor$ estaLetra(c, fin(t))
								FI}		

~
	\tadOperacion{Significado}{string/s,trie/t}{$\sigma$}{Definido(s,t)}
	\tadAxioma{Significado(s,e)}{
		\IF Tamaño(s) = 1
			THEN *(Buscar(s[0],t).significado)
			ELSE Significado(s[1:Tamaño(s)-1],Buscar(s[0],t).prox)
		FI }
~
	\tadOperacion{Buscar}{char/c,trie/t}{piso}{estaLetra(c,t)}
	\tadAxioma{Buscar(c,t)}{
		\IF primero(t).letra = c
			THEN Primero(t)
			ELSE Buscar(c,fin(t))
		FI}
	
	Donde piso es tupla<$letra$ : char, $significado$ : puntero($\sigma$), $prox$ : trie>
	
~
	
	Donde trie es diccT(string, $\sigma$)

\end{Representacion}

\begin{Algoritmos}
	
	\begin{algorithm}[H]
		\caption{iVacio}
		\begin{algorithmic}
			\Procedure{iVacio}{} $\to res$ : estr
				\State $res.lineal \gets$ Vacio() //vacio de diccionario lineal \Comment $\Ogr$(1)
				\State $res.trie \gets$ Vacio() //vacio de lista \Comment $\Ogr$(1)
			\EndProcedure
		\end{algorithmic}
		\underline{Complejidad:} $\Ogr$(1)
	\end{algorithm}
	
	\begin{algorithm}[H]
		\caption{iDefinir}
		\begin{algorithmic}
			\Procedure{iDefinir}{\texttt{in/out} e : \texttt{estr}, \texttt{in} c : \texttt{string}, \texttt{in} s : \texttt{$\sigma$}} $\to res$ : $itDiccNom(string,\sigma)$
				\State $res \gets$ DefinirRapido(e.lineal,c,s) \Comment $\Ogr$(copy(s))
				\State $piso \gets$ e.trie \Comment $\Ogr$(1)
				\State $i \gets$ 0
				\While{i $<$ |c|-1} \Comment Se repite |c|-1 veces y lo de adentro es $\Ogr$(1) $\Rightarrow$ $\Ogr$(|c|-1) $=$ $\Ogr$(|c|)
					\If{iEsta(c[i],piso)}
						\State $piso \gets$ iBuscar(c[i], piso).prox \Comment $\Ogr$(1)
					\Else
						\State AgregarAdelante(piso, <c[i],NULL,Vacio()>) \Comment $\Ogr$(1)
						\State $piso \gets$ piso[0].prox \Comment $\Ogr$(1)
					\EndIf
					\State i++ \Comment $\Ogr$(1)
				\EndWhile
				\If{iEsta(c[i],piso)} \Comment $\Ogr$(1)
					\State iBuscar(c[i],piso).significado $\gets$ puntero(SiguienteSignificado(res)) \Comment $\Ogr$(1)
				\Else
					\State AgregarAdelante(piso, <c[i],puntero(SiguienteSignificado(res)),Vacio()>) \Comment $\Ogr$(1)
				\EndIf
			\EndProcedure
		\end{algorithmic}
		\underline{Complejidad:} $\Ogr$(|c| + copy(s))
	\end{algorithm}
	
	\begin{algorithm}[H]
		\caption{iDefinido}
		\begin{algorithmic}
			\Procedure{iDefinido}{\texttt{in} e : \texttt{estr}, \texttt{in} c : \texttt{string}} $\to res$ : $bool$
				\State $piso \gets$ e.trie \Comment $\Ogr$(1)
				\State $i \gets$ 0 \Comment $\Ogr$(1)
				\While{Esta(c[i],piso)} \Comment Se repite a los sumo |c| veces y lo de adentro es $\Ogr$(1) $\Rightarrow$ $\Ogr$(|c|)
					\State $piso \gets$ Buscar(c[i],piso).prox \Comment $\Ogr$(1)
					\State $i \gets$ i++ \Comment $\Ogr$(1)
				\EndWhile
				\State $res \gets$ (i $=$ |c|)
			\EndProcedure
		\end{algorithmic}
		\underline{Complejidad:} $\Ogr$(|c|)
	\end{algorithm}	
	
	\begin{algorithm}[H]
		\caption{iEsta}

		\InterfazFuncion{Esta}{\In{c}{char}, \Inout{t}{diccT(string,$\sigma$)}}{bool}
		[esta(c,t)]
		{res $\igobs$ estaLetra(c,t)}
		[$\Ogr$(1)]
		[Devuelve verdadero si y sólo si el char de entrada esta en alguno de los elementos de la lista.]

~
		
		\begin{algorithmic}
			\Procedure{iEsta}{\texttt{in} c : \texttt{char}, \texttt{in} t : \texttt{diccT(string,$\sigma$)}} $\to res$ : $bool$
				\State $it \gets$ crearIr(t) \Comment $\Ogr$(1)
				\While{haySiguiente(it) $\land$ Siguiente(it).letra $\neq$ c} \Comment Se repite a lo sumo la cantidad de char,\\
				 \hfill que es una constate $\Rightarrow$ $\Ogr$(1)
					\State Avanzar(it) \Comment $\Ogr$(1)
				\EndWhile
				\State $res \gets$ (HaySiguiente(it)) \Comment $\Ogr$(1)
			\EndProcedure
		\end{algorithmic}
		\underline{Complejidad:} $\Ogr$(1)
	\end{algorithm}
	
	\begin{algorithm}[H]
		\caption{iBuscar}

		\InterfazFuncion{Buscar}{\In{c}{char}, \Inout{t}{diccT(string,$\sigma$)}}{tupla<$letra$ char, $significado$ puntero($\sigma$), $prox$ diccT(string,$\sigma$)>}
		[esta(c,t)]
		{alias(res $\igobs$ buscar(c,t))}
		[$\Ogr$(1)]
		[Busca el elemento de la lista que este caracterizado con el char de entrada.]
		[El resultado queda ligado con aliasing.]
		
~	

		\begin{algorithmic}
			\Procedure{iBuscar}{\texttt{in} c : \texttt{char}, \texttt{in} t : \texttt{diccT(string,$\sigma$)}} $\to res$ : $tupla<char, puntero(\sigma), diccT(string,\sigma)>$
				\State $it \gets$ crearIt(t)
				\While{Siguiente(it).letra $\neq$ c} \Comment Se repite a lo sumo la cantidad de char, que es una constate $\Rightarrow$ $\Ogr$(1)
					\State Avanzar(it) \Comment $\Ogr$(1)
				\EndWhile
				\State $res \gets$ Siguiente(it)
			\EndProcedure
		\end{algorithmic}
		\underline{Complejidad:} $\Ogr$(1)
	\end{algorithm}
	
	\begin{algorithm}[H]
		\caption{iObtener}
		\begin{algorithmic}
			\Procedure{iObtener}{\texttt{in/out} e : \texttt{estr}, \texttt{in} c : \texttt{string}} $\to res$ : $\sigma$
				\State $piso \gets$ e.trie \Comment $\Ogr$(1)
				\State $i \gets$ 0
				\While{i < |c| - 1} \Comment Se repite |c| - 1 veces y lo de adentro es $\Ogr$(1) $\Rightarrow$ $\Ogr$(|c|)
					\State $piso \gets$ Buscar(c[i],piso).prox \Comment $\Ogr$(1)
					\State i++ \Comment $\Ogr$(1)
				\EndWhile
				\State $res \gets$ *(Buscar(s[i],piso).significado) \Comment $\Ogr$(1)
			\EndProcedure
		\end{algorithmic}
		\underline{Complejidad:} $\Ogr$(|c|)
	\end{algorithm}
	
	\begin{algorithm}[H]
		\caption{iBorrar}
		\begin{algorithmic}
			\Procedure{iBorrar}{\texttt{in/out} e : \texttt{estr}, \texttt{in} c : \texttt{string}}
				\State Borrar(e.lineal, c) \Comment $\Ogr$($\#$claves(e.lineal))
				\State $piso \gets$ e.trie \Comment $\Ogr$(1)
				\State $i \gets$ 0
				\While{i < |c|-1} \Comment Se repite |c| - 1 veces y lo de adentro es $\Ogr$(1) $\Rightarrow$ $\Ogr$(|c|)
					\State $piso \gets$ Buscar(c[i],piso).prox \Comment $\Ogr$(1)
					\State i++ \Comment $\Ogr$(1)
				\EndWhile
				\State delete(Buscar(s[i],piso).significado)
				\State Buscar(s[i],piso).significado $\gets$ NULL \Comment $\Ogr$(1)
			\EndProcedure
		\end{algorithmic}
		\underline{Comlpejidad:} $\Ogr$(|c| + $\#$claves(e.lineal))
	\end{algorithm}
	
	\begin{algorithm}[H]
		\caption{iBorrarRapido}
		\begin{algorithmic}
			\Procedure{iBorrarRapido}{\texttt{in/out} e : \texttt{estr}, \texttt{in} it : \texttt{itDiccNom(string,$\sigma$)}}
				\State EliminarSiguiente(it) \Comment c es la clave asociada al siguiente del iterador $\Ogr$(|c|)
			\EndProcedure
		\end{algorithmic}
		\underline{Complejidad:} $\Ogr$(|clave|) donde clave es la clave del siguiente del iterador.
	\end{algorithm}
	
	\begin{algorithm}[H]
		\caption{i$\#$Claves}
		\begin{algorithmic}
			\Procedure{i$\#$Claves}{\texttt{in} e : \texttt{estr}}
				\State $res \gets$ $\#$claves(e.lineal) \Comment $\Ogr$(1)
			\EndProcedure
		\end{algorithmic}
		\underline{Complejidad:} $\Ogr$(1)
	\end{algorithm}	
	
\end{Algoritmos}