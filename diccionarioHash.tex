\begin{Interfaz}

	\textbf{parámetros formales}\hangindent=2\parindent\\
	\parbox{1.7cm}{\textbf{géneros}} $\sigma$\\
	
	\textbf{usa}: nat, bool, arreglo Dimensionable
	
	\textbf{géneros}: \TipoVariable{dicch(nat,$\sigma$)}
	
	\tituloModulo{Operaciones de Dicch(nat,$\sigma$)}
	
	\InterfazFuncion
	{Vacio}
	{\In{cant}{nat}}
	{dicch(nat, $\sigma$)}
	[cant >\ 0]
	{res $\igobs$ vacio}
	[$\Ogr$(n)]
	[Crea un diccionario vacio.]
	
	\InterfazFuncion
	{Definir}
	{\In{clave}{nat}, \In{significado}{$\sigma$}, \Inout{dicc}{dicch(nat,$\sigma$)}}
	{}
	[$\neg$def?(clave,dicc) $\land$ dicc=$dicc_0$]
	{dicc $\igobs$ Definir(clave,significado,$dicc_0$)}
	[$\Ogr(1)$ en caso promedio]
	[Agrega un elemento al diccionario.]	
	
	\InterfazFuncion
	{Obtener}
	{\In{clave}{nat}, \Inout{dicc}{dicch(nat,$\sigma$)}}
	{$\sigma$}
	[def?(clave,dicc) $\land$ dicc=$dicc_0$]
	{alias(res $\igobs$ obtener(clave,dicc)) 
		$\land$\\ 
		($\forall$ clave' : nat)( (clave' $\in$ claves(dicc) $\land$ clave $\neq$ clave' )$\impluego$ obtener(clave',dicc) $\igobs$ obtener(clave',$dicc_0$) )}
	[$\Ogr(1)$ en caso promedio]
	
\end{Interfaz}

\begin{Representacion}

	\tituloModulo{Representacion del DiccionaroHash(nat, $\sigma$)}
	
	\begin{Estructura}{dicch(nat,$\sigma$)}
		\begin{Tupla}
			\tupItem{cant}{nat}
			\tupItem{valores}{Adt}
		\end{Tupla}
		Donde Adt es arreglo Dimensionable de tupla<$clave$ : nat, $significado$ : $\sigma$>)
	\end{Estructura}
	
	\Rep{ e.cant $\igobs$ Tamaño(e.valores) $\yluego$ \\ 
	($\forall$ i, j : nat) 
	( i <\ e.cant $\land$ j <\ e.cant $\land$ i $\neq$ j $\land$ definido(e.valores,i) $\land$ definido(e.valores,j)) \\
	$\impluego$ e.valores[i].clave $\neq$ e.valores[j].clave
	}
	
~
	
	\AbsFc
	{dicch(nat, $\sigma$)}
	{dic : dicch(nat, $\sigma$) $/$ \\
		($\forall$ clave : nat) \\
		( Def?(clave,dic) $\Leftrightarrow$ ($\exists$ i : nat)
		( 0 $\leq$ i <\ e.cant $\land$ Definido( e.valores, i ) 
		$\yluego$ e.valores[i].clave = clave )) 
		\\	$\land$ ( Def?(clave,dic) 	\\
		$\impluego$ Obtener(clave,dic) $\igobs$ 
		e.valores[ mod( hash(clave, e ) + corrimiento(hash(clave, e), e) , cant) ].significado}
~
	\tadOperacion{hash}{nat,estr}{nat}{}
	\tadAxioma{hash(clave, e)}{mod(clave, e.cant)}		
~											
	\tadOperacion{corrimiento}{nat,estr}{nat}{}
	\tadAxioma{corrimiento(clave, e)}
	{\IF e.valores[ mod( clave, e.cant) ].clave = clave
		THEN 0
		ELSE 1 + corrimiento(clave + 1, e)
	FI}		

\end{Representacion}

\begin{Algoritmos}
	
	\begin{algorithm}
		\caption{iVacio}
		\begin{algorithmic}
			\Procedure
			{iVacio}
			{\texttt{in} n : nat} $\to res$ : estr
				\State var $e$ : estr
				\State $e.cant \gets$ n \Comment $\Theta$(1)
				\State $e.valores \gets$ CrearArreglo($n$) \Comment $\Ogr$(n)
				\State $res \gets$ e \Comment $\Theta$(1)
			\EndProcedure
		\end{algorithmic}
		\underline{Complejidad:} $\Ogr$(n)
	\end{algorithm}
	
	\begin{algorithm}
		\caption{iDefinir}
		\begin{algorithmic}
			\Procedure
			{iDefinir}
			{\texttt{in} clave : \texttt{nat}, \texttt{in} significado : \texttt{$\sigma$, \texttt{in/out} e : \texttt{estr}}}
				\State var $h$ : nat
				\State $h \gets$ mod(clave, e.cant) \Comment $\Ogr$(1)
				\While{Definido(e.valores, h)} \Comment En promedio no se entra al ciclo
						\State $h \gets$ mod(h+1, e.cant) \Comment $\Ogr$(1)
				\EndWhile
				\State $e.valores[h] \gets$ tupla<clave, significado> \Comment $\Ogr$(1)
			\EndProcedure
		\end{algorithmic}
		\underline{Complejidad:} $\Ogr$(1) promedio
		\\
		\underline{Justificación:} sabiendo que hay una distribución uniforme de 'claves', podemos asumir que en promedio no se entraría al ciclo, cerrando en promedio una suma de operaciones de complejidad $\Ogr$(1).
	\end{algorithm}
	
	\begin{algorithm}
		\caption{iObtener}
		\begin{algorithmic}
			\Procedure
			{iObtener}
			{\texttt{in} clave : \texttt{nat},\texttt{in/out} e : \texttt{estr}} $\to res$ : $\sigma$
				\State var $h$ : nat				
				\State $h \gets$ mod(clave, e.cant) \Comment $\Ogr$(1)
				\While{e.valores[h].clave $\neq$ clave} 
				\Comment En promedio no se entra al ciclo
					\State $h \gets$  mod(h+1, e.cant) \Comment $\Ogr$(1)
				\EndWhile
				\State $res \gets$ e.valores[h].significado \Comment $\Ogr$(1)
			\EndProcedure
		\end{algorithmic}
		\underline{Complejidad:} $\Ogr$(1) promedio \\
		\underline{Justificación:} Misma justificación que el algoritmo anterior
	\end{algorithm}
	
\end{Algoritmos}