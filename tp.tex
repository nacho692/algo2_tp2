\documentclass[a4paper,10pt, nofootinbib]{article}
\usepackage[width=15.5cm, left=3cm, top=2.5cm, right=1cm, left=2cm, height= 24.5cm]{geometry}
\usepackage[spanish]{babel}
\usepackage[utf8]{inputenc}
\usepackage[T1]{fontenc}
\usepackage{xspace}
\usepackage{xargs}
\usepackage{ifthen}
\usepackage{caratula}
\usepackage{fancyhdr}
\usepackage{aed2-tad,aed2-symb,aed2-itef}
\usepackage[bottom]{footmisc}
\usepackage{algorithm}
\usepackage[noend]{algpseudocode}
\usepackage{modulos}
\usepackage[colorlinks=true, linkcolor=blue]{hyperref}
\usepackage{calc}


\newcommand{\moduloNombre}[1]{\textbf{#1}}

\let\NombreFuncion=\textsc
\let\TipoVariable=\texttt
\let\ModificadorArgumento=\textbf
\newcommand{\res}{$res$\xspace}
\newcommand{\tab}{\hspace*{7mm}}

\newcommand{\Ogr}{\mathcal{O}}


\newcommand{\footnoteMio}[2]{\textsuperscript{#1} \lfoot{\footnotesize \parbox{16cm}{\textsuperscript{#1}#2}}}

\newcommandx{\TipoFuncion}[3]{%
  \NombreFuncion{#1}(#2) \ifx#3\empty\else $\to$ \res\,: \TipoVariable{#3}\fi%
}
\newcommand{\In}[2]{\ModificadorArgumento{in} \ensuremath{#1}\,: \TipoVariable{#2}\xspace}
\newcommand{\Out}[2]{\ModificadorArgumento{out} \ensuremath{#1}\,: \TipoVariable{#2}\xspace}
\newcommand{\Inout}[2]{\ModificadorArgumento{in/out} \ensuremath{#1}\,: \TipoVariable{#2}\xspace}
\newcommand{\Aplicar}[2]{\NombreFuncion{#1}(#2)}

\newcommand{\Titulo}[1]{
  \vspace*{1ex}\par\noindent\textbf{\large #1}\par
}
\newcommand{\DRef}{\ensuremath{\rightarrow}}



\author{Algoritmos y Estructuras de Datos II, DC, UBA.}
\date{}
\title{Tipos abstractos de datos básicos}

%Informacion para la caratula
\materia{Algoritmos y Estructuras de Datos II}
\titulo{\Large Trabajo Práctico Nº2}
\grupo{Grupo 19}
\integrante{Basso, Juan Cruz}{627/14}{jcbasso95@gmail.com}
\integrante{Bohe, Brian}{706/14}{brianbohe@gmail.com}
\integrante{Figari, Francisco}{719/14}{francisco.figari@hotmail.com}
\integrante{Mariotti, Ignacio}{651/14}
{nacho692@gmail.com}
\def\cuatrimestre{2}

\begin{document}

%Pagina de titulo e indice
\thispagestyle{empty}

\maketitle

\tableofcontents
\newpage


\clearpage



\section{Módulo Campus Seguro($\alpha$)}
\begin{Interfaz}
	\textbf{se explica con}: \tadNombre{Campus Seguro}.

	\textbf{géneros}: \TipoVariable{campusSeg}.

	\tituloModulo{Operaciones básicas de campusSeg}

	\InterfazFuncion{ComenzarRastrillaje}{\In{c}{campus}, \In{d}{dicc(agente,posicion)}}{campusSeg}%
	[($\forall a$: agente)def?($a$,$d$)$\impluego$ posValida?(obtener($a$,$d$))$\land \neg$ocupada?(obtener($a$,$d$),$c$)))\\
  $\land$($\forall a_0,a_1$: agente)((def?($a_0$,$d$)$\land$def?($a_1$,$d$)$\land a_0 \neq a_1$)$\impluego$obtener($a_0$,$d$)$\neq$obtener($a_1$,$d$))]
	{$res \igobs $comenzarRastrillaje($c$,$d$)}%
	[$\Ogr$($h*a + n^2$)]
	[Inicializa toda la estructura en funcion de los datos de entrada]

	\InterfazFuncion{Campus}{\In{c}{campusSeg}}{campus}
	{alias($res \igobs $campus($c$))}
	[$\Ogr(1)$]
	[Devuelve el campus de Campus Seguro]
	[El campus se devuelve por referencia]

	\InterfazFuncion{Estudiantes}{\In{c}{campusSeg}}{itdicc(nombre,posicion)}
	{alias(esPermutacion(SecuSuby($res$),secuenciarDic($d$)) $\land$ Vacia?(Anteriores($res$))}
	[$\Ogr(1)$]
	[Devuelve Iterador de todos los estudiantes actualmente en el campus]
	[res no es modificable, si hay un cambio en los estudiantes, el iterador puede quedar invalidado]
  
	\InterfazFuncion{Hippies}{\In{c}{campusSeg}}{itdicc(nombre,posicion)}
	{alias(esPermutacion(SecuSuby($res$),secuenciarDic($d$)) $\land$ Vacia?(Anteriores($res$))}
	[$\Ogr(1)$]
	[Devuelve Iterador de todos los hippies actualmente en el campus]
	[res no es modificable, si hay un cambio en los hippies, el iterador puede quedar invalidado]

	\InterfazFuncion{Agentes}{\In{c}{campusSeg}}{itdiccAg(nombre,datosAg)}
	{alias(esPermutacion(SecuSuby($res$),secuenciarDic($d$)) $\land$ Vacia?(Anteriores($res$))}
	[$\Ogr(1)$]
	[Devuelve Iterador de todos los agentes actualmente en el campus]
	[res no es modificable, si hay un cambio en el diccionario, se va a ver reflejado en el iterador]

	\InterfazFuncion{PosEstudianteYHippie}{\In{c}{campusSeg}, \In{n}{nombre}}{posicion}
	[$n \in$ estudiantes($c$)$\cup$hippies($c$)]
	{$res \igobs$posEstudianteYHippie($n$,$c$)}
	[$\Ogr(|n_m|)$ con $|n_m|$ siendo la longitud del nombre más largo de hippies y estudiantes]
	[Devuelve la posicion del estudiante o hippie]

	\InterfazFuncion{PosAgente}{\In{c}{campusSeg}, \In{p}{placa}}{posicion}
	[$pl \in$ agentes($c$)]
	{$res \igobs$ posAgente($pl$,$c$)}
	[$\Ogr(1)$ en caso promedio]
	[Devuelve la posicion del agente]

	\InterfazFuncion{CantHippiesAtrapados}{\In{c}{campusSeg}, \In{pl}{placa}}{nat}
	[$pl \in$ agentes($c$)]
	{$res \igobs$ cantHippiesAtrapados($pl$,$c$)}
	[$\Ogr(1)$ en caso promedio]
	[Devuelve la cantidad de hippies que atrapo el agente]

	\InterfazFuncion{CantSanciones}{\In{c}{campusSeg}, \In{pl}{placa}}{nat}
	[$pl \in$ agentes($c$)]
	{$res \igobs$ cantSanciones($pl$,$c$)}
	[$\Ogr(1)$ en caso promedio]
	[Devuelve la cantidad de sanciones del agente]
	

	
\end{Interfaz}

\begin{Representacion}
  
  \tituloModulo{Representación de campusSeg}

  \begin{Estructura}{campusSeg}[estr]
    \begin{Tupla}[estr]
		\tupItem{hippies}{diccNombres(nombre,posicion)}%
		\tupItem{estudiantes}{diccNombres(nombre,posicion)\\}
		\tupItem{agentes}{diccAg(placa,datosAg)}
		\tupItem{sanciones}{diccSanc(nat,conj(placa))}\\
		\tupItem{placasPos}{diccLog(placa,posicion)}\\
		\tupItem{pos}{posiciones}
		\tupItem{masVigilante}{datosAg}
		\tupItem{cp}{campus}
    \end{Tupla}
    
	\begin{Tupla}[posiciones]
		\tupItem{mAg}{matriz(tupla(bool,datosAg)}\\
		\tupItem{mH}{matriz(tupla(bool,itdicc(nombre, posicion))}\\
		\tupItem{mE}{matriz(tupla(bool,itdicc(nombre, posicion))}
	\end{Tupla}
	
    \begin{Tupla}[datosAg]
      \tupItem{pl}{placa}
      \tupItem{sanciones}{nat}
      \tupItem{premios}{nat}
      \tupItem{pos}{posicion}%
      \tupItem{kSanc}{itdicc(nat, conj(placa)}%
      \tupItem{mismasSanc}{itconj(placa)}%
    \end{Tupla}
  \end{Estructura}

  \tituloModulo{Aclaraciones}
	Dada la complejidad de la estructura, el invariante puede ser dificil de seguir, por lo que a continuacion se encuentra detallado de manera informal las lineas que se siguieron al hacer el rep.
	
	\begin{enumerate}
		\item Las claves de hippies y estudiantes intersecan vacio.
		\item No hay posiciones iguales en estudiantes, hippies y placasPos y todas se encuentran en rango.
		\item Las placas de agentes y placasPos son iguales y igual placa, misma posicion.
		\item Las claves de sanciones y las diferentes sanciones de agentes forman el mismo conjunto. Los conjuntos de placas de la clave i de sanciones tienen todas las placas cuyos agentes tengan i sanciones.
		\item Para todos los datosAg de agentes, la secuencia subyacente del iterador kSanc es igual a una secuencia ordenada de sanciones y la siguiente clave del iterador es la cantidad de sanciones del datosAg.
		\item Para todos los datosAg de agentes, la secuencia subyacente del iterador mismasSanc es permutacion del conjunto de placas del diccionario sanciones donde se encuentra su propia placa, y el siguiente del iterador es su misma placa.
		\item filas+1 y columnas+1 de cp (campus) es igual al alto y ancho de las matrices de pos.
		\item mAg, el booleano es true si y solo si hay un agente en placasPos en esa posicion.\\
		Si es true, los datosAg son iguales para esa placa.
		\item mH, el booleano es true si y solo si hay un hippie en esa posicion.\\
		Si es true,la siguiente clave del iterador es el hippie en esa posicion y la secuencia subyacente es permutacion de los hippies en hippies.
		\item mE, el booleano es true si y solo si hay un estudiante en esa posicion.\\
		Si es true,la siguiente clave del iterador es el estudiante en esa posicion y la secuencia subyacente es permutacion de los estudiantes en estudiantes.
		\item El mas vigilante es el de mayor sanciones de los agentes y, entre ellos, el de menor placa.
	\end{enumerate}
	
	Para facilitar la lectura se renombra las matrices $e$.pos.mXX a mXX
  \Rep[estr]{
	\begin{enumerate}
		\item claves($e$.hippies) $\cap$ claves(e.estudiantes) = $\emptyset$
		\item $\land$ ($\forall p$: posicion)($\#$($p$,armarMultiPos($e$)) $\geq$ 1 $\implies$ \\
		$\#$($p$,armarMultiPos($e$)) = 1 $\land$ posValida($p$,$e$.cp) $\yluego$ $\neg$ocupada?($p$,$e$.cp))
		\item $\land$ (claves($e$.agentes) $=$ claves($e$.placasPos) $\yluego$ ($\forall p$: placa) (def?($p$,e.placasPos) $\impluego$ \\
		obtener($p$,$e$.placasPos) = obtener($p$,$e$.agentes).pos))
		\item $\land$ unionSignificados($e$.sanciones) = claves($e$.agentes)\\
		$\land$ ($\forall n$: nat) def?($n$,$e$.sanciones) $\impluego$ \\
		($\neg$(obtener($n$,$e$.sanciones) = $\emptyset$) $\land$ ($\forall p$: placa) ($p \in$ obtener($n$,$e$.sanciones) $\implies$ \\
		def?($p$,$e$.agentes) $\yluego$ (obtener($p$,$e$.agentes).pl = p $\land$ obtener($p$,$e$.agentes).sanciones = n))
		\item $\land$ ($\forall p$: placa) def?($p$,$e$.agentes) $\impluego$ \\
		SecuSuby(obtener($p$,$e$.agentes).kSanc) = ordenar$\pi_1$(secuenciarDic($e$.sanciones)) \\
		$\land$ $pi_1$(Siguiente(obtener($p$,$e$.agentes).kSanc)) = obtener($p$,$e$.agentes).pl
		\item $\land$ ($\forall p$: placa) (def?($p$,$e$.agentes) $\yluego$ def?(obtener($p$,$e$.agentes).sanciones,$e$.sanciones) $\impluego$ \\
		esPermutacion(SecuSuby(obtener($p$,$e$.agentes).mismasSanc),\\
		secuenciarConj(obtener(obtener($p$,$e$.agentes).sanciones,$e$.sanciones)))\\
		$\land$ siguiente(obtener($p$,$e$.agentes).mismasSanc) = $p$
		\item $\land$ Ancho(mAg) = columnas($e$.campus) + 1 $\land$ Alto(mAg) = filas($e$.campus) + 1 \\
		$\land$ Ancho(mH) = Ancho(mAg) $\land$ Alto(mH) = Alto(mAg)\\
		$\land$ Ancho(mE) = Ancho(mH) $\land$ Alto(mE) = Alto(mH)
		\item $\land$ ($\forall i$: nat) ($\forall j$: nat) 0 $< i <$ Ancho(mAg) $\land$ 0 $< j <$ Alto(mAg) $\impluego$ \\
		($pi_1$(Valor(mAg,$i$,$j$)) $\iff$ \\
		($\exists n$: nombre)(def?($p$,$e$.placasPos) $\land$ def?($p$,$e$.agentes)) $\yluego$\\
		(obtener($p$,$e$.placasPos) = $\langle i$, $j \rangle$ $\land$ $\pi_2$(Valor(mAg,$i$,$j$)) = obtener($p$,$e$.agentes)))
		\item $\land$ ($\forall i$: nat) ($\forall j$: nat) 0 $< i <$ Ancho(mH) $\land$ 0 $< j <$ Alto(mH) $\impluego$ \\
		($\pi_1$(Valor(mH,$i$,$j$)) $\iff$ ($\exists n$: nombre)(def?($n$,$e$.hippies) $\yluego$ \\
		obtener($n$,$e$.hippies) = $\langle i$, $j \rangle$ $\land$ $\pi_1$(Siguiente($\pi_2$(Valor(mH,$i$,$j$)))) = $n$ $\land$ \\
		esPermutacion(SecuSuby($\pi_2$(Valor(mH,$i$,$j$)),secuenciarDic($e$.hippies)))
		\item $\land$ ($\forall i$: nat) ($\forall j$: nat) 0 $< i <$ Ancho(mE) $\land$ 0 $< j <$ Alto(mE) $\impluego$ \\
		($\pi_1$(Valor(mE,$i$,$j$)) $\iff$ ($\exists n$: nombre)(def?($n$,$e$.estudiantes) $\yluego$ \\
		obtener($n$,$e$.estudiantes) = $\langle i$, $j \rangle$ $\land$ $\pi_1$(Siguiente($\pi_2$(Valor(mE,$i$,$j$)))) = $n$ $\land$ \\
		esPermutacion(SecuSuby($\pi_2$(Valor(mE,$i$,$j$)),secuenciarDic($e$.estudiantes)))
		\item $\#$claves($e$.agentes) $\geq$ 1 $\implies$ ($\exists p$: placa) def?($p$,$e$.agentes) $\yluego$ \\
		(obtener($p$,$e$.agentes) = $e$.masVigilante $\land$ \\
		(($\forall p':$ placa) def?($p'$,$e$.agentes) $\land p \neq p' \impluego$\\
		(obtener($p$,$e$.agentes).premios $\geq$ obtener($p'$.$e$.agentes).premios $\land$ \\
		(obtener($p$,$e$.agentes).premios = obtener($p'$,$e$.agentes) $\implies$ $p < p'$))))

	\end{enumerate}
  }\mbox{}
	
	\clearpage

	\tadOperacion{armarMultiPos}{estr}{multiconj(posicion)}{}
	\tadAxioma{armarMultiPos($e$)}{multiSignificados($e$.placasPos,claves($e$.placasPos)) $\cup$ \\
	multiSignificados($e$.hippies,claves($e$.hippies)) $\cup$ \\
	multiSignificados($e$.estudiantes,claves($e$.estudiantes)}

  ~

  \tadOperacion{multiSignificados}{dicc($\beta$,$\alpha$)/d, conj($\beta$)/c}{multiconj($\alpha$)}{($\forall cl: \beta$) $cl \in c \implies$ def?(cl,d)}
  \tadAxioma{multiSignificados($d$,$c$)}{
  \IF $\emptyset$?($c$) THEN $\emptyset$ 
  ELSE Ag(obtener(dameUno($c$),$d$),multiSignificados($d$,sinUno($c$)))
  FI}

  ~

  \tadOperacion{unionSignificados}{dicc($\beta$,$\alpha$)/d}{conj($\alpha$)}{}
  \tadAxioma{unionSignificados($d$)}{multiAConj(multiSignificados($d$,claves($d$)))}

  ~

  \tadOperacion{multiAConj}{multiconj($\alpha$)}{conj($\alpha$)}{}
  \tadAxioma{multiAConj($mc$)}{\IF $mc$ = $\emptyset$ THEN $\emptyset$ 
  ELSE Ag(dameUno($mc$),multiAConj(sinUno($mc$)))
  FI}

  ~

  \tadOperacion{secuenciarDic}{dicc($\beta$,$\alpha$)}{secu($\langle \beta, alpha \rangle$)}{}
  \tadAxioma{secuenciarDic($d$)}{secuenciarDicAux($d$,claves($d$)}

  ~

  \tadOperacion{secuenciarDicAux}{dicc($\beta$,$\alpha$)/d, conj($\beta$)/c}{secu($\langle \beta, \alpha \rangle$)}{($\forall cl: \beta$) $cl \in c \implies$ def?(cl,d)}
  \tadAxioma{secuenciarDicAux($d$,$c$)}{
  \IF $\emptyset$?($c$) THEN $<>$ 
  ELSE $\langle$dameUno($c$),obtener(dameUno($c$),$d$)$\rangle$ $\bullet$ secuenciarDicAux($d$,sinUno($c$))
  FI}

  ~

  \tadOperacion{ordenar$\pi_1$}{secu($\langle \beta $,$ \alpha \rangle$)}{secu($\langle \beta, \alpha \rangle$)}{}
  \tadAxioma{ordenar$\pi_1$($s$)}{
  \IF vacia?($s$) THEN $<>$ 
  ELSE min$\pi_1$($s$) $\bullet$ ordenar$\pi_1$(fin($s$))
  FI}

  ~

  \tadOperacion{min$\pi_1$}{secu($\langle \beta $,$ \alpha \rangle$)/s}{$\langle \beta, \alpha \rangle$}{$\neg$vacia?($s$)}
  \tadAxioma{min$\pi_1$($s$)}{minAux$\pi_1$($s$,prim($s$)}

  ~
  
  \tadOperacion{minAux$\pi_1$}{secu($\langle \beta $,$ \alpha \rangle$)/s, $\langle \beta $ , $\alpha \rangle$}{$\langle \beta, \alpha \rangle$}{$\neg$vacia?($s$)}
  \tadAxioma{minAux$\pi_1$($s$,$e$}{\IF long($s$) = 1 THEN 
  {\IF $\pi_1$(prim($s$)) < $\pi_1$($e$) THEN prim($s$) ELSE $e$ FI} 
  ELSE minAux$\pi_1$(fin($s$),
  {\IF $\pi_1$(prim($s$)) < $\pi_1$($e$) THEN prim($s$) ELSE $e$ FI}
  FI}

  ~
  
  \AbsFc[estr]{campusSeg}[e]
  {$cs$: campusSeg $/$ e.cp = campus(cs) $\land$ claves($e$.estudiantes) = estudiantes($cs$) \\ $\land$ claves($e$.hippies) = hippies($cs$) $\land$ claves($e$.agentes) = agentes(cs) $\land$ \\
  ($\forall a$: agente)(($a \in$ agentes($cs$) $\iff$ def?($a$,$e$.agentes)) $\yluego$ ($a \in$ agentes($cs$) $\impluego$ \\
  (obtener($a$,$e$.agentes).sanciones = cantSanciones($a$,$cs$) $\land$\\
  obtener($a$,$e$.agentes).premios = cantHippiesAtrapados($a$,$cs$) $\land$\\
  obtener($a$,$e$.agentes).posicion = posAgente($a$,$cs$)))) $\yluego$ \\
  ($\forall n:$ nombre)(def?($n$,$e$.hippies) $\impluego$ posEstudianteYHippie($n$,$cs$) = obtener($a$,$e$.hippies)) $\land$\\
  (def?($n$,$e$.estudiantes) $\impluego$ posEstudianteYHippie($n$,$cs$) = obtener($a$,$e$.estudiantes)) }


\begin{Algoritmos}

\begin{algorithm}[H]
\caption{iComenzarRastrillaje}

\begin{algorithmic}[1]
\Procedure{iComenzarRastrillaje}{\texttt{in} c : \texttt{campus}, \texttt{in} d : \texttt{diccLineal(placa,posicion)}} $\to res$ : estr
	\State $res$.cp $\gets$ Copiar($c$) \Comment $\Ogr$($h*a$)
	\State $res$.posiciones.mAg $\gets$ CrearMatriz(Filas($c$),Columnas($c$), datosAg) \Comment $\Ogr$($h*a$)
	\State $res$.posiciones.mH $\gets$ CrearMatriz(Filas($c$),Columnas($c$), datosAg) \Comment $\Ogr$($h*a$)
	\State $res$.posiciones.mE $\gets$ CrearMatriz(Filas($c$),Columnas($c$), datosAg) \Comment $\Ogr$($h*a$)
	
	\State $res$.hippies $\gets$ Vacio() \Comment $\Ogr(1)$
	\State $res$.estudiantes $\gets$ Vacio() \Comment $\Ogr(1)$
	\State $res$.sanciones $\gets$ Vacio() \Comment $\Ogr(1)$
	\State $res$.agentes $\gets$ Vacio() \Comment $\Ogr(1)$
	\State $res$.placasPos $\gets$ Vacio($\#$Claves($d$)) \Comment $\Ogr(1)$ %TODO, ver la complejidad
	\State itDicc(placa,posicion) $it$ $\gets$ CrearIt($d$) \Comment $\Ogr(1)$
	\State $res$.masVigilante $\gets$ datosAg \Comment (datosAg por defecto) $\Ogr(1)$ 
	\If{HaySiguiente($it$)} \Comment $\Ogr(1)$
		\State conj(placa) $cp$ $\gets$ Vacio() \Comment $\Ogr(1)$
		\State itDicc(nat,conj(placa)) $itdic$ $\gets$ Definir(0,$cp$,$res$.sanciones) \Comment $\Ogr(1)$
		\State bool $f$ $\gets$ true \Comment $\Ogr(1)$
		\While{HaySiguiente($it$)} \Comment $n$ * $\Ogr(1)$
	 	\State placa $pl$ $\gets$ SiguienteClave($it$) \Comment $\Ogr(1)$
	 	\State posicion $pos$ $\gets$ CrearPosicion(X(SiguienteSignificado($it$)),Y(SiguienteSignificado($it$))) \Comment $\Ogr(1)$
	 	\State itConj(placa) $itc$ $\gets$ Agregar($pl$,$cp$) \Comment $\Ogr(1)$
	 	\State datosAg $ag \gets \langle pl$,0,0,$pos$,$itdic$,$itc \rangle$ \Comment $\Ogr(1)$
	 	\State Definir($res$.placasPos,$pl$,$pos$) \Comment $\Ogr(n)$
	 	\State $res$.posiciones.mAg[X($pos$)][Y($pos$)] $\gets ag$ \Comment $\Ogr(1)$
	 	\If{($f$ || $res$.masVigilante.placa $\geq pl$)} \Comment $\Ogr(1)$
	 		\State $res$.masVigilante $\gets$ ag \Comment $\Ogr(1)$
	 		\State $f \gets$ false \Comment $\Ogr(1)$
	 	\EndIf
	 \EndWhile
	 \State Definir($res$.sanciones,0,$cp$) \Comment $\Ogr(1)$ %TODO: Ver la complejidad
	 \EndIf
\EndProcedure
\end{algorithmic}

\underline{Complejidad:} $\Ogr$($h*a + n^2$)\\
Donde 
\begin{itemize}
	\item $h$ es Filas($c$)
	\item $a$ es Columnas($c$)
	\item $n$ es $\#$Claves($d$)
\end{itemize}	
\underline{Justificacion:} Sumo los $\Ogr(1)$ continuos para aclarar un poco ($C$*$\Ogr(1)$ = $\Ogr(1)$)\\
$4*\Ogr(h*a) + \Ogr(1) + n * \Ogr(1) * \Ogr(n)$ = $\Ogr(h*a) + \Ogr(n)*\Ogr(n)$ = $\Ogr(h*a) + \Ogr(n^2)$ = $\Ogr(h*a + n^2)$
\end{algorithm}


\begin{algorithm}[H]
\caption{iCampus}

\begin{algorithmic}[1]
 \Procedure{iCampus}{\texttt{in} c : \texttt{estr}} $\to res$ : campus
  \State $res \gets$ $c$.cp \Comment $\Ogr(1)$
 \EndProcedure
\end{algorithmic}

 \underline{Complejidad:} $\Ogr(1)$
\end{algorithm}


\begin{algorithm}[H]
\caption{iEstudiantes}

\begin{algorithmic}[1]
 \Procedure{iEstudiantes}{\texttt{in} c : \texttt{estr}} $\to res$ : itdic(nombre,posicion)
  \State $res \gets$ CrearIt($c$.estudiantes) \Comment $\Ogr(1)$
 \EndProcedure
\end{algorithmic}
 \underline{Complejidad:} $\Ogr(1)$
\end{algorithm}


\begin{algorithm}[H]
\begin{algorithmic}[1]
	\Procedure{iHippies}{\texttt{in} c : \texttt{estr}} $\to res$ : itdic(nombre,posicion)
	\State $res \gets$ CrearIt($c$.hippies) \Comment $\Ogr(1)$
	\EndProcedure
\end{algorithmic}
\underline{Complejidad:} $\Ogr(1)$
\end{algorithm}


\begin{algorithm}[H]
\begin{algorithmic}[1]
	\Procedure{iAgentes}{\texttt{in} c : \texttt{estr}} $\to res$ : itdic(placa,datosAg)
	\State $res \gets$ CrearIt($c$.agentes) \Comment $\Ogr(1)$
	\EndProcedure
\end{algorithmic}
\underline{Complejidad:} $\Ogr(1)$
\end{algorithm}



\begin{algorithm}[H]
\caption{iPosEstudianteYHippie}

\begin{algorithmic}[1]
\Procedure{iPosEstudianteYHippie}{\texttt{in} c : \texttt{estr}, \texttt{in} n : \texttt{nombre}} $\to res$ : posicion
	\If{Definido($c$.hippies,$n$} \Comment $\Ogr(|h_m|)$
		\State $res \gets$ Obtener($n$,$c$.hippies) \Comment $\Ogr(|h_m|)$
	\Else
		\If{Definido($c$.estudiantes,$n$)} \Comment $\Ogr(|e_m|)$
			\State $res \gets$ Obtener($n$,$c$.estudiantes) \Comment $\Ogr(|e_m|)$
		\EndIf
	\EndIf	
 \EndProcedure
\end{algorithmic}

\underline{Complejidad:} $\Ogr(|n_m|)$ \\
\underline{Justificacion:} $h_m$ y $e_m$ son el hippie y estudiante con nombre mas largo, sea cual sea la rama del if a la que se entra la complejidad llega a ser 
$\Ogr(|n_m|)$ + $\Ogr(|n_m|)$ = $\Ogr(|n_m|)$ con $|n_m|$ = max($|h_m|$,$|e_m|$)
\end{algorithm}


\begin{algorithm}[H]
\caption{iCampus}
\begin{algorithmic}[1]
 \Procedure{iPosAgente}{\texttt{in} c : \texttt{estr}, \texttt{in} p : \texttt{placa}} $\to res$ : posicion
  \State $res \gets$ Obtener($c$.agentes, $pl$) \Comment $\Ogr(1)$ promedio
 \EndProcedure
\end{algorithmic}

 \underline{Complejidad:} $\Ogr(1)$ promedio
\end{algorithm}


\begin{algorithm}[H]
\caption{iCantSanciones}

\begin{algorithmic}[1]
\Procedure{iCantSanciones}{\texttt{in} c : \texttt{estr}, \texttt{in} pl : \texttt{placa}} $\to res$ : nat
	\State $res \gets$ Obtener($c$.agentes,$pl$).sanciones \Comment $\Ogr(1)$ promedio
\EndProcedure
\end{algorithmic}
\underline{Complejidad:} $\Ogr(1)$ promedio
\end{algorithm}


\begin{algorithm}[H]
\caption{iCantHippiesAtrapados}
\begin{algorithmic}[1]
\Procedure{iCantHippiesAtrapados}{\texttt{in} c : \texttt{estr}, \texttt{in} pl : \texttt{placa}} $\to res$ : nat
	\State $res \gets$ Obtener($c$.agentes,$pl$).premios \Comment $\Ogr(1)$ promedio
\EndProcedure
\end{algorithmic}
\underline{Complejidad:} $\Ogr(1)$ promedio
\end{algorithm}


\begin{algorithm}[H]
\caption{iMasVigilante}

\begin{algorithmic}[1]
\Procedure{iMasVigilante}{\texttt{in} c : \texttt{estr}} $\to res$ : placa
	\State $res$ $\gets$ $c$.masVigilante.pl \Comment $\Ogr(1)$
\EndProcedure
\end{algorithmic}

\underline{Complejidad:} $\Ogr(1)$
\end{algorithm}

\end{Algoritmos}

\end{Representacion}




\end{document}


\end{document}
