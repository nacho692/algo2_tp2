\documentclass[a4paper,10pt, nofootinbib]{article}
\usepackage[width=15.5cm, left=3cm, top=2.5cm, right=1cm, left=2cm, height= 24.5cm]{geometry}
\usepackage[spanish]{babel}
\usepackage[utf8]{inputenc}
\usepackage[T1]{fontenc}
\usepackage{xspace}
\usepackage{xargs}
\usepackage{ifthen}
\usepackage{caratula}
\usepackage{fancyhdr}
\usepackage{aed2-tad,aed2-symb,aed2-itef}
\usepackage[bottom]{footmisc}
\usepackage{algorithm}
\usepackage[noend]{algpseudocode}
\usepackage{modulos}
\usepackage[colorlinks=true, linkcolor=blue]{hyperref}
\usepackage{calc}
\usepackage{lastpage}
\usepackage{titlesec}

\newcommand{\tituloModulo}[1]{
  \vspace*{1ex}\par\noindent\textbf{\large #1}\par
}
\let\NombreFuncion=\textsc
\let\TipoVariable=\texttt
\let\ModificadorArgumento=\textbf
\newcommand{\res}{$res$\xspace}
\newcommand{\tab}{\hspace*{7mm}}
\newcommand{\Ogr}{\mathcal{O}}
\newcommand{\footnoteMio}[2]{\textsuperscript{#1} \lfoot{\footnotesize \parbox{16cm}{\textsuperscript{#1}#2}}}
\newcommandx{\TipoFuncion}[3]{%
  \NombreFuncion{#1}(#2) \ifx#3\empty\else $\to$ \res\,: \TipoVariable{#3}\fi%
}
\newcommand{\In}[2]{\ModificadorArgumento{in} \ensuremath{#1}\,: \TipoVariable{#2}\xspace}
\newcommand{\Out}[2]{\ModificadorArgumento{out} \ensuremath{#1}\,: \TipoVariable{#2}\xspace}
\newcommand{\Inout}[2]{\ModificadorArgumento{in/out} \ensuremath{#1}\,: \TipoVariable{#2}\xspace}
\newcommand{\Aplicar}[2]{\NombreFuncion{#1}(#2)}

\newcommand{\DRef}{\ensuremath{\rightarrow}}

\titleformat*{\section}{\Large\bfseries}
\titleformat*{\subsection}{\Large\bfseries}
\titleformat*{\subsubsection}{\Large}


% Acomodo fancyhdr.
\pagestyle{fancy}
\thispagestyle{fancy}
\lhead{Algoritmos y Estructuras de Datos II}
\rhead{$2^\mathrm{do}$ cuatrimestre de 2015}
\cfoot{\thepage /\pageref{LastPage}}
\renewcommand{\footrulewidth}{0.4pt}
\setlength{\headheight}{13pt}


%Informacion para la caratula
\materia{Algoritmos y Estructuras de Datos II}
\titulo{\Large Trabajo Práctico Nº2}
\grupo{Grupo 19}
\integrante{Basso, Juan Cruz}{627/14}{jcbasso95@gmail.com}
\integrante{Bohe, Brian}{706/14}{brianbohe@gmail.com}
\integrante{Figari, Francisco}{719/14}{francisco.figari@hotmail.com}
\integrante{Mariotti, Ignacio}{651/14}
{nacho692@gmail.com}
\def\cuatrimestre{2}

\begin{document}

%Pagina de titulo e indice
\thispagestyle{empty}

\maketitle

\tableofcontents
\newpage


\clearpage
\section{Nociones previas}
\begin{itemize}
\item Para las complejidades se utilizan las definiciones del contexto de uso y se agregan algunas otras
\begin{itemize}
	\item $|n_m|$ es el nombre mas largo de estudiantes y hippies
	\item $|h_m|$ es el nombre mas largo de hippies
	\item $|e_m|$ es el nombre mas largo de estudiantes
	\item $N_a$ es la cantidad de agentes
	\item $N_h$ es la cantidad de hippies
	\item $N_e$ es la cantidad de estudiantes
\end{itemize}
\item Algunas consideraciones de tipos son
\begin{itemize}
	\item $direccion$ es un enumerado {izq,der,arriba,abajo}
	\item $placa$ es nat
	\item $agente$ es nat
	\item $nombre$ es string
\end{itemize}
\item Al no haber un modulo de arreglos explicito en modulos basicos, pueden haber diferencias triviales en los diferentes algoritmos

\end{itemize}

\section{Modulos}
\subsection{Modulo Campus Seguro}

Se sigue el siguiente orden para las funciones que proveen los movimientos de estudiantes, agentes o hippies.
\begin{enumerate}
	\item Apariciones y Movimientos
	\item Enhippizacion (Estudiantes en hippies)
	\item Educacion (Hippies en estudiantes)
	\item Premios de capturas de hippies
	\item Captura de hippies
	\item Sanciones relacionadas con el evento
\end{enumerate}

\begin{Interfaz}

	\textbf{se explica con}: \tadNombre{Campus Seguro}.

	\textbf{géneros}: \TipoVariable{campusSeg}.

	\tituloModulo{Operaciones básicas de campusSeg}

	\InterfazFuncion{Campus}{\In{c}{campusSeg}}{campus}
	{$res \igobs $campus($c$)}
	[$\Ogr(1)$]
	[Devuelve el campus de Campus Seguro]
	[El campus se devuelve por referencia]

	\InterfazFuncion{Estudiantes}{\In{c}{campusSeg}}{itClavesDiccN(nombre,posicion)}
	{esPermutacion(Siguientes($res$),secuenciarConj(estudiantes($d$))}
	[$\Ogr(1)$]
	[Devuelve Iterador no modificable de todos los estudiantes actualmente en el campus]
	[Si hay un cambio en los estudiantes, el iterador puede quedar invalidado]
  
	\InterfazFuncion{Hippies}{\In{c}{campusSeg}}{itClavesDiccN(nombre,posicion)}
	{esPermutacion(Siguientes($res$),secuenciarConj(hippies($d$))}
	[$\Ogr(1)$]
	[Devuelve Iterador no modificable de todos los hippies actualmente en el campus]
	[Si hay un cambio en los hippies, el iterador puede quedar invalidado]

	\InterfazFuncion{Agentes}{\In{c}{campusSeg}}{itClavesDiccLog(nat)}
	{esPermutacion(Siguientes($res$),secuenciarConj(agentes($c$)))}
	[$\Ogr(1)$]
	[Devuelve Iterador no modificable de todos los agentes actualmente en el campus]
	[Si hay un cambio en el diccionario, se va a ver reflejado en el iterador]

	\InterfazFuncion{PosEstudianteYHippie}{\In{c}{campusSeg}, \In{n}{nombre}}{posicion}
	[$n \in$ estudiantes($c$)$\cup$hippies($c$)]
	{$res \igobs$posEstudianteYHippie($n$,$c$)}
	[$\Ogr(|n_m|)$ con siendo $|n_m|$ la longitud del nombre más largo de hippies y estudiantes]
	[Devuelve la posicion del estudiante o hippie]

	\InterfazFuncion{PosAgente}{\In{c}{campusSeg}, \In{p}{placa}}{posicion}
	[$pl \in$ agentes($c$)]
	{$res \igobs$ posAgente($pl$,$c$)}
	[$\Ogr(1)$ en caso promedio]
	[Devuelve la posicion del agente]

	\InterfazFuncion{CantSanciones}{\In{c}{campusSeg}, \In{pl}{placa}}{nat}
	[$pl \in$ agentes($c$)]
	{$res \igobs$ cantSanciones($pl$,$c$)}
	[$\Ogr(1)$ en caso promedio]
	[Devuelve la cantidad de sanciones del agente]
	
	\InterfazFuncion{CantHippiesAtrapados}{\In{c}{campusSeg}, \In{pl}{placa}}{nat}
	[$pl \in$ agentes($c$)]
	{$res \igobs$ cantHippiesAtrapados($pl$,$c$)}
	[$\Ogr(1)$ en caso promedio]
	[Devuelve la cantidad de hippies que atrapo el agente]

	\InterfazFuncion{ComenzarRastrillaje}{\In{c}{campus}, \In{d}{dicc(agente,posicion)}}{campusSeg}%
	[($\forall a$: agente)def?($a$,$d$)$\impluego$ posValida?(obtener($a$,$d$))$\land \neg$ocupada?(obtener($a$,$d$),$c$)))\\
  $\land$($\forall a_0,a_1$: agente)((def?($a_0$,$d$)$\land$def?($a_1$,$d$)$\land a_0 \neq a_1$)$\impluego$obtener($a_0$,$d$)$\neq$obtener($a_1$,$d$))]
	{$res \igobs $comenzarRastrillaje($c$,$d$)}%
	[$\Ogr$($h*a + n^2$) + $\Ogr(n)$ prom]
	[Inicializa toda la estructura en funcion de los datos de entrada]

	\InterfazFuncion{IngresarEstudiante}{\Inout{c}{campusSeg}, \In{n}{nombre}, \In{p}{posicion}}{}
	[$c$ = $c_0 \land n \notin$ (estudiantes($c$)$\cup$hippies($c$))$\land$esIngreso($p$,campus($c$))$\land \neg$ estaOcupada?($p$,$c$)]
	{$c$ = ingresarEstudiante($n$,$p$,$c_0$)}
	[$\Ogr(|n_m|)$]
	
	\InterfazFuncion{MasVigilante}{\In{c}{campusSeg}}{placa}
	[$\#Agentes(c) \geq$ 1]
	{$res \igobs$masVigilante($c$)}
	[$\Ogr(1)$]


\end{Interfaz}

\begin{Representacion}
  
  \tituloModulo{Representación de campusSeg}

  \begin{Estructura}{campusSeg}[estr]
    \begin{Tupla}[estr]
		\tupItem{hippies}{diccNombres(nombre,posicion)}%
		\tupItem{estudiantes}{diccNombres(nombre,posicion)}\\
		\tupItem{agentes}{diccLog(nat,datosAg)}\\
		\tupItem{sanciones}{diccSanc(nat,conj(placa))}\\
		\tupItem{agentesRapido}{diccH(nat,itDiccLog(nat,posicion))}\\
		\tupItem{pos}{posiciones}
		\tupItem{masVigilante}{tupla(pl: placa, premios: nat}
		\tupItem{cp}{campus}
    \end{Tupla}
    
	\begin{Tupla}[posiciones]
		\tupItem{mAg}{matriz(tupla(def: bool,datos: itDiccLog(nat,posicion))}\\
		\tupItem{mH}{matriz(tupla(def: bool,datos: itDiccN(nombre, posicion))}\\
		\tupItem{mE}{matriz(tupla(def: bool,datos: itDiccN(nombre, posicion))}
	\end{Tupla}
	
    \begin{Tupla}[datosAg]
      \tupItem{pl}{placa}
      \tupItem{sanciones}{nat}
      \tupItem{premios}{nat}
      \tupItem{pos}{posicion}%
      \tupItem{kSanc}{itDiccS(nat, conj(placa)}%
      \tupItem{mismasSanc}{itconj(placa)}%
    \end{Tupla}
  \end{Estructura}

  \tituloModulo{Aclaraciones}
	Dada la complejidad de la estructura, el invariante puede ser dificil de seguir, por lo que a continuacion se encuentra detallado de manera informal las lineas que se siguieron al hacer el rep.
	
	\begin{enumerate}
		\item Las claves de hippies y estudiantes intersecan vacio.
		\item No hay posiciones iguales en estudiantes, hippies y agentes y todas se encuentran en rango.
		\item Las placas de agentes y agentesRapido son iguales y, a igual placa, el siguiente del iterador de agentesRapido es el agente en agentes y la secuencia subyacente del iterador es permutacion de la secuencia de tuplas del diccionario de agentes.
		\item Las claves de sanciones y las diferentes sanciones de agentes forman el mismo conjunto. Los conjuntos de placas de la clave i de sanciones tienen todas las placas cuyos agentes tengan i sanciones.
		\item Para todos los datosAg de agentes, la secuencia subyacente del iterador kSanc es igual a una secuencia ordenada de sanciones y la siguiente clave del iterador es la cantidad de sanciones del datosAg.
		\item Para todos los datosAg de agentes, la secuencia subyacente del iterador mismasSanc es permutacion del conjunto de placas del diccionario sanciones donde se encuentra su propia placa, y el siguiente del iterador es su misma placa.
		\item filas+1 y columnas+1 de cp (campus) es igual al alto y ancho de las matrices de pos.
		\item mAg, el booleano es true si y solo si hay un agente en placasPos en esa posicion.\\
		Si es true, la siguiente clave del iterador es el agente en esa posicion y la secuencia subyacente es permutacion de los agentes en agentes.
		\item mH, el booleano es true si y solo si hay un hippie en esa posicion.\\
		Si es true, la siguiente clave del iterador es el hippie en esa posicion y la secuencia subyacente es permutacion de los hippies en hippies.
		\item mE, el booleano es true si y solo si hay un estudiante en esa posicion.\\
		Si es true,la siguiente clave del iterador es el estudiante en esa posicion y la secuencia subyacente es permutacion de los estudiantes en estudiantes.
		\item El mas vigilante es el de mayor sanciones de los agentes y, entre ellos, el de menor placa.
	\end{enumerate}
	
	Para facilitar la lectura se renombra las matrices $e$.pos.mXX a mXX
  \Rep[estr]{
	\begin{enumerate}
		\item claves($e$.hippies) $\cap$ claves(e.estudiantes) = $\emptyset$
		\item $\land$ ($\forall p$: posicion)($\#$($p$,armarMultiPos($e$)) $\geq$ 1 $\implies$ \\
		$\#$($p$,armarMultiPos($e$)) = 1 $\land$ posValida($p$,$e$.cp) $\yluego$ $\neg$ocupada?($p$,$e$.cp))
		\item $\land$ (claves($e$.agentes) $=$ claves($e$.agentesRapido) $\yluego$ ($\forall p$: placa) (def?($p$,e.agentes) $\impluego$ \\
		$\pi_1$(Siguiente(obtener($p$,$e$.agentesRapido))) = obtener($p$,$e$.agentes))\\
		$\land$ esPermutacion(SecuSuby(obtener($p$,$e$.agentesRapido)),secuenciarDic($e$.agentes)))
		\item $\land$ unionSignificados($e$.sanciones) = claves($e$.agentes)\\
		$\land$ ($\forall n$: nat) def?($n$,$e$.sanciones) $\impluego$ \\
		($\neg$(obtener($n$,$e$.sanciones) = $\emptyset$) $\land$ ($\forall p$: placa) ($p \in$ obtener($n$,$e$.sanciones) $\implies$ \\
		def?($p$,$e$.agentes) $\yluego$ (obtener($p$,$e$.agentes).pl = p $\land$ obtener($p$,$e$.agentes).sanciones = n))
		\item $\land$ ($\forall p$: placa) def?($p$,$e$.agentes) $\impluego$ \\
		SecuSuby(obtener($p$,$e$.agentes).kSanc) = ordenar$\pi_1$(secuenciarDic($e$.sanciones)) \\
		$\land$ $pi_1$(Siguiente(obtener($p$,$e$.agentes).kSanc)) = obtener($p$,$e$.agentes).pl
		\item $\land$ ($\forall p$: placa) (def?($p$,$e$.agentes) $\yluego$ def?(obtener($p$,$e$.agentes).sanciones,$e$.sanciones) $\impluego$ \\
		esPermutacion(SecuSuby(obtener($p$,$e$.agentes).mismasSanc),\\
		secuenciarConj(obtener(obtener($p$,$e$.agentes).sanciones,$e$.sanciones)))\\
		$\land$ siguiente(obtener($p$,$e$.agentes).mismasSanc) = $p$
		\item $\land$ Ancho(mAg) = columnas($e$.campus) + 1 $\land$ Alto(mAg) = filas($e$.campus) + 1 \\
		$\land$ Ancho(mH) = Ancho(mAg) $\land$ Alto(mH) = Alto(mAg)\\
		$\land$ Ancho(mE) = Ancho(mH) $\land$ Alto(mE) = Alto(mH)
		\item $\land$ ($\forall i$: nat) ($\forall j$: nat) 0 $< i <$ Ancho(mAg) $\land$ 0 $< j <$ Alto(mAg) $\impluego$ \\
		($pi_1$(Valor(mAg,$i$,$j$)) $\iff$ \\
		($\exists p$: placa) def?($p$,$e$.agentes) $\yluego$ (obtener($p$,$e$.agentes).pos = $\langle i$, $j \rangle$ $\land$ \\ 
		esPermutacion(SecuSuby($\pi_2$(Valor(mAg,$i$,$j$))), secuenciarDic(agentes)) \\
		$\land$ $\pi_1$(Siguiente($\pi_2$(Valor(mAg,$i$,$j$))) = obtener($p$,$e$.agentes).pl)
		\item $\land$ ($\forall i$: nat) ($\forall j$: nat) 0 $< i <$ Ancho(mH) $\land$ 0 $< j <$ Alto(mH) $\impluego$ \\
		($\pi_1$(Valor(mH,$i$,$j$)) $\iff$ ($\exists n$: nombre)(def?($n$,$e$.hippies) $\yluego$ \\
		obtener($n$,$e$.hippies) = $\langle i$, $j \rangle$ $\land$ $\pi_1$(Siguiente($\pi_2$(Valor(mH,$i$,$j$)))) = $n$ $\land$ \\
		esPermutacion(SecuSuby($\pi_2$(Valor(mH,$i$,$j$)),secuenciarDic($e$.hippies)))
		\item $\land$ ($\forall i$: nat) ($\forall j$: nat) 0 $< i <$ Ancho(mE) $\land$ 0 $< j <$ Alto(mE) $\impluego$ \\
		($\pi_1$(Valor(mE,$i$,$j$)) $\iff$ ($\exists n$: nombre)(def?($n$,$e$.estudiantes) $\yluego$ \\
		obtener($n$,$e$.estudiantes) = $\langle i$, $j \rangle$ $\land$ $\pi_1$(Siguiente($\pi_2$(Valor(mE,$i$,$j$)))) = $n$ $\land$ \\
		esPermutacion(SecuSuby($\pi_2$(Valor(mE,$i$,$j$)),secuenciarDic($e$.estudiantes)))
		\item $\#$claves($e$.agentes) $\geq$ 1 $\implies$ ($\exists p$: placa) def?($p$,$e$.agentes) $\yluego$ \\
		(obtener($p$,$e$.agentes).pl = $e$.masVigilante.pl $\land$ \\
		obtener($p$,$e$.agentes).premios = $e$.masVigilante.premios $\land$ \\
		(($\forall p':$ placa) def?($p'$,$e$.agentes) $\land p \neq p' \impluego$\\
		(obtener($p$,$e$.agentes).premios $\geq$ obtener($p'$.$e$.agentes).premios $\land$ \\
		(obtener($p$,$e$.agentes).premios = obtener($p'$,$e$.agentes) $\implies$ $p < p'$))))

	\end{enumerate}
  }\mbox{}
	
	\clearpage

	\tadOperacion{armarMultiPos}{estr}{multiconj(posicion)}{}
	\tadAxioma{armarMultiPos($e$)}{multiSignificados(soloPosAgentes($e$.agentes,claves($e$.agentes)),claves($e$.placasPos)) $\cup$ \\
	multiSignificados($e$.hippies,claves($e$.hippies)) $\cup$ \\
	multiSignificados($e$.estudiantes,claves($e$.estudiantes)}

  
  ~


  \tadOperacion{soloPosAgentes}{dicc(nat,datosAg)/d, conj($\beta$)/c}{dicc(nat,posicion)}{$c \subseteq claves(d)$}
  \tadAxioma{soloPosAgentes($d$,$c$)}{
  \IF $\emptyset$?($c$) THEN Vacio() 
  ELSE definir(dameUno($c$), obtener(dameUno($c$),$d$).pos,soloPosAgentes($d$,sinUno($c$)))
  FI}

  ~
  
  
  \tadOperacion{multiSignificados}{dicc($\beta$,$\alpha$)/d, conj($\beta$)/c}{multiconj($\alpha$)}{($\forall cl: \beta$) $cl \in c \implies$ def?(cl,d)}
  \tadAxioma{multiSignificados($d$,$c$)}{
  \IF $\emptyset$?($c$) THEN $\emptyset$ 
  ELSE Ag(obtener(dameUno($c$),$d$),multiSignificados($d$,sinUno($c$)))
  FI}

  ~

  \tadOperacion{unionSignificados}{dicc($\beta$,$\alpha$)/d}{conj($\alpha$)}{}
  \tadAxioma{unionSignificados($d$)}{multiAConj(multiSignificados($d$,claves($d$)))}

  ~

  \tadOperacion{multiAConj}{multiconj($\alpha$)}{conj($\alpha$)}{}
  \tadAxioma{multiAConj($mc$)}{\IF $mc$ = $\emptyset$ THEN $\emptyset$ 
  ELSE Ag(dameUno($mc$),multiAConj(sinUno($mc$)))
  FI}

  ~

  \tadOperacion{secuenciarDic}{dicc($\beta$,$\alpha$)}{secu($\langle \beta, alpha \rangle$)}{}
  \tadAxioma{secuenciarDic($d$)}{secuenciarDicAux($d$,claves($d$)}

  ~

  \tadOperacion{secuenciarDicAux}{dicc($\beta$,$\alpha$)/d, conj($\beta$)/c}{secu($\langle \beta, \alpha \rangle$)}{($\forall cl: \beta$) $cl \in c \implies$ def?(cl,d)}
  \tadAxioma{secuenciarDicAux($d$,$c$)}{
  \IF $\emptyset$?($c$) THEN $<>$ 
  ELSE $\langle$dameUno($c$),obtener(dameUno($c$),$d$)$\rangle$ $\bullet$ secuenciarDicAux($d$,sinUno($c$))
  FI}

  ~

  \tadOperacion{ordenar$\pi_1$}{secu($\langle \beta $,$ \alpha \rangle$)}{secu($\langle \beta, \alpha \rangle$)}{}
  \tadAxioma{ordenar$\pi_1$($s$)}{
  \IF vacia?($s$) THEN $<>$ 
  ELSE min$\pi_1$($s$) $\bullet$ ordenar$\pi_1$(fin($s$))
  FI}

  ~

  \tadOperacion{min$\pi_1$}{secu($\langle \beta $,$ \alpha \rangle$)/s}{$\langle \beta, \alpha \rangle$}{$\neg$vacia?($s$)}
  \tadAxioma{min$\pi_1$($s$)}{minAux$\pi_1$($s$,prim($s$)}

  ~
  
  \tadOperacion{minAux$\pi_1$}{secu($\langle \beta $,$ \alpha \rangle$)/s, $\langle \beta $ , $\alpha \rangle$}{$\langle \beta, \alpha \rangle$}{$\neg$vacia?($s$)}
  \tadAxioma{minAux$\pi_1$($s$,$e$}{\IF long($s$) = 1 THEN 
  {\IF $\pi_1$(prim($s$)) < $\pi_1$($e$) THEN prim($s$) ELSE $e$ FI} 
  ELSE minAux$\pi_1$(fin($s$),
  {\IF $\pi_1$(prim($s$)) < $\pi_1$($e$) THEN prim($s$) ELSE $e$ FI}
  FI}

  ~
  
  \AbsFc[estr]{campusSeg}[e]
  {$cs$: campusSeg $/$ e.cp = campus(cs) $\land$ claves($e$.estudiantes) = estudiantes($cs$) \\ $\land$ claves($e$.hippies) = hippies($cs$) $\land$ claves($e$.agentes) = agentes(cs) $\land$ \\
  ($\forall a$: agente)(($a \in$ agentes($cs$) $\iff$ def?($a$,$e$.agentes)) $\yluego$ ($a \in$ agentes($cs$) $\impluego$ \\
  (obtener($a$,$e$.agentes).sanciones = cantSanciones($a$,$cs$) $\land$\\
  obtener($a$,$e$.agentes).premios = cantHippiesAtrapados($a$,$cs$) $\land$\\
  obtener($a$,$e$.agentes).posicion = posAgente($a$,$cs$)))) $\yluego$ \\
  ($\forall n:$ nombre)(def?($n$,$e$.hippies) $\impluego$ posEstudianteYHippie($n$,$cs$) = obtener($a$,$e$.hippies)) $\land$\\
  (def?($n$,$e$.estudiantes) $\impluego$ posEstudianteYHippie($n$,$cs$) = obtener($a$,$e$.estudiantes)) }


\end{Representacion}


\begin{Algoritmos}

\begin{algorithm}[H]
\caption{Campus}

\begin{algorithmic}[1]
 \Procedure{iCampus}{\texttt{in} c : \texttt{estr}} $\to res$ : campus
  \State $res \gets$ $c$.cp \Comment $\Ogr(1)$
 \EndProcedure
\end{algorithmic}

 \underline{Complejidad:} $\Ogr(1)$
\end{algorithm}


\begin{algorithm}[H]
\caption{Estudiantes}

\begin{algorithmic}[1]
 \Procedure{iEstudiantes}{\texttt{in} c : \texttt{estr}} $\to res$ : itClavesDiccN(string)
  \State $res \gets$ CrearIt($c$.estudiantes) \Comment $\Ogr(1)$
 \EndProcedure
\end{algorithmic}
 \underline{Complejidad:} $\Ogr(1)$
\end{algorithm}


\begin{algorithm}[H]
\caption{Hippies}
\begin{algorithmic}[1]
	\Procedure{iHippies}{\texttt{in} c : \texttt{estr}} $\to res$ : itClavesDiccN(string)
	\State $res \gets$ CrearIt($c$.hippies) \Comment $\Ogr(1)$
	\EndProcedure
\end{algorithmic}
\underline{Complejidad:} $\Ogr(1)$
\end{algorithm}


\begin{algorithm}[H]
\caption{Agentes}
\begin{algorithmic}[1]
	\Procedure{iAgentes}{\texttt{in} c : \texttt{estr}} $\to res$ : itClavesDiccLog(nat)
	\State $res \gets$ CrearIt($c$.agentes) \Comment $\Ogr(1)$
	\EndProcedure
\end{algorithmic}
\underline{Complejidad:} $\Ogr(1)$
\end{algorithm}



\begin{algorithm}[H]
\caption{Posicion de Estudiante Y Hippie}

\begin{algorithmic}[1]
\Procedure{iPosEstudianteYHippie}{\texttt{in} c : \texttt{estr}, \texttt{in} n : \texttt{nombre}} $\to res$ : posicion
	\If{Definido($c$.hippies,$n$} \Comment $\Ogr(|h_m|)$
		\State $res \gets$ Obtener($n$,$c$.hippies) \Comment $\Ogr(|h_m|)$
	\Else
		\If{Definido($c$.estudiantes,$n$)} \Comment $\Ogr(|e_m|)$
			\State $res \gets$ Obtener($n$,$c$.estudiantes) \Comment $\Ogr(|e_m|)$
		\EndIf
	\EndIf	
 \EndProcedure
\end{algorithmic}

\underline{Complejidad:} $\Ogr(|n_m|)$ \\
\underline{Justificacion:} $h_m$ y $e_m$ son el hippie y estudiante con nombre mas largo, sea cual sea la rama del if a la que se entra la complejidad llega a ser 
$\Ogr(|n_m|)$ + $\Ogr(|n_m|)$ = $\Ogr(|n_m|)$ con $|n_m|$ = max($|h_m|$,$|e_m|$)
\end{algorithm}


\begin{algorithm}[H]
\caption{Posicion de Agente}
\begin{algorithmic}[1]
 \Procedure{iPosAgente}{\texttt{in} c : \texttt{estr}, \texttt{in} p : \texttt{placa}} $\to res$ : posicion
  \State $res \gets$ Obtener($c$.agentes, $pl$) \Comment $\Ogr(1)$ promedio
 \EndProcedure
\end{algorithmic}

 \underline{Complejidad:} $\Ogr(1)$ promedio
\end{algorithm}


\begin{algorithm}[H]
\caption{Cantidad de Sanciones}

\begin{algorithmic}[1]
\Procedure{iCantSanciones}{\texttt{in} c : \texttt{estr}, \texttt{in} pl : \texttt{placa}} $\to res$ : nat
	\State $res \gets$ Obtener($c$.agentes,$pl$).sanciones \Comment $\Ogr(1)$ promedio
\EndProcedure
\end{algorithmic}
\underline{Complejidad:} $\Ogr(1)$ promedio
\end{algorithm}


\begin{algorithm}[H]
\caption{Cantidad de Hippies Atrapados}
\begin{algorithmic}[1]
\Procedure{iCantHippiesAtrapados}{\texttt{in} c : \texttt{estr}, \texttt{in} pl : \texttt{placa}} $\to res$ : nat
	\State $res \gets$ Obtener($c$.agentes,$pl$).premios \Comment $\Ogr(1)$ promedio
\EndProcedure
\end{algorithmic}
\underline{Complejidad:} $\Ogr(1)$ promedio
\end{algorithm}


\begin{algorithm}[H]
\caption{Comenzar Rastrillaje}

\begin{algorithmic}[1]
\Procedure{iComenzarRastrillaje}{\texttt{in} c : \texttt{campus}, \texttt{in} d : \texttt{diccLineal(placa,posicion)}} $\to res$ : estr
	\State $res$.cp $\gets$ Copiar($c$) \Comment $\Ogr$($h*a$)
	\State $res$.posiciones.mAg $\gets$ CrearMatriz(Filas($c$),Columnas($c$), CrearItDiccLog()) \Comment $\Ogr$($h*a$)
	\State $res$.posiciones.mH $\gets$ CrearMatriz(Filas($c$),Columnas($c$), CrearItDiccN()) \Comment $\Ogr$($h*a$)
	\State $res$.posiciones.mE $\gets$ CrearMatriz(Filas($c$),Columnas($c$), CrearItDiccN()) \Comment $\Ogr$($h*a$)
	
	\State $res$.hippies $\gets$ Vacio() \Comment $\Ogr(1)$
	\State $res$.estudiantes $\gets$ Vacio() \Comment $\Ogr(1)$
	\State $res$.sanciones $\gets$ Vacio() \Comment $\Ogr(1)$
	\State $res$.agentesRapido $\gets$ Vacio($\#$Claves($d$)) \Comment $\Ogr(1)$
	\State $res$.agentes $\gets$ Vacio($\#$Claves($d$)) \Comment $\Ogr(1)$ %TODO, ver la complejidad
	\State itDicc(placa,posicion) $it$ $\gets$ CrearIt($d$) \Comment $\Ogr(1)$
	\If{HaySiguiente($it$)} \Comment $\Ogr(1)$
		\State conj(placa) $cp$ $\gets$ Vacio() \Comment $\Ogr(1)$
		\State itDicc(nat,conj(placa)) $itdic$ $\gets$ Definir(0,$cp$,$res$.sanciones) \Comment $\Ogr(1)$
		\State bool $f$ $\gets$ true \Comment $\Ogr(1)$
		\While{HaySiguiente($it$)} \Comment $n$ * $\Ogr(1)$
	 	\State placa $pl$ $\gets$ SiguienteClave($it$) \Comment $\Ogr(1)$
	 	\State posicion $pos$ $\gets$ CrearPosicion(X(SiguienteSignificado($it$)),Y(SiguienteSignificado($it$))) \Comment $\Ogr(1)$
	 	\State itConj(placa) $itc$ $\gets$ AgregarRapido($pl$,Obtener($res$.sanciones,0)) \Comment $\Ogr(n) + \Ogr(1)$
	 	\State datosAg $ag \gets \langle pl$,0,0,$pos$,$itdic$,$itc \rangle$ \Comment $\Ogr(1)$
	 	\State itDicLog(nat,datosAg) $itdl \gets$ Definir($res$.agentes,$pl$,$ag$) \Comment $\Ogr(n)$ %TODO: Definir lento?
	 	\State $res$.posiciones.mAg[X($pos$)][Y($pos$)] $\gets itdl$ \Comment $\Ogr(1)$
	 	\State Definir($res$.agentesRapido,$pl$,$itdl$) \Comment $\Ogr(1)$ promedio
	 	\If{($f$ || $res$.masVigilante.placa $\geq pl$)} \Comment $\Ogr(1)$
	 		\State $res$.masVigilante.pl $\gets$ ag.pl \Comment $\Ogr(1)$
	 		\State $res$.masVigilante.premios $\gets$ ag.premios \Comment $\Ogr(1)$
	 		\State $f \gets$ false \Comment $\Ogr(1)$
	 	\EndIf
	 \EndWhile
	 \EndIf
\EndProcedure
\end{algorithmic}

\underline{Complejidad:} $\Ogr(h*a + n^2) + \Ogr(n)$ prom\\
Donde 
\begin{itemize}
	\item $h$ es Filas($c$)
	\item $a$ es Columnas($c$)
	\item $n$ es $\#$Claves($d$)
\end{itemize}	
\underline{Justificacion:} Sumo los $\Ogr(1)$ continuos para aclarar un poco ($C$*$\Ogr(1)$ = $\Ogr(1)$)\\
$4*\Ogr(h*a) + \Ogr(1) + n * \Ogr(1) * (\Ogr(n) + \Ogr(n) + \Ogr(1)$ prom = $\Ogr(h*a) + \Ogr(n)*(\Ogr(n) + \Ogr(1)$ prom = $\Ogr(h*a) + \Ogr(n^2) + \Ogr(n)$ prom = $\Ogr(h*a + n^2) + \Ogr(n)$ prom
\end{algorithm}


\begin{algorithm}[H]
\caption{Ingresar Estudiante}
\begin{algorithmic}[1]
\Procedure{iIngresarEstudiante}{\texttt{in/out} c : \texttt{estr}, \texttt{in} n : \texttt{nombre}, \texttt{in} p : \texttt{posicion}}
	\State $c$.pos.mE[X($p$)][Y($p$)] $\gets$ $\langle$true, Definir($c$.estudiantes,$n$,$p$)$\rangle$ \Comment $\Ogr(|e_m|)$
	\State enhippizar($c$, $p$) \Comment $\Ogr(|h_m|)$
	\State estudiantizar($c$, $p$) \Comment $\Ogr(|e_m|)$
	\State premiar($c$,$p$) \Comment $\Ogr(1)$
	\State capturar($c$,$p$) \Comment $\Ogr(|h_m|)$
	\State sancionar($c$,$p$) \Comment $\Ogr(1)$
\EndProcedure
\end{algorithmic}
\underline{Complejidad:} $\Ogr(|n_m|)$\\
\underline{Justificacion:} $\Ogr(|e_m|) + \Ogr(|h_m|) + \Ogr(|e_m|) + \Ogr(|h_m|)$ =  $\Ogr(2|e_m|) + \Ogr(|2h_m|)$ = $\Ogr(2max(|e_m|,|h_m|))$ = $\Ogr(|n_m|)$
\end{algorithm}


\begin{algorithm}[H]
\caption{Ingresar Hippie}
\begin{algorithmic}[1]
\Procedure{iIngresarHippie}{\texttt{in/out} c : \texttt{estr}, \texttt{in} n : \texttt{nombre}, \texttt{in} p : \texttt{posicion}}
	\State $c$.pos.mH[X($p$)][Y($p$)] $\gets$ $\langle$true, Definir($c$.hippies,$n$,$p$) $\rangle$ \Comment $\Ogr(1)$
	\State enhippizar($c$, $p$) \Comment $\Ogr(|h_m|)$
	\State premiar($c$,$p$) \Comment $\Ogr(1)$
	\State capturar($c$,$p$) \Comment $\Ogr(1)$
	\State sancionar($c$,$p$) \Comment $\Ogr(1)$
\EndProcedure
\end{algorithmic}
\underline{Complejidad:} $\Ogr(|n_m|)$ \\
\underline{Justificacion:} $\Ogr(|h_m|)$ = $\Ogr(|n_m|)$
\end{algorithm}

\begin{algorithm}[H]
\caption{Mover Estudiante}
\begin{algorithmic}[1]
\Procedure{iMoverEstudiante}{\texttt{in/out} c : \texttt{estr}, \texttt{in} n : \texttt{nombre}, \texttt{in} d : \texttt{direccion}}
	\State posicion $p \gets$ PosEstudianteYHippie($c$,$n$) \Comment $\Ogr(|n_m|)$
	\State $c$.pos.mE[X($p$)][Y($p$)].def $\gets$ false \Comment $\Ogr(1)$
	\State posicion $proxp \gets$ ProxPosicion($c$.cp,$p$,$d$) \Comment $\Ogr(1)$
	\If{EsIngreso($c$.cp,$proxp$)} \Comment $\Ogr(1)$
		\State EliminarSiguiente($c$.pos.mE[X($p$)][Y($p$)].dato) \Comment $\Ogr(|e_m|)$
	\Else
		\State $c$.pos.mE[X($proxp$)][Y($proxp$)].def $\gets$ true \Comment $\Ogr(1)$
		\State $c$.pos.mE[X($proxp$)][Y($proxp$)].dato $\gets$ $c$.pos.mE[X($p$)][Y($p$)].dato \Comment $\Ogr(1)$
	
		\State enhippizar($c$, $p$) \Comment $\Ogr(|h_m|)$
		\State estudiantizar($c$, $p$) \Comment $\Ogr(|e_m|)$
		\State premiar($c$,$p$) \Comment $\Ogr(1)$
		\State capturar($c$,$p$) \Comment $\Ogr(|h_m|)$
		\State sancionar($c$,$p$) \Comment $\Ogr(1)$
	\EndIf
\EndProcedure
\end{algorithmic}
\underline{Complejidad:} $\Ogr(|n_m|)$\\
\underline{Justificacion:} $\Ogr(|n_m|) + \Ogr(|h_m|) + \Ogr(|e_m|) + \Ogr(|h_m|)$ = $\Ogr(|n_m|) + \Ogr(2|h_m|) + \Ogr(|e_m|)$ = $\Ogr(|n_m|) + \Ogr(|e_m| + |h_m|)$ = $\Ogr(|n_m|) + \Ogr(|n_m|)$ = $\Ogr(|n_m|)$
\end{algorithm}


\begin{algorithm}[H]
\caption{Mover Hippie}
\begin{algorithmic}[1]
\Procedure{iMoverHippie}{\texttt{in/out} c : \texttt{estr}, \texttt{in} n : \texttt{nombre}}
	\State posicion $p \gets$ PosEstudianteYHippie($c$,$n$) \Comment $\Ogr(|n_m|)$
	\State $c$.pos.mH[X($p$)][Y($p$)].def $\gets$ false \Comment $\Ogr(1)$
	\State posicion $proxp \gets$ nuevaPosHippieOAgente($c$,$p$,CrearIt($c$.estudiantes)) \Comment $\Ogr(N_e)$
	\State $c$.pos.mH[X($proxp$)][Y($proxp$)].def $\gets$ true \Comment $\Ogr(1)$
	\State $c$.pos.mH[X($proxp$)][Y($proxp$)].dato $\gets$ $c$.pos.mH[X($p$)][Y($p$)].dato \Comment $\Ogr(1)$
	
	\State enhippizar($c$, $p$) \Comment $\Ogr(|h_m|)$
	\State premiar($c$,$p$) \Comment $\Ogr(1)$
	\State capturar($c$,$p$) \Comment $\Ogr(|h_m|)$
	\State sancionar($c$,$p$) \Comment $\Ogr(1)$
\EndProcedure
\end{algorithmic}
\underline{Complejidad:} $\Ogr(|n_m|) + \Ogr(N_e)$\\
\underline{Justificacion:} $\Ogr(|n_m|) + \Ogr(N_e) + \Ogr(|h_m|) + \Ogr(|h_m|)$ = $\Ogr(n_m) + \Ogr(N_e) +2\Ogr(|h_m|)$ = $\Ogr(|n_m|) + \Ogr(N_e) + 2\Ogr(|n_m|)$ = $3\Ogr(|n_m|) + \Ogr(N_e)$ = $\Ogr(|n_m|) + \Ogr(N_e)$
\end{algorithm}


\begin{algorithm}[H]
\caption{Mover Agente}
\begin{algorithmic}[1]
\Procedure{iMoverAgente}{\texttt{in/out} c : \texttt{estr}, \texttt{in} pl : \texttt{placa}}
	\State posicion $p \gets$ Obtener($c$.agentes,$pl$) \Comment $\Ogr(log N_a)$
	\State $c$.pos.mAg[X($p$)][Y($p$)].def $\gets$ false \Comment $\Ogr(1)$
	\State posicion $proxp \gets$ nuevaPosHippieOAgente($c$,$p$,CrearIt($c$.hippies)) \Comment $\Ogr(N_h)$
	\State $c$.pos.mAg[X($proxp$)][Y($proxp$)].def $\gets$ true \Comment $\Ogr(1)$
	\State $c$.pos.mAg[X($proxp$)][Y($proxp$)].dato $\gets$ $c$.pos.mAg[X($p$)][Y($p$)].dato \Comment $\Ogr(1)$
	
	\State premiar($c$,$p$) \Comment $\Ogr(1)$
	\State capturar($c$,$p$) \Comment $\Ogr(|h_m|)$
	\State sancionar($c$,$p$) \Comment $\Ogr(1)$
\EndProcedure
\end{algorithmic}
\underline{Complejidad:} $\Ogr(|n_m|) + \Ogr(log N_a) + \Ogr(N_h)$
\end{algorithm}



\begin{algorithm}[H]
\caption{Nueva Posicion Hippie o Agente}
\begin{algorithmic}[1]
\Procedure{nuevaPosHippieOAgente}{\texttt{in} c : \texttt{estr}, \texttt{in} p : \texttt{posicion}, \texttt{in/out} itObj : \texttt{itDiccN(string,posicion)}}  $\to res$ : posicion
	\State nat $min \gets$ 0 \Comment $\Ogr(1)$
	\State posicion $objetivo$ \Comment $\Ogr(1)$
	\While{HaySiguiente($itObj$)} \Comment $n*\Ogr(1)$
		\If{$min$ == 0 || $min \geq$ DistanciaPosiciones($p$,SiguienteSignificado($itObj$))} \Comment $\Ogr(1)$
			\State $min \gets$ DistanciaPosiciones($p$,SiguienteSignificado($itObj$)) \Comment $\Ogr(1)$
			\State $objetivo \gets$ SiguienteSignificado($itObj$) \Comment $\Ogr(1)$
		\EndIf
		\State Avanzar($itObj$) \Comment $\Ogr(1)$
	\EndWhile
	\If{$min$ == 0} \Comment $\Ogr(1)$
		\State $objetivo \gets$ Siguiente(CrearIt(IngresosMasCercanos($c$.cp,$p$)))	 \Comment $\Ogr(1)$	
	\EndIf
	
	\State conj(posicion) $vecinosLibres \gets$ libres($c$,Vecinos($c$.cp,$p$)) \Comment $\Ogr(k)$
	\State itconj(posicion) $itVL \gets$ CrearIt($vecinosLibres$) \Comment $\Ogr(1)$
	\State $min \gets$ 0 \Comment $\Ogr(1)$
	\While{HaySiguiente($itVL$)} \Comment $m*\Ogr(1)$
		\If{$min$ == 0 || $min \geq$ DistanciaPosiciones($objetivo$,Siguiente($itVL$))}
			\State $min \gets$ DistanciaPosiciones($objetivo$,Siguiente($itVL$)) \Comment $\Ogr(1)$
			\State $res \gets$ Siguiente($itVL$) \Comment $\Ogr(1)$
		\EndIf
		\State Avanzar($itVL$) \Comment $\Ogr(1)$
	\EndWhile
\EndProcedure
\end{algorithmic}
\underline{Complejidad:} $\Ogr(n)$ siendo $n$ la cantidad de iteraciones de $itObj$ \\
\underline{Justificacion:} $n*Ogr(1)+\Ogr(k)+\Ogr(m)$ = $\Ogr(n+k+m)$\\
Donde
\begin{itemize}
	\item $n$ es la cantidad de iteraciones de $itObj$
	\item $m$ es la cantidad de posiciones libres de vecinos de una posicion, acotado por 4
	\item $k$ es la cantidad de vecinos de una posicion, acotado por 4
\end{itemize}
$\Ogr(n + 4 + 4)$ = $\Ogr(n + 8)$ = $\Ogr(n)$
\end{algorithm}


\begin{algorithm}[H]
\caption{Posiciones Libres}
\begin{algorithmic}[1]
\Procedure{libres}{\texttt{in} c : \texttt{estr}, \texttt{in} cp : \texttt{conj(posicion)}}  $\to res$ : conj(posicion)
	\State itconj(posicion) $itc \gets$ CrearIt($cp$) \Comment $\Ogr(1)$
	\State $res \gets$ Vacio() \Comment $\Ogr(1)$
	\While{HaySiguiente($itc$)} \Comment $n*\Ogr(1)$
		\If{!ocupada($c$,Siguiente($p$))} \Comment $\Ogr(1)$
			\State AgregarRapido($res$,Siguiente($p$)) \Comment $\Ogr(1)$
		\EndIf
		\State Avanzar($itc$) \Comment $\Ogr(1)$
	\EndWhile
\EndProcedure
\end{algorithmic}
\underline{Complejidad:} $\Ogr(n)$ siendo n $\#$Claves($cp$)
\end{algorithm}


\begin{algorithm}[H]
\caption{De estudiante a hippie}
\begin{algorithmic}[1]
	\Procedure{enhippizar}{\texttt{in/out} c : \texttt{estr}, \texttt{in} p : \texttt{posicion}}
	\State itConj(posicion) $itc$ $\gets$ CrearIt(AgregarRapido($p$,Vecinos($c$.cp,$p$))) \Comment $\Ogr(1)$
	\While {HaySiguiente($itc$)} \Comment $n+\Ogr(1)$
	\State posicion $p$ $\gets$ Siguiente($itc$) \Comment $\Ogr(1)$
		\If{esEstudiante($c$,$p$) $\&\&$ enhippizado($c$,$p$)} \Comment $\Ogr(1)$ 
			\State itDiccN(string,posicion) $itEaH \gets$ $c$.pos.mE[X($p$)][Y($p$)].dato \Comment $\Ogr(1)$
			\State EliminarSiguiente($itEaH$) \Comment $\Ogr(h_m)$
			\State $c$.pos.mE[X($p$)][Y($p$)].def $\gets$ false \Comment $\Ogr(1)$
			\State $c$.pos.mH[X($p$)][Y($p$)].def $\gets$ true \Comment $\Ogr(1)$
			\State $c$.pos.mH[X($p$)][Y($p$)].dato $\gets$ Definir($c$.hippies,$n$,$p$) \Comment $\Ogr(|h_m|)$
		\EndIf
		\State Avanzar($itc$) \Comment $\Ogr(1)$
	\EndWhile
	\EndProcedure
\end{algorithmic}
\underline{Complejidad:} $\Ogr(|h_m|)$\\
\underline{Justificacion:} $\Ogr(1) + (n*\Ogr(1))* 2\Ogr(|h_m|)$ = $\Ogr(n)*2\Ogr(|h_m|)$ con $n$ siendo la cantidad de vecinos + 1, acotado por 5. \\
$\Ogr(5)*2\Ogr(|h_m|)$ = $\Ogr(|h_m|)$
\end{algorithm}

\begin{algorithm}[H]
\caption{De hippie a estudiante}
\begin{algorithmic}[1]
	\Procedure{estudiantizar}{\texttt{in/out} c : \texttt{estr}, \texttt{in} p : \texttt{posicion}}
	\State itConj(posicion) $itc$ $\gets$ CrearIt(AgregarRapido($p$,Vecinos($c$.cp,$p$))) \Comment $\Ogr(1)$
	\While {HaySiguiente($itc$)} \Comment $n + \Ogr(1)$
	\State posicion $p$ $\gets$ Siguiente($itc$) \Comment $\Ogr(1)$
	\If{esHippie($c$,$p$) $\&\&$ estudiantizado($c$,$p$)} \Comment $\Ogr(1)$
		\State itDiccN(string,posicion) $itHaE \gets$ $c$.pos.mH[X($p$)][Y($p$)] \Comment $\Ogr(1)$
		\State EliminarSiguiente($itHaE$) \Comment $\Ogr(|e_m|)$
		\State $c$.pos.mH[X($p$)][Y($p$)].def $\gets$ false \Comment $\Ogr(1)$
		\State $c$.pos.mE[X($p$)][Y($p$)].def $\gets$ true \Comment $\Ogr(1)$
		\State $c$.pos.mE[X($p$)][Y($p$)].dato $\gets$ Definir($c$.estudiantes,$n$,$p$) \Comment $\Ogr(|e_m|)$
	\EndIf
	\EndWhile
	\EndProcedure
\end{algorithmic}
\underline{Complejidad:} $\Ogr(|e_m|)$\\
\underline{Justificacion:} $\Ogr(1) + (n*\Ogr(1))* 2\Ogr(|e_m|)$ = $\Ogr(n)*2\Ogr(|e_m|)$ con $n$ siendo la cantidad de vecinos + 1, acotado por 5. \\
$\Ogr(5)*2\Ogr(|e_m|)$ = $\Ogr(|e_m|)$
\end{algorithm}



\begin{algorithm}[H]
\caption{Premiar Agentes}
\begin{algorithmic}[1]
	\Procedure{premiar}{\texttt{in/out} c : \texttt{estr}, \texttt{in} p : \texttt{posicion}}
	\If{rodeado($c$,$p$) $\&\&$ esHippie($c$,$p$) $\&\&$ $\#$Claves(agentes($c$,Vecinos($c$.cp,$p$))} \Comment $\Ogr(k)$
			\State itConj(posicion) $itca \gets$ agentes($c$,Vecinos($c$.cp,$p$)) \Comment $\Ogr(k)$
			\While{HaySiguiente($itca$)} \Comment $n*\Ogr(1)$
				\State Siguiente($c$.pos.mAg[X($p$)][Y($p$)]).premios ++ \Comment $\Ogr(1)$
				\State actualizarMasVigilante($c$,pos.mAg[X($p$)][Y($p$)]) \Comment $\Ogr(1)$
				\State Avanzar($itca$) \Comment $\Ogr(1)$
			\EndWhile
	\Else
	\State itConj(posicion) $itc$ $\gets$ CrearIt(Vecinos($c$.cp, $p$)) \Comment $\Ogr(1)$
	\While {HaySiguiente($itc$)} \Comment $n*\Ogr(1)$
	\State posicion $p$ $\gets$ Siguiente($itc$) \Comment $\Ogr(1)$
		\If{rodeado($p$) $\&\&$ esHippie($p$)} \Comment $\Ogr(1)$
			\State itConj(posicion) $itca \gets$ agentes($c$,Vecinos($c$.cp,$p$)) \Comment $\Ogr(k)$
			\While{HaySiguiente($itca$)} \Comment $m*\Ogr(1)$
				\State $pa \gets$ Siguiente($itca$) \Comment $\Ogr(1)$
				\State Siguiente(pos.mAg[X($pa$)][Y($pa$)]).premios ++ \Comment $\Ogr(1)$
				\State actualizarMasVigilante($c$,pos.mAg[X($pa$)][Y($pa$)]) \Comment $\Ogr(1)$
				\State Avanzar($itca$) \Comment $\Ogr(1)$
			\EndWhile
		\EndIf
	\EndWhile
	\EndIf
	\EndProcedure
\end{algorithmic}
\underline{Complejidad:} $\Ogr(1)$\\
\underline{Justificacion:} $\Ogr(1) + n*\Ogr(1)*(\Ogr(k) + m\Ogr(1))$ = $\Ogr(n*(k + m))$\\
Donde
\begin{itemize}
	\item $n$ es la cantidad de vecinos de una posicion, acotado por 4
	\item $m$ es la cantidad de agentes vecinos de una posicion, acotado por 4
	\item $k$ es la cantidad de vecinos de una posicion, acotado por 4
\end{itemize}
$\Ogr(4*(4+4))$ = $\Ogr(32)$ = $\Ogr(1)$
\end{algorithm}


\begin{algorithm}[H]
\caption{Capturar Hippies}
\begin{algorithmic}[1]
	\Procedure{capturar}{\texttt{in/out} c : \texttt{estr}, \texttt{in} p : \texttt{posicion}}
	\If{rodeado($c$,$p$) $\&\&$ esHippie($c$,$p$) $\&\&$ $\#$Claves(agentes($c$,Vecinos($c$.cp,$p$)) $\geq$ 1} \Comment $\Ogr(k)$
			\State itDiccN(string,posicion) $itH \gets$ $res$.pos.mH[X($p$)][Y($p$)] \Comment $\Ogr(1)$
			\State EliminarSiguiente($itH$) \Comment $\Ogr(|h_m|)$
			\State $res$.pos.mE[X($p$)][Y($p$)].def $\gets$ false \Comment $\Ogr(1)$
	\Else
	\State itConj(posicion) $itc$ $\gets$ CrearIt(Vecinos($c$.cp,$p$)) \Comment $\Ogr(1)$
	\While {HaySiguiente($itc$)} \Comment $n*\Ogr(1)$
		\State posicion $p$ $\gets$ Siguiente($itc$) \Comment $\Ogr(1)$
		\If{rodeado($c$,$p$) $\&\&$ esHippie($c$,$p$) $\&\&$ $\#$Claves(agentes($c$,Vecinos($c$.cp,$p$))) $\geq$ 1} \Comment $\Ogr(k)$
			\State itDiccN(string,posicion) $itH \gets$ $c$.pos.mH[X($p$)][Y($p$)] \Comment $\Ogr(1)$
			\State EliminarSiguiente($itH$) \Comment $\Ogr(|h_m|)$
			\State $c$.pos.mE[X($p$)][Y($p$)].def $\gets$ false \Comment $\Ogr(1)$
		\EndIf
		\State Avanzar($itc$) \Comment $\Ogr(1)$
	\EndWhile
	
	\EndIf
	\EndProcedure
\end{algorithmic}
\underline{Complejidad:} $\Ogr(|h_m|)$\\
\underline{Justificacion:} $\Ogr(k) + n*\Ogr(1)*(\Ogr(|h_m|) + \Ogr(k))$ = $\Ogr(k) + \Ogr(n)*\Ogr(|h_m|+ k)$ = $\Ogr(k) + \Ogr(n*|h_m| + n*k)$ = $\Ogr(n*|h_m| + n*k)$ ($n*k > k$)\\
Donde
\begin{itemize}
	\item $n$ es la cantidad de vecinos de una posicion, acotado por 4
	\item $|h_m|$ es el hippie con nombre mas largo
	\item $k$ es la cantidad de vecinos de una posicion, acotado por 4
\end{itemize}
$\Ogr(4*|h_m| + 4*4)$ = $\Ogr(4*|h_m| + 16)$ = $\Ogr(4*|h_m|)$ = $\Ogr(|h_m|)$
\end{algorithm}

\begin{algorithm}[H]
\caption{Sancionar Agentes}
\begin{algorithmic}[1]
	\Procedure{sancionar}{\texttt{in/out} c : \texttt{estr}, \texttt{in} p : \texttt{posicion}}
	
	\State itConj(posicion) $itc$ $\gets$ CrearIt(AgregarRapido($p$,Vecinos($c$.cp,$p$)) \Comment $\Ogr(1)$
	\While {HaySiguiente($itc$)} \Comment $n*\Ogr(1)$
		\State posicion $p$ $\gets$ Siguiente($itc$) \Comment $\Ogr(1)$
		\State conj(posicion) $cag \gets$ agentes($c$,Vecinos($c$.cp,$p$) \Comment $\Ogr(k)$
		\If{rodeado($c$,$p$) $\&\&$ esEstudiante($c$,$p$) $\&\&$ $\#$Claves($cag$) $\geq$ 1} \Comment $\Ogr(1)$
		\State itConj(posicion) $itca \gets cag$ \Comment $\Ogr(1)$
			\While{HaySiguiente($itca$)} \Comment $m*\Ogr(1)$
				\State posicion $pa \gets$ Siguiente($itca$) \Comment $\Ogr(1)$
				\State Siguiente($c$.pos.mAg[X($pa$)][Y($pa$)]).dato).sanciones ++ \Comment $\Ogr(1)$
				\State datosAg $datos \gets$ Siguiente($c$.pos.mAg[X($pa$)][Y($pa$)]).dato \Comment $\Ogr(1)$
				\State EliminarSiguiente($datos$.mismasSanc) \Comment $\Ogr(1)$
				\If{$\#$(SiguienteSignificado($datos$.kSanc)) == 0} \Comment $\Ogr(1)$
					\State EliminarSiguiente($datos$.kSanc) \Comment $\Ogr(1)$
				\EndIf
				\If{$datos$.sanciones $<$ SiguienteClave($datos$.kSanc)} \Comment $\Ogr(1)$
					\State InsertarAdelante($datos$.kSanc,$datos$.sanciones,Vacio()) \Comment $\Ogr(1)$
				\Else
					\State Avanzar($datos$.kSanc) \Comment $\Ogr(1)$
				\EndIf
				\State $datos$.mismasSanc $\gets$ AgregarRapido(SiguienteSignificado($datos$.kSanc),$datos$.pl) \Comment $\Ogr(1)$
			\EndWhile
			
		\EndIf
		\State Avanzar($itc$) \Comment $\Ogr(1)$
	\EndWhile
	\EndProcedure
\end{algorithmic}
\underline{Complejidad:} $\Ogr(1)$\\
\underline{Justificacion:} $n*(\Ogr(k) + m*\Ogr(1))$ = $\Ogr(n)*\Ogr(k+m)$ = $\Ogr(n*k + n*m)$\\
Donde
\begin{itemize}
	\item $n$ es la cantidad de vecinos de una posicion + 1, acotado por 5
	\item $m$ es la cantidad de agentes vecinos de una posicion, acotado por 4
	\item $k$ es la cantidad de vecinos de una posicion, acotado por 4
\end{itemize}
$\Ogr(5*4 + 5*4)$ = $\Ogr(40)$ = $\Ogr(1)$

\end{algorithm}


\begin{algorithm}[H]
\caption{¿Se transforma en hippie?}
\begin{algorithmic}[1]
	\Procedure{enhippizado}{\texttt{in} c : \texttt{estr}, \texttt{in} p : \texttt{posicion}} $\to res$ : bool
	\State $res \gets$ false \Comment $\Ogr(1)$
	\State itConj(posicion) $itc$ $\gets$ CrearIt(Vecinos($c$.cp, $p$)) \Comment $\Ogr(1)$
	\State nat $i \gets 0$ \Comment $\Ogr(1)$
	\While{HaySiguiente($itc$)} \Comment $n * \Ogr(1)$
		\If{esHippy($c$,Siguiente($itc$))} \Comment $\Ogr(1)$
			\State $i \gets i+1$ \Comment $\Ogr(1)$
		\EndIf
		\State Avanzar($itc$) \Comment $\Ogr(1)$
	\EndWhile
	$res \gets (i \geq 2)$ \Comment $\Ogr(1)$
	\EndProcedure
\end{algorithmic}
\underline{Complejidad:} $\Ogr(1)$\\
\underline{Justificacion:} $\Ogr(1) + n * \Ogr(1)$ =  $\Ogr(n)$ siendo $n$ la cantidad de vecinos, que como maximo es 4 \\
$\Ogr(4)$ = $\Ogr(1)$
\end{algorithm}


\begin{algorithm}[H]
\caption{¿Es hippie?}
\begin{algorithmic}[1]
	\Procedure{esEstudiante}{\texttt{in} c : \texttt{estr}, \texttt{in} p : \texttt{posicion}} $\to res$ : bool
	\State $res \gets$ $c$.pos.mH[X($p$)][Y($p$)].def \Comment $\Ogr(1)$
	\EndProcedure
\end{algorithmic}
\underline{Complejidad:} $\Ogr(1)$
\end{algorithm}


\begin{algorithm}[H]
\caption{¿Se transforma en estudiante?}
\begin{algorithmic}[1]
	\Procedure{enhippizado}{\texttt{in} c : \texttt{estr}, \texttt{in} p : \texttt{posicion}} $\to res$ : bool
	\State $res \gets$ false \Comment $\Ogr(1)$
	\State itConj(posicion) $itc$ $\gets$ CrearIt(Vecinos($c$.cp, $p$)) \Comment $\Ogr(1)$
	\State nat $i \gets 0$ \Comment $\Ogr(1)$
	\While{HaySiguiente($itc$)}  \Comment $n*\Ogr(1)$
		\If{esEstudiante(Siguiente($itc$))}  \Comment $\Ogr(1)$
			\State $i \gets i+1$  \Comment $\Ogr(1)$
		\EndIf
		\State Avanzar($itc$)  \Comment $\Ogr(1)$
	\EndWhile
	$res \gets (i = 4)$  \Comment $\Ogr(1)$
	\EndProcedure
\end{algorithmic}
\underline{Complejidad:} $\Ogr(1)$\\
\underline{Justificacion:} $\Ogr(1) + n * \Ogr(1)$ =  $\Ogr(n)$ siendo $n$ la cantidad de vecinos, que como maximo es 4 \\
$\Ogr(4)$ = $\Ogr(1)$
\end{algorithm}


\begin{algorithm}[H]
\caption{¿Es estudiante?}
\begin{algorithmic}[1]
	\Procedure{esEstudiante}{\texttt{in} c : \texttt{estr}, \texttt{in} p : \texttt{posicion}} $\to res$ : bool
	\State $res \gets$ $c$.pos.mE[X($p$)][Y($p$)].def \Comment $\Ogr(1)$
	\EndProcedure
\end{algorithmic}
\underline{Complejidad:} $\Ogr(1)$
\end{algorithm}


\begin{algorithm}[H]
\caption{¿Esta Rodeado?}
\begin{algorithmic}[1]
	\Procedure{rodeado}{\texttt{in} c : \texttt{estr}, \texttt{in} p : \texttt{posicion}} $\to res$ : bool
	\State itConj(posicion) $itc$ $\gets$ CrearIt(Vecinos($c$.cp, $p$)) \Comment $\Ogr(1)$
	\State nat $i \gets$ 0 \Comment $\Ogr(1)$
		\While{HaySiguiente($itc$)} \Comment $n*\Ogr(1)$
			\If{ocupado(Siguiente($p$))} \Comment $\Ogr(1)$
				\State $i$++ \Comment $\Ogr(1)$
			\EndIf
			Avanzar($itc$) \Comment $\Ogr(1)$
		\EndWhile
		$res \gets (i==4)$ \Comment $\Ogr(1)$
	\EndProcedure
\end{algorithmic}
\underline{Complejidad:} $\Ogr(1)$\\
\underline{Justificacion:} $n*\Ogr(1)$ = $\Ogr(n)$ siendo n la cantidad de vecinos de una posicion, con cota 4\\
$\Ogr(4)$ = $\Ogr(1)$
\end{algorithm}



\begin{algorithm}[H]
\caption{¿Esta Ocupado?}
\begin{algorithmic}[1]
	\Procedure{rodeado}{\texttt{in} c : \texttt{estr}, \texttt{in} p : \texttt{posicion}} $\to res$ : bool
	\State $res \gets$ (esHippie($c$,$p$) || esAgente($c$,$p$) || esEstudiante($c$,$p$) || Ocupada($c$.cp,$p$)) \Comment $\Ogr(1)$
	\EndProcedure
\end{algorithmic}
\underline{Complejidad:} $\Ogr(1)$
\end{algorithm}

\begin{algorithm}[H]
\caption{Agentes en posiciones}
\begin{algorithmic}[1]
	\Procedure{agentes}{\texttt{in} c : \texttt{estr}, \texttt{in} cp : \texttt{conj(posicion)}} $\to res$ : conj(posicion)
	\State itConj(posicion) $itc$ $\gets$ CrearIt($cp$) \Comment $\Ogr(1)$
	$res \gets $Vacio();
	\While{HaySiguiente($itc$)} \Comment $n*\Ogr(1)$
			\If{esAgente(Siguiente($p$))} \Comment $\Ogr(1)$
				\State AgregarRapido($res$,$p$) \Comment $\Ogr(1)$
			\EndIf
			Avanzar($itc$) \Comment $\Ogr(1)$
		\EndWhile
	\EndProcedure
\end{algorithmic}
\underline{Complejidad:} $\Ogr(n)$ siendo $n$ = $\#$(Claves($cp$))
\end{algorithm}


\begin{algorithm}[H]
\caption{¿Es agente?}
\begin{algorithmic}[1]
	\Procedure{esAgente}{\texttt{in} c : \texttt{estr}, \texttt{in} p : \texttt{posicion}} $\to res$ : bool
	\State $res \gets$ $c$.pos.mAg[X($p$)][Y($p$)].def \Comment $\Ogr(1)$
	\EndProcedure
\end{algorithmic}
\underline{Complejidad:} $\Ogr(1)$
\end{algorithm}


\begin{algorithm}[H]
\caption{Actualizar mas Vigilante}
\begin{algorithmic}[1]
	\Procedure{actualizarVigilante}{\texttt{in/out} c : \texttt{estr}, \texttt{in} datos : \texttt{datosAg}}
	\If{$datos$.premios $> c$.masVigilante.premios} \Comment $\Ogr(1)$
		\State $c$.masVigilante.pl $\gets datos$.pl	\Comment $\Ogr(1)$		
		\State $c$.masVigilante.premios $\gets datos$.premios \Comment $\Ogr(1)$
	\EndIf
	\If{$datos$.premios == $c$.masVigilante.premios} \Comment $\Ogr(1)$
		\If{$datos$.pl $< c$.masVigilante.pl} \Comment $\Ogr(1)$
			\State $c$.masVigilante.pl $\gets datos$.pl	\Comment $\Ogr(1)$		
			\State $c$.masVigilante.premios $\gets datos$.premios \Comment $\Ogr(1)$
		\EndIf
	\EndIf
	\EndProcedure
\end{algorithmic}
\underline{Complejidad:} $\Ogr(1)$
\end{algorithm}

\end{Algoritmos}


\subsection{Campus}
\begin{Interfaz}
  
	\textbf{se explica con}: \tadNombre{Campus}
	
	\textbf{usa}: nat, bool, Matriz(bool), posicion
	
	\textbf{géneros}: \TipoVariable{campus}
	
	\tituloModulo{Operaciones de Campus}

	\InterfazFuncion{CrearCampus}{\In{ancho}{nat}, \In{alto}{nat}}{campus}
	{res $\igobs$ crearCampus(ancho,alto)}
	[$\Ogr$(alto x ancho)]
	[Crea un campus sin obstaculos]
	
	\InterfazFuncion{AgregarObstaculo}{\Inout{c}{campus}, \In{p}{posicion}}{}
	[posValida?(p,c) $\land$ $\neg$ocupada(p, c) $\land$ c = $c_{0}$]
	{c $\igobs$ agregarObstáculo(p, $c_{0}$)}
	[$\Ogr$(1)]
	[Agrega un obstáculo en la posición dada]
	
	\InterfazFuncion{Filas}{\In{c}{campus}}{nat}
	{res $\igobs$ filas(c)}
	[$\Ogr$(1)]
	[Indica la cantidad de filas del campus dado]
	
	\InterfazFuncion{Columnas}{\In{c}{campus}}{nat}
	{res $\igobs$ columnas(c)}
	[$\Ogr$(1)]
	[Indica la cantidad de columnas del campus dado]	
	
	\InterfazFuncion{Ocupada}{\In{c}{campus}, \In{p}{posicion}}{bool}
	[posValida?(p,c)]
	{res $\igobs$ ocupada(p,c)}
	[$\Ogr$(1)]
	[Indica si la posición dada tiene un obstáculo]
	
	\InterfazFuncion{PosValida}{\In{c}{campus}, \In{p}{posicion}}{bool}
	{res $\igobs$ posValida?(p,c)}
	[$\Ogr$(1)]
	
	\InterfazFuncion{IngresoSuperior}{\In{c}{campus}, \In{p}{posicion}}{bool}
	{res $\igobs$ ingresoSuperior?(p,c)}
	[$\Ogr$(1)]
	
	\InterfazFuncion{IngresoInferior}{\In{c}{campus}, \In{p}{posicion}}{bool}
	{res $\igobs$ ingresoInferior?(p,c)}
	[$\Ogr$(1)]
	
	\InterfazFuncion{EsIngreso}{\In{c}{campus}, \In{p}{posicion}}{bool}
	{res $\igobs$ esIngreso?(p,c)}
	[$\Ogr$(1)]
	
	\InterfazFuncion{Vecinos}{\In{c}{campus}, \In{p}{posicion}}{con(posicion)}
	[posValida?(p,c)]
	{res $\igobs$ vecinos?(p,c)}
	[$\Ogr$(1)]
	[Dada una posición, devuelve las 4 posiciones vecinas]
		
	\InterfazFuncion{VecinosComunes}{\In{c}{campus}, \In{p_1}{posicion}, \In{p_2}{posicion}}{conj(posicion)}
	[posValida?($p_1$,c) $\land$ posValida?($p_1$,c)]
	{res $\igobs$ vecinosComunes($p_1$,$p_2$,c)}
	[$\Ogr$(1)]
	
	\InterfazFuncion{VecinosValidos}{\In{c}{campus}, \Inout{cp}{conj(posicion)}}{}
	[c = $c_0$]
	{c $\igobs$ vecinosValidos}
	[$\Ogr$( |cp| )]
	[Dado un conjunto de posiciones devuelve el conjunto resultado de quitar las posiciones no validas]
	
	\InterfazFuncion{ProxPosicion}{\In{c}{campus}, \In{p}{posicion}, \In{d}{direccion}}{posicion}
	[posValida?(p,c)]
	{res $\igobs$ proxPosición(p,d,c)}
	[$\Ogr$(1)]
	[Devuelve el resultado de mover la posición indicada en la dirección dada]
		
	\InterfazFuncion{IngresosMasCercanos}{\In{c}{campus}, \In{p}{posicion}}{conj(posicion)}
	[posValida?(p,c)]
	{res $\igobs$ ingresosMásCercanos(p,c)}
	[$\Ogr$(1)]
	[Devuelve el conjunto de ingresos más cercanos a la posición dada]
	
	\InterfazFuncion{Copiar}{\In{c}{campus}}{campus}
	{res $\igobs$ c}
	[$\Ogr$(Filas(c) * Columnas(c))]
	[Crea una copia del campus]
				
\end{Interfaz}

\begin{Representacion}

	\tituloModulo{Representación del campus}
	
	\begin{Estructura}{campus}
	donde estr es Matriz(bool)
	\end{Estructura}
	
	\Rep{true}
~	
	\AbsFc{campus}{c $:$ campus $/$\\
	filas(c) $=$ Alto(e) $\land$ columnas(c) $=$ Ancho(e)\\
	$\land$ ($\forall$ p : posicion)( posValida?(p,c) $\impluego$ ocupada?(p,c) = e[p.X $-$ 1][p.Y $-$ 1] )
	}
	
	
\begin{Algoritmos}

	\begin{algorithm}[H]
		\caption{iCrearCampus}
		
		\begin{algorithmic}[1]
			\Procedure{iCrearCampus}{\texttt{in} ancho : \texttt{nat}, \texttt{in} alto : \texttt{nat}} $\to res$ : $estr$
				\State $res \gets$ Crear(ancho,alto,false) \Comment $\Ogr$(ancho x alto)
			
			\EndProcedure
		\end{algorithmic}
		\underline{Complejidad:} $\Ogr$(ancho * alto)
				
	\end{algorithm}
	
	\begin{algorithm}[H]
		\caption{iAgregar}
		\begin{algorithmic}[1]
			\Procedure{iAgregar}{\texttt{in/out} e : \texttt{estr}, \texttt{in} p : \texttt{posicion}}
				\State e[p.X $-$ 1][p.Y $-$ 1] $\gets$ true \Comment $\Ogr$(1)
			
			\EndProcedure
		\end{algorithmic}
		\underline{Complejidad:} $\Ogr$(1)
	
	
	\end{algorithm}	
	
	\begin{algorithm}[H]
		\caption{iFila}
		\begin{algorithmic}[1]
			\Procedure{iFila}{\texttt{in} e : \texttt{estr}} $\to res$ : $nat$
				\State $res \gets$ Alto($e$)	\Comment $\Ogr$(1)
			\EndProcedure		
		\end{algorithmic}
		\underline{Complejidad:} $\Ogr$(1)
	\end{algorithm}
	
	
	\begin{algorithm}[H]
		\caption{iColumna}
		\begin{algorithmic}[1]
			\Procedure{iColumna}{\texttt{in} e : \texttt{estr}} $\to res$ : $nat$
				\State $res \gets$ Ancho($e$) \Comment $\Ogr$(1)
			\EndProcedure		
		\end{algorithmic}
		\underline{Complejidad:} $\Ogr$(1)
	\end{algorithm}
	
	\begin{algorithm}
		\caption{iOcupada}
		\begin{algorithmic}[1]
			\Procedure{iOcupada}{\texttt{in} e : \texttt{estr}, \texttt{in} p: \texttt{posicion}} $\to res$ : $bool$	
				\State $res \gets$ e[p.X $-$ 1][e.Y $-$ 1] \Comment $\Ogr$(1)
			\EndProcedure		
		\end{algorithmic}
		\underline{Complejidad:} $\Ogr$(1)
	\end{algorithm}
	
	\begin{algorithm}
		\caption{iPosValida}
		\begin{algorithmic}
			\Procedure{iPosValida}{\texttt{in} e: \texttt{estr}, \texttt{in} p : \texttt{posicion}} $\to res$ : $bool$
				\State 0 $<$ p.X $\land$ p.X $\leq$ Ancho(e) $\land$ 0 $<$ p.Y $\land$ p.Y $\leq$ Alto(e) \Comment $\Ogr$(1)
			\EndProcedure		
		\end{algorithmic}
		\underline{Complejidad:} $\Ogr$(1)
	\end{algorithm}

	\begin{algorithm}[H]
		\caption{iIngresoSuperior}
		\begin{algorithmic}[1]
			\Procedure{iIngresoSuperior}{\texttt{in} e : \texttt{estr}, \texttt{in} p : \texttt{posicion}} $to res$ : $bool$
				\State $res \gets$ (p.Y $=$ 1) \Comment $\Ogr$(1)		
			\EndProcedure		
		\end{algorithmic}
		\underline{Complejidad:} $\Ogr$(1)
	\end{algorithm}
	
	\begin{algorithm}[H]
		\caption{iIngresoInferior}
		\begin{algorithmic}[1]
			\Procedure{iIngresoInferior}{\texttt{in} e : \texttt{estr}, \texttt{in} p : \texttt{posicion}} $to res$ : $bool$
				\State $res \gets$ (p.Y $=$ Alto()) \Comment $\Ogr$(1)		
			\EndProcedure		
		\end{algorithmic}
		\underline{Complejidad:} $\Ogr$(1)
	\end{algorithm}
	
	\begin{algorithm}[H]
		\caption{iEsIngreso}
		\begin{algorithmic}[1]
			\Procedure{iEsIngreso}{\texttt{in} e : \texttt{estr}, \texttt{in} p : \texttt{posicion}} $\to res$ : $bool$
				\State $res \gets$ (iEsIngresoSuperior(e,p) $\land$ iEsIngresoInferior(e,p)) \Comment $\Ogr$(1)
			\EndProcedure		
		\end{algorithmic}
		\underline{Complejidad:} $\Ogr$(1)
	\end{algorithm}
	
	\begin{algorithm}[H]
		\caption{iVecinos}
		\begin{algorithmic}[1]
			\Procedure{iVecinos}{\texttt{in} e : \texttt{estr}, \texttt{in} p : \texttt{posicion}} $\to res$ : $conj(posicion)$
				\State $res \gets$ Vacio() //vacio de conjunto \Comment $\Ogr$(1)
				\State AgregarRapido(res, <p.X $-$ 1, p.Y>) \Comment $\Ogr$(1)
				\State AgregarRapido(res, <p.X $+$ 1, p.Y>) \Comment $\Ogr$(1)
				\State AgregarRapido(res, <p.X, p.Y $-$ 1>) \Comment $\Ogr$(1)
				\State AgregarRapido(res, <p.X, p.Y $+$ 1>) \Comment $\Ogr$(1)
				\State $res \gets$ iVecinosValidos(e, res) \Comment $\Ogr$($\#$res) $=$ $\Ogr$(4) $=$ $\Ogr$(1)				
			\EndProcedure		
		\end{algorithmic}
		\underline{Complejidad:} $\Ogr$(1)
	\end{algorithm}
		
	\begin{algorithm}[H]
		\caption{iVecinosComunes}
		\begin{algorithmic}[1]
			\Procedure{iVecinosComunes}{\texttt{in} e : \texttt{estr}, \texttt{in} $p_1$ : \texttt{posicion}, \texttt{in} $p_2$ : \texttt{posicion}} $\to res$ : $conj(posicion)$
				\State $vec_1 \gets$ iVecinos(e, $p_1$) \Comment $\Ogr$(1)
				\State $vec_2 \gets$ iVecinos(e, $p_2$) \Comment $\Ogr$(1)
				\State $res \gets$ Vacio() //vacio de conjunto \Comment $\Ogr$(1)
				\State $it_1 \gets$ CrearIt($vec_1$) \Comment $\Ogr$(1)
				\While{HaySiguiente($it_1$)} \Comment Se repite 4 veces y lo de adentro es $\Ogr$(1) $\Rightarrow$ $\Ogr$(1)	
					\State $it_2 \gets$ CrearIt($vec_2$) \Comment $\Ogr$(1)
					\While{HaySiguiente($it_2$)} \Comment Se repite 4 veces y lo de adentro es $\Ogr$(1) $\Rightarrow$ $\Ogr$(1)
						\If{Siguiente($it_1$) $=$ Siguiente($it_2$)} \Comment $\Ogr$(1)
							\State AgregarRapido(res, Siguiente($it_1$))
						\EndIf
						\State Avanzar($it_2$) \Comment $\Ogr$(1)
					\EndWhile
					\State Avanzar($it_1$) \Comment $\Ogr$(1)
				\EndWhile		
			\EndProcedure		
		\end{algorithmic}
		\underline{Complejidad:} $\Ogr$(1)
	\end{algorithm}
			
	\begin{algorithm}[H]
		\caption{iVecinosValidos}
		\begin{algorithmic}[1]
			\Procedure{iVecinosValidos}{\texttt{in} e : \texttt{estr},\texttt{in/out} cp : $conj(posicion)$}
				\State $it \gets$ CrearIt(cp) \Comment $\Ogr$(1)
				\While{HaySiguiente(it)} \Comment Se repite $\#$cp veces y lo de adentro es $\Ogr$(1) $\Rightarrow$ $\Ogr$($\#$cp)
					\If{$\neg$ iPosValida(e, Siguiente(it))} \Comment $\Ogr$(1)
						\State EliminarSiguiente(it) \Comment $\Ogr$(1)
					\Else
						\State Avanzar(it) \Comment $\Ogr$(1)
					\EndIf
				\EndWhile
			\EndProcedure		
		\end{algorithmic}
		\underline{Complejidad:} $\Ogr$($\#$cp)
	\end{algorithm}
				
	\begin{algorithm}[H]
		\caption{iProxPosicion}
		\begin{algorithmic}[1]
			\Procedure{iProxPosicion}{\texttt{in} e : \texttt{estr}, \texttt{in} p : \texttt{posicion}, \texttt{in} dir : \texttt{direccion}} $\to res$ : $posicion$
				\State $res \gets$ <p.X $+$ $\beta$(dir $=$ der) $-$ $\beta$(dir $= $ izq), p.Y $+$ $\beta$(dir $=$ abajo) $-$ $\beta$(dir $=$ arriba)> \Comment $\Ogr$(1)
			\EndProcedure		
		\end{algorithmic}
		\underline{Complejidad:} $\Ogr$(1)
	\end{algorithm}
				
	\begin{algorithm}[H]
		\caption{iIngresosMasCercanos}
		\begin{algorithmic}[1]
			\Procedure{iIngresosMasCercanos}{\texttt{in} e : \texttt{estr}, \texttt{in} p : \texttt{posicion}} $\to res$ : $conj(posicion)$
				\State $res \gets$ Vacio() //vacio de conjunto \Comment $\Ogr$(1)
				\If{Dist(p, <p.X, 1>) $<$ Dist(p, <p.Y, iFilas(e)>)} \Comment $\Ogr$(1)
					\State AgregarRapido(res, <p.X, 1>) \Comment $\Ogr$(1)
				\Else
					\If{Dist(p, <p.X, 1>) $>$ Dist(p, <p.Y, iFilas(e)>)} \Comment $\Ogr$(1)
						\State AgregarRapido(res, <p.X, iFilas(e)>) \Comment $\Ogr$(1)
					\Else
						\State AgregarRapido(res, <p.X, 1>) \Comment $\Ogr$(1)
						\State AgregarRapido(res, <p.X, iFilas(e)>) \Comment $\Ogr$(1)
					\EndIf				
				\EndIf
			\EndProcedure		
		\end{algorithmic}
		\underline{Complejidad:} $\Ogr$(1)
	\end{algorithm}
	
	\begin{algorithm}[H]
		\caption{iCopiar}
		\begin{algorithmic}[1]
			\Procedure{iCopiar}{\texttt{in} e : \texttt{estr}} $\to res$ : $estr$
				\State $res \gets$ Copiar(e) \Comment $\Ogr$(Copiar(e)) $=$ $\Ogr$(Ancho(e) * Alto(e))
			\EndProcedure
		\end{algorithmic}
		\underline{Complejidad:} $\Ogr$(Ancho(e) * Alto(e))
	\end{algorithm}	
	
\end{Algoritmos}
	    
    
\end{Representacion}

\subsection{Diccionario de Nombres}
\begin{Interfaz}

	\textbf{parámetros formales}\hangindent=2\parindent\\
	\parbox{1.7cm}{\textbf{géneros}} $\sigma$\\
	\parbox[t]{1.7cm}{\textbf{función}}\parbox[t]{\textwidth-2\parindent-1.7cm}{%
		\InterfazFuncion{Copiar}{\In{s}{$\sigma$}}{$\sigma$}
	    	{$res \igobs a$}
		[$\Theta(copy(s))$]
		[función de copia de $\sigma$'s]
	}
	
	\textbf{se explica con}: \tadNombre{diccionario(string,$\sigma$)}
	
	\textbf{usa}: nat, bool
	
	\textbf{géneros}: \TipoVariable{diccNom(string, $\sigma$)}
	
	\tituloModulo{Operaciones de DiccionarioNombres(string,$\sigma$)}
	
	\InterfazFuncion{Vacio}{}{diccNom(string, $\sigma$)}
	{res $\igobs$ vacio}
	[$\Ogr$(1)]
	[Crea un diccionario vacio]
	
	\InterfazFuncion{Definir}{\Inout{d}{diccNom(string,$\sigma$)}, \In{c}{string}, \In{s}{$\sigma$}}{itDiccNom}
	[$\neg$def?(c,d) $\land$ d=$d_0$]
	{d $\igobs$ Definir(c,s,$d_0$) $\land$ haySiguiente(res) $\yluego$ siguiente(res) $\igobs$ <k,s> $\land$ alias(esPermutacion(SecuSuby(res),d))}
	[$\Ogr$(|c| + copy(s))]
	[Agrega un elemento al diccionario y devuelve un iterador apuntando a ese elemento]	
	
	\InterfazFuncion{Borrar}{\Inout{d}{diccNom(string,$\sigma$)}, \In{c}{string}}{}
	[def?(c,d) $\land$ d=$d_0$]
	{d $\igobs$ borrar(c,$d_0$)}
	[$\Ogr$($\#$claves(d) $+$ |c|)]
	[Borra el elemento correspondiente a la clave c del diccionario]
	
	\InterfazFuncion{BorrarRapido}{\Inout{d}{diccNom(string,$\sigma$)}, \In{it}{itDiccNom}}{}
	[def?(c,d) $\land$ d=$d_0$]
	{d $\igobs$ borrar(c,$d_0$)}
	[$\Ogr$(|c|)]
	[Borra el elemento correspondiente a la clave c del diccionario]
	
	\InterfazFuncion{Obtener}{\Inout{d}{diccNom(string,$\sigma$)}, \In{c}{string}}{$\sigma$}
	[def?(c,d) $\land$ d=$d_0$]
	{alias(res $\igobs$ obtener(c,d)) $\land$ ($\forall$ c' : string)( c' $\in$ claves(s) $\impluego$ obtener(c,d) $\igobs$ obtener(c,$d_0$) )}
	[$\Ogr$(|c|)]
	
\end{Interfaz}

\begin{Representacion}

	\tituloModulo{Representacion del DiccionaroNombres(string, $\sigma$)}
	
	\begin{Estructura}{diccNom(string,$\sigma$)}
		\begin{Tupla}
			\tupItem{lineal}{dicc(string,$\sigma$)}
			\tupItem{trie}{diccT(string,$\sigma$)}
		\end{Tupla}
		Donde diccT(string,$\sigma$) es Lista(tupla<$letra$ : char, $significado$ : puntero($\sigma$), $prox$ : diccT(string, $\sigma$)>)
	\end{Estructura}
	
	\Rep{Rep(e.lineal) $\land$ tamaño(e.trie) $<$ 256 $\land$ SinRepetidos(Letras(e.trie)) \\
		($\forall$ dT : diccT(string,$\sigma$)) ( Esta?(dT, e.trie.prox) $\impluego$ Rep(dT) )	
	}
	
	\tadOperacion{Letra}{li}{secu(char)}{}
	\tadAxioma{Letra(l)}{\IF vacio?(l)
							THEN \secuencia{}
							ELSE prim(l).letra $\bullet$ Letra(fin(l))
						FI}
	Donde li es Lista(tupla<$letra$ : char, $significado$ : puntero($\sigma$), $prox$ : diccT(string, $\sigma$)>
	
~	
	
	\tadOperacion{SinRepetidos}{secu($\alpha$)}{bool}{}
	\tadAxioma{SinRepetidos(s)}{\IF vacio?(s)
									THEN true
									ELSE $\neg$esta?(prim(s),fin(s)) $\land$ SinRepetidos(fin(s))
								FI}
	
~
~
	
	
	\AbsFc{diccNom(string, $\sigma$)}{dN : diccNom(string, $\sigma$) $/$ \\
		Abs(e.lineal) $\igobs$ dN \\
		$\land$ ($\forall$ c : string) (Def?(c,d) $\Leftrightarrow$ Definido(c,e.trie))\\
		$\land$ ($\forall$ c : string) (Def?(c,d) $\impluego$ Obtener(c,d) $\igobs$ Significado(c, e.trie))}
	
~
	\tadOperacion{Definido}{string,trie}{bool}{}
	\tadAxioma{Definido(s,t)}{estaLetra(s[0],t) $\yluego$ \IF Tamaño(s) == 1 
															THEN true
															ELSE Definido(c[1:Tamaño(c)-1],t)
														FI}		

~											
	\tadOperacion{estaLetra}{char,trie}{bool}{}
	\tadAxioma{estaLetra(c,t)}{	\IF vacio?(t)
									THEN false
									ELSE Primero(t).letra $\igobs$ c $\lor$ estaLetra(c, fin(t))
								FI}		

~
	\tadOperacion{Significado}{string/s,trie/t}{$\sigma$}{Definido(s,t)}
	\tadAxioma{Significado(s,e)}{
		\IF Tamaño(s) = 1
			THEN *(Buscar(s[0],t).significado)
			ELSE Significado(s[1:Tamaño(s)-1],Buscar(s[0],t).prox)
		FI }
~
	\tadOperacion{Buscar}{char/c,trie/t}{piso}{estaLetra(c,t)}
	\tadAxioma{Buscar(c,t)}{
		\IF primero(t).letra = c
			THEN Primero(t)
			ELSE Buscar(c,fin(t))
		FI}
	
	Donde piso es tupla<$letra$ : char, $significado$ : puntero($\sigma$), $prox$ : trie>
	
~
	
	Donde trie es diccT(string, $\sigma$)

\end{Representacion}

\begin{Algoritmos}
	
	\begin{algorithm}
		\caption{iVacio}
		\begin{algorithmic}
			\Procedure{iVacio}{} $\to res$ : estr
				\State $res.lineal \gets$ Vacio() //vacio de diccionario lineal \Comment $\Ogr$(1)
				\State $res.trie \gets$ Vacio() //vacio de lista \Comment $\Ogr$(1)
			\EndProcedure
		\end{algorithmic}
		\underline{Complejidad:} $\Ogr$(1)
	\end{algorithm}
	
	\begin{algorithm}
		\caption{iDefinir}
		\begin{algorithmic}
			\Procedure{iDefinir}{\texttt{in/out} e : \texttt{estr}, \texttt{in} c : \texttt{string}, \texttt{in} s : \texttt{$\sigma$}} $\to res$ : $itDiccNom$
				\State $res \gets$ DefinirRapido(e.lineal,c,s) \Comment $\Ogr$(copy(s))
				\State $piso \gets$ e.trie \Comment $\Ogr$(1)
				\State $i \gets$ 0
				\While{i $<$ |c|-1} \Comment Se repite |c|-1 veces y lo de adentro es $\Ogr$(1) $\Rightarrow$ $\Ogr$(|c|-1) $=$ $\Ogr$(|c|)
					\If{iEsta(c[i],piso)}
						\State $piso \gets$ iBuscar(c[i], piso).prox \Comment $\Ogr$(1)
					\Else
						\State AgregarAdelante(piso, <c[i],NULL,Vacio()>) \Comment $\Ogr$(1)
						\State $piso \gets$ piso[0].prox
					\EndIf
					\State i++ \Comment $\Ogr$(1)
				\EndWhile
				\If{iEsta(c[i],piso) \Comment $\Ogr$(1)}
					\State iBuscar(c[i],piso).significado $\gets$ puntero(s) \Comment $\Ogr$(1)
				\Else
					\State AgregarAdelante(piso, <c[i],puntero(s),Vacio()>)
				\EndIf
			\EndProcedure
		\end{algorithmic}
		\underline{Complejidad:} $\Ogr$(|c| + copy(s))
	\end{algorithm}
	
	\begin{algorithm}
		\caption{iEsta}

		\InterfazFuncion{Esta}{\In{c}{char}, \Inout{t}{diccT(string,$\sigma$)}}{bool}
		[esta(c,t)]
		{res $\igobs$ estaLetra(c,t)}
		[$\Ogr$(1)]
		[Devuelve verdadero si y sólo si el char de entrada esta en alguno de los elementos de la lista]
		
		\begin{algorithmic}
			\Procedure{iEsta}{\texttt{in} c : \texttt{char}, \texttt{in} t : \texttt{diccT(string,$\sigma$)}} $\to res$ : $bool$
				\State $it \gets$ crearIr(t) \Comment $\Ogr$(1)
				\While{haySiguiente(it) $\land$ Siguiente(it).letra $\neq$ c} \Comment Se repite a lo sumo la cantidad de char, que es una constate $\Rightarrow$ $\Ogr$(1)
					\State Avanzar(it) \Comment $\Ogr$(1)
				\EndWhile
				\State $res \gets$ (HaySiguiente(it)) \Comment $\Ogr$(1)
			\EndProcedure
		\end{algorithmic}
		\underline{Complejidad:} $\Ogr$(1)
	\end{algorithm}
	
	\begin{algorithm}
		\caption{iBuscar}

		\InterfazFuncion{Buscar}{\In{c}{char}, \Inout{t}{diccT(string,$\sigma$)}}{tupla<$letra$ char, $significado$ puntero($\sigma$), $prox$ diccT(string,$\sigma$)>}
		[esta(c,t)]
		{alias(res $\igobs$ buscar(c,t))}
		[$\Ogr$(1)]
		[Busca el elemento de la lista que este caracterizado con el char de entrada]
		[El resultado queda ligado con aliasing]
		
		\begin{algorithmic}
			\Procedure{iBuscar}{\texttt{in} c : \texttt{char}, \texttt{in} t : \texttt{diccT(string,$\sigma$)}} $\to res$ : $tupla<char, puntero(\sigma), diccT(string,\sigma)>$
				\State $it \gets$ crearIt(t)
				\While{Siguiente(it).letra $\neq$ c} \Comment Se repite a lo sumo la cantidad de char, que es una constate $\Rightarrow$ $\Ogr$(1)
					\State Avanzar(it) \Comment $\Ogr$(1)
				\EndWhile
				\State $res \gets$ Siguiente(it)
			\EndProcedure
		\end{algorithmic}
		\underline{Complejidad:} $\Ogr$(1)
	\end{algorithm}
	
	\begin{algorithm}
		\caption{iObtener}
		\begin{algorithmic}
			\Procedure{iObtener}{\texttt{in/out} e : \texttt{estr}, \texttt{in} c : \texttt{string}} $\to res$ : $\sigma$
				\State $piso \gets$ e.trie \Comment $\Ogr$(1)
				\State $i \gets$ 0
				\While{i < |c| - 1} \Comment Se repite |c| - 1 veces y lo de adentro es $\Ogr$(1) $\Rightarrow$ $\Ogr$(|c|)
					\State $piso \gets$ Buscar(c[i],piso).prox \Comment $\Ogr$(1)
					\State i++ \Comment $\Ogr$(1)
				\EndWhile
				\State $res \gets$ *(Buscar(s[i],piso).significado) \Comment $\Ogr$(1)
			\EndProcedure
		\end{algorithmic}
		\underline{Complejidad:} $\Ogr$(|c|)
	\end{algorithm}
	
	\begin{algorithm}
		\caption{iBorrar}
		\begin{algorithmic}
			\Procedure{iBorrar}{\texttt{in/out} e : \texttt{estr}, \texttt{in} c : \texttt{string}}
				\State Borrar(e.lineal, c) \Comment $\Ogr$($\#$claves(e.lineal))
				\State $piso \gets$ e.trie \Comment $\Ogr$(1)
				\State $i \gets$ 0
				\While{i < |c|-1} \Comment Se repite |c| - 1 veces y lo de adentro es $\Ogr$(1) $\Rightarrow$ $\Ogr$(|c|)
					\State $piso \gets$ Buscar(c[i],piso).prox \Comment $\Ogr$(1)
					\State i++ \Comment $\Ogr$(1)
				\EndWhile
				\State delete(Buscar(s[i],piso).significado)
				\State Buscar(s[i],piso).significado $\gets$ NULL \Comment $\Ogr$(1)
			\EndProcedure
		\end{algorithmic}
		\underline{Comlpejidad:} $\Ogr$(|c| + $\#$claves(e.lineal))
	\end{algorithm}
	
	\begin{algorithm}
		\caption{iBorrarRapido}
		\begin{algorithmic}
			\Procedure{iBorrarRapido}{\texttt{in/out} e : \texttt{estr}, \texttt{in} it : \texttt{itDiccNom}}
				\State $c \gets$ SiguienteClave(it) \Comment $\Ogr$(1)
				\State EliminarSiguiente(it) \Comment $\Ogr$(1)
				\State $piso \gets$ e.trie \Comment $\Ogr$(1)
				\State $i \gets$ 0
				\While{i < |c|-1} \Comment Se repite |c| - 1 veces y lo de adentro es $\Ogr$(1) $\Rightarrow$ $\Ogr$(|c|)
					\State $piso \gets$ Buscar(c[i],piso).prox \Comment $\Ogr$(1)
					\State i++ \Comment $\Ogr$(1)
				\EndWhile
				\State delete(Buscar(s[i],piso).significado)
				\State Buscar(s[i],piso).significado $\gets$ NULL \Comment $\Ogr$(1)
			\EndProcedure
		\end{algorithmic}
		\underline{Complejidad:} $\Ogr$(|clave|) donde clave es la clave del siguiente del iterador
	\end{algorithm}
	
\end{Algoritmos}
\subsubsection{Iterador de Diccionario de Nombres}
\begin{Interfaz}

	\textbf{parámetros formales}\hangindent=2\parindent\\
	\parbox{1.7cm}{\textbf{géneros}} $\sigma$\\
	\parbox[t]{1.7cm}{\textbf{función}}\parbox[t]{\textwidth-2\parindent-1.7cm}{%
		\InterfazFuncion{Copiar}{\In{s}{$\sigma$}}{$\sigma$}
	    	{$res \igobs a$}
		[$\Theta(copy(s))$]
		[función de copia de $\sigma$'s]
	}
	
	\textbf{se explica con}: \tadNombre{Iterador Unidireccional Modificable (tupla<string, $\sigma$>)}
	
	\textbf{usa}: nat, bool
	
	\textbf{géneros}: \TipoVariable{itDiccNom}
	
	\tituloModulo{Operaciones de DiccionarioNombres(string,$\sigma$)}
	
	\InterfazFuncion{CrearIt}{\In{d}{DiccNom(string,$\sigma$)}}{itDiccNom}
	{alias(esPermutacion(secuSuby(res),d)) $\land$ vacia?(Anteriores(res))}
	[$\Ogr$(1)]
	[Devuelve un iterador al diccionario]
	
	\InterfazFuncion{HaySiguiente}{\In{it}{itDiccNom}}{bool}
	{res $\igobs$ haySiguiente(it)}	
	[$\Ogr$(1)]
	[Devuelve verdadero si y sólo si el iterador tiene un siguiente elemento]
	
	\InterfazFuncion{Siguiente}{\In{it}{itDiccNom}}{tupla<string,$\sigma$>}
	[haySiguiente(it)]
	{alias(res $\igobs$ siguiente(it))}
	[$\Ogr$(1)]
	
	\InterfazFuncion{Avanzar}{\Inout{it}{itDiccNom}}{}
	[haySiguiente(it) $\land$ it=$it_0$]
	{it = avanzar($it_0$)}	
	[$\Ogr$(1)]

\end{Interfaz}
\subsubsection{Iterador de Claves de Diccionario de Nombres}
\begin{Interfaz}

		\textbf{parámetros formales}\hangindent=2\parindent\\
	\parbox{1.7cm}{\textbf{géneros}} $\sigma$\\
	\parbox[t]{1.7cm}{\textbf{función}}\parbox[t]{\textwidth-2\parindent-1.7cm}{%
		\InterfazFuncion{Copiar}{\In{s}{$\sigma$}}{$\sigma$}
	    	{$res \igobs a$}
		[$\Theta(copy(s))$]
		[función de copia de $\sigma$'s]
	}
	
	\textbf{se explica con}: \tadNombre{Iterador Unidireccional (tupla<string, $\sigma$>)}
	
	\textbf{usa}: nat, bool, DiccionarioNombre(string,$\sigma$), IteradorDiccionario(tupla<string,$\sigma$>)
	
	\textbf{géneros}: \TipoVariable{itClavesDiccN(string)}
	
	\tituloModulo{Operaciones de Iterador de Claves  de DiccionarioNombres(string)}

	\InterfazFuncion{CrearIt}{\Inout{it}{diccNom(string,$\sigma$)}}{itClavesDiccN(string)}
	{res $\igobs$ CrearItUni(Primeros(secuSuby(it)))}
	[$\Ogr$(1)]
	[Crea un iterador no modificable de las claves del diccionario modificable del diccionario nombre]
	
	\tadOperacion{Primeros}{secu(tupla<$\alpha$,$\beta$>)}{secu($\alpha$)}{}
	\tadAxioma{Primeros(s)}{
		\IF vacio?(s)
			THEN <>
			ELSE $\Pi_1$(prim(s)) $\bullet$ Primeros(fin(s))
		FI}	
	
	\InterfazFuncion{HaySiguiente}{\In{it}{itClavesDiccN(tupla<string,$\sigma$>)}}{bool}
	{res $\igobs$ HayMas?(it)}
	[$\Ogr$(1)]
	[Devuelve verdadero si y sólo si el iterador tiene un siguiente elemento]
	
	\InterfazFuncion{SiguienteClave}{\In{it}{itClavesDiccN(tupla<string,$\sigma$>)}}{string}
	[HayMas?(it)]
	{res $\igobs$ $\Pi_1$(Actual(it))}
	[$\Ogr$(1)]
	[Devuelve la siguiente clave del diccionario]
	[res no es modificable]
	
	\InterfazFuncion{Avanzar}{\Inout{it}{itClavesDiccN(tupla<string,$\sigma$>)}}{}
	[HayMas?(it) $\land$ it=$it_0$ ]
	{it $\igobs$ Avanzar($it_0$)}
	[$\Ogr$(1)]
	[Avanza el iterador]
	
	
\end{Interfaz}

\begin{Representacion}

	\begin{Estructura}{itClavesDiccN(string)}[ite]
		Donde $ite$ es $it$ : itDiccNom(tupla<string,$\sigma$>)
	\end{Estructura}
	
~
	
	\Rep[ite][i]{Rep(ite.it)}
	
~

	\AbsFc[ite]{itClavesDiccN(string)}[i]{Abs(i.it)}
	

\end{Representacion}

\begin{Algoritmos}

	\begin{algorithm}[H]
		\caption{iCrearIt}
		\begin{algorithmic}
			\Procedure{iCrearIt}{\texttt{in} d : \texttt{diccNom(string,$\sigma$)}} $\to res$ : $ite$
				\State $res \gets$ CrearIt(d) // crearIt del iterador de Diccionario Nombre \Comment $\Ogr$(1)
			\EndProcedure
		\end{algorithmic}
		\underline{Complejidad:} $\Ogr$(1)
	\end{algorithm}	
	
	\begin{algorithm}[H]
		\caption{iHaySiguiente}
		\begin{algorithmic}
			\Procedure{iHaySiguiente}{\texttt{in} it : \texttt{ite}} $\to res$ : $bool$
				\State $res \gets$ HaySiguiente(it) \Comment $\Ogr$(1)
			\EndProcedure
		\end{algorithmic}
		\underline{Complejidad:} $\Ogr$(1)
	\end{algorithm}
	
	\begin{algorithm}[H]
		\caption{iSiguienteClave}
		\begin{algorithmic}
			\Procedure{iSiguienteClave}{\texttt{in} it : \texttt{ite}} $\to res$ : $string$
				\State $res \gets$ SiguienteClave(ite) \Comment $\Ogr$(1)
			\EndProcedure
		\end{algorithmic}
		\underline{Complejidad:} $\Ogr$(1)
	\end{algorithm}
	
	\begin{algorithm}[H]
		\caption{iAvanzar}
		\begin{algorithmic}
			\Procedure{iAvanzar}{\texttt{in/out} it : \texttt{ite}}
				\State Avanzar(ite) \Comment $\Ogr$(1)
			\EndProcedure
		\end{algorithmic}
		\underline{Complejidad:} $\Ogr$(1)
	\end{algorithm}	
	
\end{Algoritmos}


\subsection{Diccionario Logaritmico}
\begin{Interfaz}
	
	\textbf{parámetros formales}\hangindent=2\parindent\\
	\parbox{1.7cm}{\textbf{géneros}} $\sigma$\\
	\parbox[t]{1.7cm}{\textbf{función}}\parbox[t]{\textwidth-2\parindent-1.7cm}{%
		\InterfazFuncion{Copiar}{\In{s}{$\sigma$}}{$\sigma$}
	    	{$res \igobs a$}
		[$\Theta(copy(s))$]
		[función de copia de $\sigma$'s]
	}
	
	\textbf{se explica con}: \tadNombre{diccionarioAcotado(nat,$\sigma$)}
	
	\textbf{usa}: nat, bool
	
	\textbf{géneros}: \TipoVariable{diccLog(nat, $\sigma$)}
	
	\tituloModulo{Operaciones de DiccionarioLog(nat,$\sigma$)}
	
	\InterfazFuncion{Vacio}{\In{tam}{nat}}{diccLog(nat, $\sigma$)}
	{res $\igobs$ vacio(tam)}
	[$\Ogr$(1)]
	[Crea un diccionario vacío con tam como límite de claves]
	
	\InterfazFuncion{Definir}{\Inout{d}{diccLog(nat, $\sigma$)}, \In{c}{nat}, \In{s}{$\sigma$}}{}
	[($\forall$ c' : nat)( c' $\in$ claves(d) $\Rightarrow$ c $>$ c') $\land$ d=$d_0$ $\land$ $\#$claves(d) $<$ tamaño(d)]
	{d $\igobs$ Definir(c,s,$d_0$)}
	[$\Ogr$(1)]
	
	\InterfazFuncion{Definido}{\Inout{d}{diccLog(nat,$\sigma$)}, \In{c}{nat}}{bool}
	{res $\igobs$ def?(c,d)}
	[$\Ogr$(log $\#$claves(d))]
	
	\InterfazFuncion{Obtener}{\Inout{d}{diccLog(nat,$\sigma$)}, \In{c}{nat}}{$\sigma$}
	[def?(c,d) $\land$ d=$d_0$]
	{alias(res $\igobs$ obtener(c,d)) $\land$ ($\forall$ c' : nat)( (c' $\in$ claves(d) $\land$ c $\neq$ c') 
	$\Rightarrow$ obtener(c',d) $\igobs$ obtener(c,d) )}
	[$\Ogr$(log $\#$claves(d))]
	
\end{Interfaz}

\begin{Representacion}

	\tituloModulo{Representación DiccionarioLog(nat, $\sigma$)}
	
	\begin{Estructura}{diccLog(nat, $\sigma$)}
		\begin{Tupla}
			\tupItem{tamaño}{nat}
			\tupItem{valores}{arreglo dimensionable de tupla<$clave$ : nat, $significado$ : $\sigma$>}
			\tupItem{proxADefinir}{nat}
		\end{Tupla}
	\end{Estructura}
	
	\Rep{SinRepetidos(Claves(e.valores)) $\land$ Ordenada(Claves(e.valores))\\
		e.tamaño $\igobs$ tamaño(e.valores) $\land$ e.proxADefinir $\leq$ e.tamaño}
		
~
	\tadOperacion{SinRepetidos}{secu(nat)}{bool}{}
	\tadAxioma{SinRepetidos(s)}{
		\IF vacio?(s)
			THEN true
			ELSE	 $\neg$Esta?(prim(s),fin(s)) $\land$ sinRepetidos(fin(s))
		FI}
	
~	
	val es arreglo dimensionable de tupla<$clave$ : nat, $significado$ : $\sigma$>
	\tadOperacion{Claves}{val}{secu(nat)}{}
	\tadAxioma{Claves(v)}{
		\IF tamaño(v) = 0
			THEN <>
			ELSE Primero(v).clave $\bullet$ Claves(fin(v))
		FI}
	
~

	\tadOperacion{Ordenada}{secu(nat)}{bool}{}
	\tadAxioma{Ordenada(s)}{
		\IF vacia?(s)
			THEN true
			ELSE ($\forall$ n : nat) (n $\in$ fin(s) $\Rightarrow$ n $>$ Primero(s)) $\land$ Ordenada(fin(s))
		FI}
	
~

	\AbsFc{diccLog(nat,$\sigma$)}{d : diccLog(nat,$\sigma$)/\\
	e.tamaño $\igobs$ tamaño(d)\\
	($\forall$ c : nat)( def?(c,d) $\Leftrightarrow$ ($\exists$ i : nat)( 0 $\leq$ i $<$ e.tamaño) $\land$ definido(e.valores,i) $\yluego$ e.valores[i].clave $\igobs$ c )\\
	($\forall$ c : nat)( def(c,d) $\impluego$ Obtener(c,d) $\igobs$ Significado(c,e) )}
	
~

	\tadOperacion{Significado}{nat,estr}{$\sigma$}{def?(c,d)}
	\tadAxioma{Significado(c,e)}{ e.valores[Posicion(c,e.valores)].significado}
	
~

	\tadOperacion{Posicion}{nat,val}{nat}{}	
	\tadAxioma{Posicion(c,v)}{
		\IF Primero(v).clave = c
			THEN 0
			ELSE 1 + Posicion(c,fin(v))
		FI}
	
\end{Representacion}
\subsubsection{Iterador de Diccionario Logaritmico}
\begin{Interfaz}

	\textbf{parámetros formales}\hangindent=2\parindent\\
	\parbox{1.7cm}{\textbf{géneros}} $\sigma$\\
	\parbox[t]{1.7cm}{\textbf{función}}\parbox[t]{\textwidth-2\parindent-1.7cm}{%
		\InterfazFuncion{Copiar}{\In{s}{$\sigma$}}{$\sigma$}
	    	{$res \igobs a$}
		[$\Theta(copy(s))$]
		[función de copia de $\sigma$'s]
	}
	
	\textbf{se explica con}: \tadNombre{Iterador Unidireccional Modificable(tupla<nat,$\sigma$>)}
	
	\textbf{usa}: nat, bool, diccionarioLog(nat,$\sigma$)
	
	\textbf{géneros}: \TipoVariable{itDiccLog(tupla<nat, $\sigma$>)}
	
	\tituloModulo{Operaciones de Iterador DiccionarioLog(nat, $\sigma$)}


	\InterfazFuncion{HaySiguiente}{\In{it}{itDiccLog(tupla<nat,$\sigma$>)}}{bool}
	{res $\igobs$ HayMas?(it)}
	[$\Ogr$(1)]
	[Devuelve verdadero si y sólo si quedan elementos que recorrer en el iterador.]
	
	\InterfazFuncion{SiguienteClave}{\In{it}{itDiccLog(tupla<nat, $\sigma$>)}}{nat}
	[HayMas?(it)]
	{res $\igobs$ $\Pi_1$(Siguiente(it))}
	[$\Ogr$(1)]
	[Devuelve la clave del siguiente elemento del iterador.]
	[res no es modificable.]
	
	\InterfazFuncion{SiguienteSignificado}{\Inout{it}{itDiccLog(tupla<nat, $\sigma$>)}}{$\sigma$}
	[HayMas?(it)]
	{alias(res $\igobs$ $\Pi_2$(Siguiente(it)))}
	[$\Ogr$(1)]
	[Devuelve el siguiente del siguiente elemento del iterador.]
	[res es modificable si y sólo si $\sigma$ es modificable.]

\end{Interfaz}	

\begin{Representacion}
	
	\begin{Estructura}{itDiccLog(tupla<nat, $\sigma$>)}
		\begin{Tupla}
			\tupItem{pos}{nat}
			\tupItem{dic}{puntero(estr)}
		\end{Tupla}
	\tab \hspace{1.2mm}donde $estr$ es la representación del diccionarioLog(nat,$\sigma$)
	\end{Estructura}
		
~
	
	\Rep[ite][i]{i.pos $\leq$ *dic.proxADefinir}

~

	\Abs[ite]{itDiccLog(tupla<nat,$\sigma$>)}[i]{itL}{Siguientes(itL) $\igobs$ \\
	SecuenciarClavesArreglo(*dic.valores, *dic.tamaño, i.pos)}
	
~

	\tadOperacion{SecuenciarClavesArreglo}{arreglo(tupla<$clave$ : nat, $significado$ : $\sigma$>),nat,nat}{secu(nat)}{}
	\tadAxioma{SecuenciaClavesArreglo(var,tam,pos)}{
		\IF pos $\igobs$ tam
			THEN <>
			ELSE var[pos].clave $\bullet$ SecuenciarArreglo(var,tam,pos $+$ 1 )
		FI}	
	
\end{Representacion}

\begin{Algoritmos}
	
	\begin{algorithm}[H]
		\caption{iHaySiguiente}
		\begin{algorithmic}
			\Procedure{iHaySiguiente}{\texttt{in} i : \texttt{ite}} $\to res$ : $bool$
				\State $res \gets$ (i.pos $<$ *(i.dic).tamaño) \Comment $\Ogr$(1)
			\EndProcedure
		\end{algorithmic}
		\underline{Complejidad:} $\Ogr$(1)
	\end{algorithm}

	\begin{algorithm}[H]
		\caption{iSiguienteClave}
		\begin{algorithmic}
			\Procedure{iSiguienteClave}{\texttt{in} i : \texttt{ite}} $\to res$ : $nat$
				\State $res \gets$ *(i.dic).valores[i.pos].clave \Comment $\Ogr$(1)
			\EndProcedure
		\end{algorithmic}
		\underline{Complejidad:} $\Ogr$(1)
	\end{algorithm}
	
	\begin{algorithm}[H]
		\caption{iSiguienteSignificado}
		\begin{algorithmic}
			\Procedure{iSiguienteClave}{\texttt{in} i : \texttt{ite}} $\to res$ : $nat$
				\State $res \gets$ *(i.dic).valores[i.pos].significado \Comment $\Ogr$(1)
			\EndProcedure
		\end{algorithmic}
		\underline{Complejidad:} $\Ogr$(1)
	\end{algorithm}
	
\end{Algoritmos}
\subsubsection{Iterador de Claves de Diccionario Logaritmico}
\begin{Interfaz}

	\textbf{se explica con}: \tadNombre{Iterador Unidireccional (nat)}
	
	\textbf{usa}: nat, bool, DiccionarioLog(nat,$\sigma$)
	
	\textbf{géneros}: \TipoVariable{itClavesDiccLog(nat)}
	
	\tituloModulo{Operaciones de Iterador de Claves  de DiccionarioPlacas(nat)}

	\InterfazFuncion{CrearIt}{\In{d}{diccLog(nat,$\sigma$)}}{itClavesDiccLog(nat)}
	{Permutacion( Siguientes(res), Secuenciar(claves(d)) )}
	[$\Ogr$(1)]
	[Crea un iterador no modificable de las claves del diccionario  de entrada. El iterador se invalida si se modifica el diccionario.]

~
	
	\tadOperacion{Secuenciar}{conj(string)}{secu(string)}{}
	\tadAxioma{Secuenciar(cs)}{
		\IF $\emptyset$?(cs)
			THEN <>
			ELSE dameUno(cs) $\bullet$ Secuenciar(sinUno(cs))
		FI}	

~
	
	\tadOperacion{Permutacion}{secu(nat),secu(nat)}{bool}{}
	\tadAxioma{Permutacion(s,t)}{ Iguales( MultiConjuntizar(s), MultiConjuntizar(t) )}

~
	
	\tadOperacion{MultiConjuntizar}{secu(nat)}{multiconj(nat)}{}
	\tadAxioma{MultiConjuntizar(s)}{
		\IF vacio?(s)
			THEN $\emptyset$
			ELSE Ag(prim(s), MultiConjuntizar(fin(s)) )
		FI}

~

	\tadOperacion{Iguales}{multiconj(nat),multiconj(nat)}{bool}{}
	\tadAxioma{Iguales(m$_1$, m$_2$)}{$\#$(dameUno(m$_1$),m$_1$) $=$ $\#$(dameUno(m$_1$),m$_1$)\\
		$\land$ Iguales(sinUno(m$_1$), m$_2$ $-\{$dameUno(m$_1$)$\}$ )}
		

~
	
	\InterfazFuncion{HaySiguiente}{\In{it}{itClavesDiccLog(nat)}}{bool}
	{res $\igobs$ HayMas?(it)}
	[$\Ogr$(1)]
	[Devuelve verdadero si y sólo si el iterador tiene un siguiente elemento]
	
	\InterfazFuncion{SiguienteClave}{\In{it}{itClavesDiccLog(nat)}}{string}
	[HayMas?(it)]
	{res $\igobs$ $\Pi_1$(Actual(it))}
	[$\Ogr$(1)]
	[Devuelve la siguiente clave del diccionario]
	[res no es modificable]
	
	\InterfazFuncion{Avanzar}{\Inout{it}{itClavesDiccLog(nat)}}{}
	[HayMas?(it) $\land$ it=$it_0$ ]
	{it $\igobs$ Avanzar($it_0$)}
	[$\Ogr$(1)]
	[Avanza el iterador]	
	
\end{Interfaz}

\begin{Representacion}

	\begin{Estructura}{itClavesDiccLog(nat)}[ite]
		Donde $ite$ es $pos$ : $nat$
	\end{Estructura}	
	
~	

	\Rep[ite][i]{pos $\leq$ e.proxADefinir}
	
~

	\Abs[ite]{itClavesDiccLog(nat)}[i]{itC}{Siguientes(itC) $\igobs$ SecuenciarClavesArreglo(e.valores, e.tamaño, i.pos)}
	
~

	\tadOperacion{SecuenciarClavesArreglo}{arreglo(tupla<$clave$ : nat, $significado$ : $\sigma$>),nat,nat}{secu(nat)}{}
	\tadAxioma{SecuenciaClavesArreglo(var,tam,pos)}{
		\IF pos $\igobs$ tam
			THEN <>
			ELSE var[pos].clave $\bullet$ SecuenciarArreglo(var,tam,pos $+$ 1 )
		FI}	
	
\end{Representacion}

\begin{Algoritmos}
	
	\begin{algorithm}[H]
		\caption{iCrearIt}
		\begin{algorithmic}
			\Procedure{iCrearIt}{\texttt{in} e : \texttt{estr}} $\to res$ : $ite$
				\State $	res.pos \gets$ 0 \Comment $\Ogr$(1)
			\EndProcedure
		\end{algorithmic}
		\underline{Complejidad:} $\Ogr$(1)
	\end{algorithm}	

	\begin{algorithm}[H]
		\caption{iHaySiguiente}
		\begin{algorithmic}
			\Procedure{iHaySiguiente}{\texttt{in} i : \texttt{ite}} $\to res$ : $bool$
				\State $res \gets$ ( i.pos = e.tamaño )	
			\EndProcedure
		\end{algorithmic}
		\underline{Complejidad:} $\Ogr$(1)
	\end{algorithm}
	
	\begin{algorithm}[H]
		\caption{iSiguienteClave}
		\begin{algorithmic}
			\Procedure{iSiguienteClave}{\texttt{in} i : \texttt{ite}} $\to res$ : $nat$
				\State $res \gets$ e.valores[i.pos] \Comment $\Ogr$(1)
			\EndProcedure
		\end{algorithmic}
		\underline{Complejidad:} $\Ogr$(1)
	\end{algorithm}
	
	\begin{algorithm}[H]
		\caption{iAvanzar}
		\begin{algorithmic}
			\Procedure{iAvanzar}{\texttt{in/out} i : \texttt{ite}}
				\State $i.pos \gets$ i.pos $+$ 1 \Comment $\Ogr$(1)
			\EndProcedure
		\end{algorithmic}
		\underline{Complejidad:} $\Ogr$(1)
	\end{algorithm}
	
\end{Algoritmos}


\subsection{Diccionario de Sanciones}
\begin{Interfaz}

	\textbf{se explica con}: \tadNombre{Diccionario(nat, conj(placa)), Iterador Bidireccional($\alpha$)}.

	\textbf{géneros}: \TipoVariable{diccS, its}.

	\tituloModulo{Operaciones básicas de diccionario de sanciones}

	\InterfazFuncion{CrearDiccS}{\In{significado}{conj(placa)}}{diccS}
	{$res \igobs$ definir($0$, significado, vacio)}
	[$\Ogr$(1)]
	[genera un diccionario con todos los agentes con $0$ sanciones]

	\InterfazFuncion{Definir}{\Inout{ds}{diccS}, \In{s}{nat}, \In{significado}{conj(placa)}}{its}
	[ds$_0$=ds $\land$ ($\forall$ i: nat) i $\in$ claves(ds) $\Rightarrow$ s $\geq$ i]
	{(ds $\igobs$ definir(s, significado, ds)) $\land$ haySiguiente($res$) $\land_l$ siguiente($res$)=$<$s, significado$>$ $\land$\\
	secuSuby($res$)$\igobs$ordenar$\pi$1(secuenciarDic(ds))}%uso el definir de diccLineal
	[$\Ogr$(1)]
	[defino la clave $s$ con el significado $significado$ en mi diccionario $ds$]
	
	\InterfazFuncion{Obtener}{\Inout{ds}{diccS}, \In{s}{nat}}{conj(placa)}
	{$res \igobs$ obtener(s, ds)}
	[$\Ogr$(1)]
	[me retorna el conjunto de placas con $s$ sanciones del $ds$]

	\InterfazFuncion{Def?}{\In{ds}{diccS}, \In{s}{nat}}{bool}
	[def?(s, ds)]
	{$res \igobs$ def?(s, ds)}  
	[$\Ogr$(1)]
	[me dice si esta definida la clave $s$ en el diccionario $ds$]
	
	
	\tituloModulo{Operaciones básicas de iterador de sanciones}
	
	Para simplificar las cosas usaremos clave y significado en vez de $\Pi1$ y $\Pi2$ cuanto utilicemos una tupla(nat, conj(placa))
	
	\InterfazFuncion{CrearIts}{\In{ds}{diccS}}{itS}
	{SecuSuby($res$)$\igobs$ordenar$\pi1$(secuenciarDic(ds))}
	[$\Ogr$(1)]
	[crea iterador de sanciones]
	
	\InterfazFuncion{HaySiguiente}{\In{its}{itS}}{bool}
	{$res \igobs$ haySiguiente?(its)}
	[$\Ogr$(1)]
	[me retorna true si hay quedan elementos para avanzar]
	
	\InterfazFuncion{Siguiente}{\In{its}{itS}}{tupla(nat, conj(placa))}
	[haySiguiente?(its)]
	{alias($res \igobs$ siguiente(its))}
	[$\Ogr$(1)]
	[devuelve el elemento siguiente del iterador]
	
	\InterfazFuncion{SiguienteClave}{\In{its}{itS}}{nat}
	[haySiguiente?(its)]
	{alias($res \igobs$ siguiente(its).clave)}
	{$\Ogr$(1)}
	{devuelve la clave del siguiente iterador}
	
	\InterfazFuncion{SiguienteSignificado}{\In{its}{itS}}{conj(placa)}
	[haySiguiente?(its)]
	{alias($res \igobs$ siguiente(its).significado)}
	{$\Ogr$(1)}
	{devuelve el significado del siguiente iterador}


\end{Interfaz}

\begin{Representacion}
	
	\tituloModulo{Representacion del Diccionaro de Sanciones}
	\begin{Estructura}{diccS}
		\begin{Tupla}
			\tupItem{diccLineal}{dicc(nat, conj(placa))}
			\tupItem{diccLog}{diccLog(nat, con(placa))}
			\tupItem{actualizar}{bool}
		\end{Tupla}
	\end{Estructura}
	
	\Rep{$\neg$e.actualizar $\Rightarrow$ (($\forall$ n: nat)(n $\in$ claves(e.diccLog) $\iff$ n $\in$ claves(e.diccLineal))$\land_l$ ($\forall$ cp: conj(placa)) cp$=$obtener(n, e.diccLog)$\iff$ cp$=$obtener(n, e.diccLineal))
		\\
		$\land$e.actualizar$\Rightarrow$ Union(Significados(e.diccLog))$=$Union(Significados(e.diccLineal))}
	
	\tadOperacion{Significados}{dicc(nat, conj(placa))}{conj(conj((placa)))}{}
	\tadAxioma{Significados(d)}{\IF $\#$claves(d)$=0$
		THEN $\emptyset$
		ELSE Ag(obtener(dameUno(claves(d)), Significados(borrar(dameUno(claves(d)),d)))
		FI}
	Uso el Significados(dicc(nat, conj(placa))) para mi diccLineal y diccLog aunque deberia hacer uno especifico para el diccLog, pero es trivial y es mas amigable a la vista
	
	~	
	
	\tadOperacion{Union}{conj(conj(plc))}{conj(plc)}{}
	\tadAxioma{Union(c)}{\IF $\#$c$=0$
		THEN $\emptyset$
		ELSE dameUno(c)$\cup$Union(sinUno(c))
		FI}
	
	~
	~
	
	
	\AbsFc{diccS}{d : diccS $/$ \\
		($\forall$ n : nat) n$\in$claves(e.diccLineal) $\Rightarrow_L$ obtener(n, e.diccLineal$=$obtener(n, d))\\
		$\land$ def?(n, e.diccLineal) $=$ Def?(n, d) }
	
	~
	~
	
	
	\tituloModulo{Representacion del Iterador de Diccionaro de Sanciones}
	\begin{Estructura}{itS}[it]
		Donde $it$ es itDicc(nat, conj(placa))
	\end{Estructura}
	
	\Rep[itS][it]{true}
	
	~
	~
	
	
	\AbsFc[itS]{itBi(tupla(nat, conj(placa)))}[i]{crearItBi(Anteriores(it), Siguientes(it))}

\begin{Algoritmos}
	
	\begin{algorithm}[H]
		\caption{iCrearDiccS}
		\begin{algorithmic}[1]
			\Procedure{iCrearDiccS}{\texttt{in} sig : \texttt{conj(placa)}} $\to res$ : estr
			\State $res$.diccLog$\leftarrow$ Vacio($\#$(sig)) //vacio de diccionario logaritmico \Comment $\Ogr$(1)
			\State $res$.diccLineal$\leftarrow$ Vacio() //vacio de diccionario lineal \Comment $\Ogr$(1)
			\State $res$.actualizar $\leftarrow$ false \Comment $\Ogr$(1) 
			\EndProcedure
		\end{algorithmic}
		\underline{Complejidad:} $\Ogr$(1)
	\end{algorithm}
	
	\begin{algorithm}[H]
		\caption{iDefinir}
		\begin{algorithmic}[1]
			\Procedure{iDefinir}{\texttt{in/out} e : \texttt{estr}, \texttt{in} s : \texttt{nat}, \texttt{in} sig : \texttt{conj(placa)}} $\to res$ : itS
			\State it$\leftarrow$ Definir(e.diccLineal, s, sig) //definir de diccLineal \Comment $\Ogr$($\#$sig)
			\State $res \leftarrow$ it \Comment $\Ogr$(1)
			\State $res$.actualizar $\leftarrow$ true \Comment $\Ogr$(1) 
			\EndProcedure
		\end{algorithmic}
		\underline{Complejidad:} $\Ogr$($\#$(sig))
	\end{algorithm}

	\begin{algorithm}[H]
		\caption{iObtener}
		\begin{algorithmic}[1]
			\Procedure{iObtener}{\texttt{in/out} e : \texttt{estr}, \texttt{in} s : \texttt{nat}} $\to res$ : conj(placa)
			\If{e.actualizar} \Comment $\Ogr$(1)
				\State it $\leftarrow$ CrearIt(e) \Comment $\Ogr$(1)
				\While{HaySiguiente(it)} // $\Ogr$(1) cantidad de claves de diccLineal veces \Comment $\Ogr$($\#$claves(e.diccLineal))
					\State e.diccLog $\leftarrow$ Definir(e.diccLog, SiguienteClave(it), SiguienteSignificado(it)) //Definir de diccLog \Comment $\Ogr$(1) 
					\State avanzar(it) \Comment $\Ogr$(1) 
				\EndWhile
				\State e.actualizar $\leftarrow$ false \Comment $\Ogr$(1) 
			\EndIf
			\State Obtener(e.diccLog, s) $\Ogr$(1)
			\EndProcedure
		\end{algorithmic}
		\underline{Complejidad:} $\Ogr$($\#$claves(e.diccLineal))
		\\
		\underline{Justificacion:} $\Ogr$($\#$claves(e.diccLineal))+$\Ogr$(log $\#$claves(e.diccLineal)). En el peor de los casos sera la $\Ogr$($\#$claves(e.diccLineal)) pues $\Ogr$($\#$claves(e.diccLog))$>\Ogr$(log $\#$claves(e.diccLineal)). Y en el peor de los casos $\#$claves(e.diccLog) sera igual a la cantidad de agentes (esto sera cuando todos los agentes tengan diferentes sanciones). Cuando la cantidad de sanciones se mantenga (e.actualizar $=$ false) entonces tendremos una complejidad de $\Ogr$(log $\#$claves(e.diccLineal)) puesto que nunca entraremos en el while.
	\end{algorithm}	
	
	
\end{Algoritmos}
\end{Representacion}
\subsubsection{Iterador de Diccionario de Sanciones}
\begin{Interfaz}

	\textbf{se explica con}: \tadNombre{Iterador Bidireccional($\alpha$)}.

	\textbf{usa}: nat, bool, diccS, tupla, conj(placa)

	\textbf{géneros}: \TipoVariable{its}.

	\tituloModulo{Operaciones básicas de iterador de sanciones}
	
	Para simplificar las cosas usaremos clave y significado en vez de $\Pi1$ y $\Pi2$ cuanto utilicemos una tupla(nat, conj(placa))
	
	\InterfazFuncion{CrearIts}{\In{ds}{diccS}}{itS}
	{SecuSuby($res$)$\igobs$ordenar$\pi1$(secuenciarDic(ds))}
	[$\Ogr$(1)]
	[crea iterador de sanciones]
	
	\InterfazFuncion{HaySiguiente}{\In{its}{itS}}{bool}
	{$res \igobs$ haySiguiente?(its)}
	[$\Ogr$(1)]
	[me retorna true si hay quedan elementos para avanzar]
	
	\InterfazFuncion{Siguiente}{\In{its}{itS}}{tupla(nat, conj(placa))}
	[haySiguiente?(its)]
	{alias($res \igobs$ siguiente(its))}
	[$\Ogr$(1)]
	[devuelve el elemento siguiente del iterador]
	
	\InterfazFuncion{SiguienteClave}{\In{its}{itS}}{nat}
	[haySiguiente?(its)]
	{alias($res \igobs$ siguiente(its).clave)}
	{$\Ogr$(1)}
	{devuelve la clave del siguiente iterador}
	
	\InterfazFuncion{SiguienteSignificado}{\In{its}{itS}}{conj(placa)}
	[haySiguiente?(its)]
	{alias($res \igobs$ siguiente(its).significado)}
	[$\Ogr$(1)]
	[devuelve el significado del siguiente iterador]
	
	\InterfazFuncion{Avanzar}{\Inout{its}{itS}}{itS}
	[it$=$it$_0$ $\land$ haySiguiente?(its)]
	{it $\igobs$ avanzar(it$_0$)}
	[$\Ogr$(1)]
	[avanza en uno el iterador]

	\InterfazFuncion{EliminarSiguiente}{\Inout{its}{itS}}{}
	[it$=$it$_0$ $\land$ haySiguiente?(its)]
	{it $\igobs$ eliminarSiguiente(it$_0$)}
	[$\Ogr$(1)]
	[elimina la clave del elemento que se encuentra en la posicion siguiente]


\end{Interfaz}

\begin{Representacion}
	
	
	\tituloModulo{Representacion del Iterador de Diccionaro de Sanciones}
	\begin{Estructura}{itS}[itDic]
		Donde $itDic$ es itDicc(nat, conj(placa))
	\end{Estructura}
	
	\Rep[itS][it]{true}
	
	~
	~
	
	
	\AbsFc[itS]{itBi(tupla(nat, conj(placa)))}[i]{crearItBi(Anteriores(it), Siguientes(it))}

\begin{Algoritmos}
	
	\begin{algorithm}[H]
		\caption{iCrearIt}
		
		\begin{algorithmic}[1]
			\Procedure{iCrearIt}{\texttt{in} e : \texttt{estr}} $\to res$ itDic
			\State $res \leftarrow$ CrearIt(e.diccLineal) //uso el crearIt de diccLineal \Comment $\Ogr$(1)
			\EndProcedure 
		\end{algorithmic}
		\underline{Complejidad:} $\Ogr$(1)
	\end{algorithm}
	
	\begin{algorithm}[H]
		\caption{iHaySiguiente}
		
		\begin{algorithmic}[1]
			\Procedure{iHaySiguiente}{\texttt{in} itd : \texttt{itDic}} $\to res$ bool
			\State $res \leftarrow$ HaySiguiente(itd) //uso el HaySiguiente de itDicc \Comment $\Ogr$(1)
			\EndProcedure 
		\end{algorithmic}
		\underline{Complejidad:} $\Ogr$(1)
	\end{algorithm}
	
	\begin{algorithm}[H]
		\caption{iSiguiente}
		
		\begin{algorithmic}[1]
			\Procedure{iSiguiente}{\texttt{in} itd : \texttt{itDic}} $\to res$ $<$nat, conj(placa)$>$
			\State $res \leftarrow$ Siguiente(itd) //uso el Siguiente de itDicc \Comment $\Ogr$(1)
			\EndProcedure 
		\end{algorithmic}
		\underline{Complejidad:} $\Ogr$(1)
	\end{algorithm}
	
	\begin{algorithm}[H]
		\caption{iSiguienteClave}
		
		\begin{algorithmic}[1]
			\Procedure{iSiguienteClave}{\texttt{in} itd : \texttt{itDic}} $\to res$ nat
			\State $res \leftarrow$ SiguienteClave(itd) //uso el SiguienteClave de itDicc \Comment $\Ogr$(1)
			\EndProcedure 
		\end{algorithmic}
		\underline{Complejidad:} $\Ogr$(1)
	\end{algorithm}
	
	\begin{algorithm}[H]
		\caption{iSiguienteSignificado}
		
		\begin{algorithmic}[1]
			\Procedure{iSiguienteSignificado}{\texttt{in} itd : \texttt{itDic}} $\to res$ conj(placa)
			\State $res \leftarrow$ SiguienteSignificado(itd) //uso el SiguienteSignificado de itDicc \Comment $\Ogr$(1)
			\EndProcedure 
		\end{algorithmic}
		\underline{Complejidad:} $\Ogr$(1)
	\end{algorithm}
	
	\begin{algorithm}[H]
		\caption{iAvanzar}
		
		\begin{algorithmic}[1]
			\Procedure{iAvanzar}{\texttt{in/out} itd : \texttt{itDic}} $\to res$ itDic
			\State $res \leftarrow$ Avanzar(itd) //uso el Avanzar de itDicc \Comment $\Ogr$(1)
			\EndProcedure 
		\end{algorithmic}
		\underline{Complejidad:} $\Ogr$(1)
	\end{algorithm}
	
	\begin{algorithm}[H]
		\caption{iEliminarSiguiente}
		
		\begin{algorithmic}[1]
			\Procedure{iEliminarSiguiente}{\texttt{in/out} itd : \texttt{itDic}, \texttt{in/out} e : \texttt{estr}}
			\State EliminarSiguiente(itd) //uso el EliminarSiguiente de itDicc \Comment $\Ogr$(1)
			\State e.actualizar $\leftarrow$ true \Comment $\Ogr$(1)
			\EndProcedure 
		\end{algorithmic}
		\underline{Complejidad:} $\Ogr$(1)
	\end{algorithm}	
	
\end{Algoritmos}
\end{Representacion}

\subsection{Diccionario$_H$}
\begin{Interfaz}

	\textbf{parámetros formales}\hangindent=2\parindent\\
	\parbox{1.7cm}{\textbf{géneros}} $\sigma$\\
	
	\textbf{usa}: nat, bool, arreglo Dimensionable
	
	\textbf{géneros}: \TipoVariable{dicch(nat,$\sigma$)}
	
	\tituloModulo{Operaciones de Dicch(nat,$\sigma$)}
	
	\InterfazFuncion
	{Vacio}
	{\In{cant}{nat}}
	{dicch(nat, $\sigma$)}
	[cant >\ 0]
	{res $\igobs$ vacio}
	[$\Ogr$(n)]
	[Crea un diccionario vacio.]
	
	\InterfazFuncion
	{Definir}
	{\In{clave}{nat}, \In{significado}{$\sigma$}, \Inout{dicc}{dicch(nat,$\sigma$)}}
	{}
	[$\neg$def?(clave,dicc) $\land$ dicc=$dicc_0$]
	{dicc $\igobs$ Definir(clave,significado,$dicc_0$)}
	[$\Ogr(1)$ en caso promedio]
	[Agrega un elemento al diccionario.]	
	
	\InterfazFuncion
	{Obtener}
	{\In{clave}{nat}, \Inout{dicc}{dicch(nat,$\sigma$)}}
	{$\sigma$}
	[def?(clave,dicc) $\land$ dicc=$dicc_0$]
	{alias(res $\igobs$ obtener(clave,dicc)) 
		$\land$\\ 
		($\forall$ clave' : nat)( (clave' $\in$ claves(dicc) $\land$ clave $\neq$ clave' )$\impluego$ obtener(clave',dicc) $\igobs$ obtener(clave',$dicc_0$) )}
	[$\Ogr(1)$ en caso promedio]
	
\end{Interfaz}

\begin{Representacion}

	\tituloModulo{Representacion del DiccionaroHash(nat, $\sigma$)}
	
	\begin{Estructura}{dicch(nat,$\sigma$)}
		\begin{Tupla}
			\tupItem{cant}{nat}
			\tupItem{valores}{Adt}
		\end{Tupla}
		Donde Adt es arreglo Dimensionable de tupla<$clave$ : nat, $significado$ : $\sigma$>)
	\end{Estructura}
	
	\Rep{ e.cant $\igobs$ Tamaño(e.valores) $\yluego$ \\ 
	($\forall$ i, j : nat) 
	( i <\ e.cant $\land$ j <\ e.cant $\land$ i $\neq$ j $\land$ definido(e.valores,i) $\land$ definido(e.valores,j)) \\
	$\impluego$ e.valores[i].clave $\neq$ e.valores[j].clave
	}
	
~
	
	\AbsFc
	{dicch(nat, $\sigma$)}
	{dic : dicch(nat, $\sigma$) $/$ \\
		($\forall$ clave : nat) \\
		( Def?(clave,dic) $\Leftrightarrow$ ($\exists$ i : nat)
		( 0 $\leq$ i <\ e.cant $\land$ Definido( e.valores, i ) 
		$\yluego$ e.valores[i].clave = clave )) 
		\\	$\land$ ( Def?(clave,dic) 	\\
		$\impluego$ Obtener(clave,dic) $\igobs$ 
		e.valores[ mod( hash(clave, e ) + corrimiento(hash(clave, e), e) , cant) ].significado}
~
	\tadOperacion{hash}{nat,estr}{nat}{}
	\tadAxioma{hash(clave, e)}{mod(clave, e.cant)}		
~											
	\tadOperacion{corrimiento}{nat,estr}{nat}{}
	\tadAxioma{corrimiento(clave, e)}
	{\IF e.valores[ mod( clave, e.cant) ].clave = clave
		THEN 0
		ELSE 1 + corrimiento(clave + 1, e)
	FI}		

\end{Representacion}

\begin{Algoritmos}
	
	\begin{algorithm}
		\caption{iVacio}
		\begin{algorithmic}
			\Procedure
			{iVacio}
			{\texttt{in} n : nat} $\to res$ : estr
				\State var $e$ : estr
				\State $e.cant \gets$ n \Comment $\Theta$(1)
				\State $e.valores \gets$ CrearArreglo($n$) \Comment $\Ogr$(n)
				\State $res \gets$ e \Comment $\Theta$(1)
			\EndProcedure
		\end{algorithmic}
		\underline{Complejidad:} $\Ogr$(n)
	\end{algorithm}
	
	\begin{algorithm}
		\caption{iDefinir}
		\begin{algorithmic}
			\Procedure
			{iDefinir}
			{\texttt{in} clave : \texttt{nat}, \texttt{in} significado : \texttt{$\sigma$, \texttt{in/out} e : \texttt{estr}}}
				\State var $h$ : nat
				\State $h \gets$ mod(clave, e.cant) \Comment $\Ogr$(1)
				\While{Definido(e.valores, h)} \Comment En promedio no se entra al ciclo
						\State $h \gets$ mod(h+1, e.cant) \Comment $\Ogr$(1)
				\EndWhile
				\State $e.valores[h] \gets$ tupla<clave, significado> \Comment $\Ogr$(1)
			\EndProcedure
		\end{algorithmic}
		\underline{Complejidad:} $\Ogr$(1) promedio
		\\
		\underline{Justificación:} sabiendo que hay una distribución uniforme de 'claves', podemos asumir que en promedio no se entraría al ciclo, cerrando en promedio una suma de operaciones de complejidad $\Ogr$(1).
	\end{algorithm}
	
	\begin{algorithm}
		\caption{iObtener}
		\begin{algorithmic}
			\Procedure
			{iObtener}
			{\texttt{in} clave : \texttt{nat},\texttt{in/out} e : \texttt{estr}} $\to res$ : $\sigma$
				\State var $h$ : nat				
				\State $h \gets$ mod(clave, e.cant) \Comment $\Ogr$(1)
				\While{e.valores[h].clave $\neq$ clave} 
				\Comment En promedio no se entra al ciclo
					\State $h \gets$  mod(h+1, e.cant) \Comment $\Ogr$(1)
				\EndWhile
				\State $res \gets$ e.valores[h].significado \Comment $\Ogr$(1)
			\EndProcedure
		\end{algorithmic}
		\underline{Complejidad:} $\Ogr$(1) promedio \\
		\underline{Justificación:} Misma justificación que el algoritmo anterior
	\end{algorithm}
	
\end{Algoritmos}

\subsection{Posicion}
\begin{Interfaz}
	
	\textbf{se explica con}: \tadNombre{TAD Posicion}.
	
	\textbf{géneros}: \TipoVariable{posicion}.
	
	\tituloModulo{Operaciones básicas de posicion}
	
	\InterfazFuncion{CrearPosicion}{\In{x}{nat},\In{y}{nat}}{posicion}
	{$res \igobs$ <x,y>}
	[$\Theta$(1)]
	[Crea una nueva instancia de posición haciendo copia por valor de los parámetros de entrada.]
	
	\InterfazFuncion{X}{\In{p}{posicion}}{nat}
	{alias($res \igobs$ X(p))}
	[$\Theta$(1)]
	[Devuelve el valor de la variable "X" de la posición]
	[Hay aliasing entre el resultado de la función y el valor "X" de la instancia posicion.]
	
	\InterfazFuncion{Y}{\In{p}{posicion}}{nat}
	{alias($res \igobs$ Y(p))}
	[$\Theta$(1)]
	[Devuelve el valor de la variable "Y" de la posicion]
	[Hay aliasing entre el resultado de la función y el valor "Y" de la instancia posicion.]
	
	\InterfazFuncion{SumarPosiciones}
	{\In{p_1}{posicion},\In{p_2}{posicion}}{posicion}
	{$res \igobs$ <X(p$_1$) $+$ X(p$_2$), Y(p$_1$) $+$ Y(p$_2$)>}
	[$\Theta$(1)]
	[Devuelve una nueva instancia del tipo posicion, resultado de sumar ambas X's e Y's.]
	
	\InterfazFuncion{RestarPosiciones}
	{\In{p_1}{posicion},\In{p_2}{posicion}}{posicion}
	[Y(p$_2$) $<$ Y(p$_1$) $\land$ X(p$_2$) $<$ X(p$_1$)]
	{$res \igobs$ <X(p$_1$) $-$ X(p$_2$), Y(p$_1$) $-$ Y(p$_2$)>}
	[$\Theta$(1)]
	[Devuelve una nueva instancia del tipo posicion, resultado de sumar ambas X's e Y's.]
	
	\InterfazFuncion{DistanciaPosiciones}{\In{p_1}{posicion},\In{p_2}{posicion}}{nat}
	{$res  \igobs$ Max(X(p$_1$), X(p$_2$)) - Min(X(p$_1$),X(p$_2$)) $+$ Max(Y(p$_1$),Y(p$_2$)) - Max(Y(p$_1$),Y(p$_2$)) }
	[$\Theta$(1)]
	[Devuelve el resultado de calcular la distancia entre dos instancias de posicion, en base a sus obserdores X e Y.]
	
\end{Interfaz}

\begin{Representacion}
	
	\tituloModulo{Representación de posicion}
	
	\begin{Estructura}{posicion}[estr]
		\begin{Tupla}[estr]
			\tupItem{x}{nat}%
			\tupItem{y}{nat}%
		\end{Tupla}
	\end{Estructura}
	
	\Rep[estr][e]{true}
	
	~
	
	\Abs[estr]{posicion}[e]{$p$}
	{X($p$) $=$ $e$.x $\land$ Y($p$) $=$ $e$.y}
	
	~
	
	\tadOperacion{Max}{nat,nat}{nat}{}
	\tadAxioma{Max(x,y)}{	
		\IF x >\ y
		THEN x
		ELSE y
		FI}	
	
	~
	
	\tadOperacion{Min}{nat,nat}{nat}{}
	\tadAxioma{Min(x,y)}{	
		\IF x >\ y
		THEN y
		ELSE x
		FI}
	
\end{Representacion}

\begin{Algoritmos}
	
	\begin{algorithm}[H]
		\caption{iCrearPosicion}
		\begin{algorithmic}
			\Procedure{iCrearPosicion}
			{\texttt{in} x : \texttt{nat}, \texttt{in} y : \texttt{nat}} $\to res$ : estr
			\State var pos : estr
			\State $pos.x \gets$ x \Comment $\Theta$(1)
			\State $pos.y \gets$ y \Comment $\Theta$(1)
			\State $res \gets$ pos \Comment $\Theta$(1)
			\EndProcedure
		\end{algorithmic}
		\underline{Complejidad:}$\Theta$(1)
	\end{algorithm}
	
	\begin{algorithm}[H]
		\caption{iX}
		\begin{algorithmic}
			\Procedure{iX}{\texttt{in} pos : \texttt{estr}} $\to res$ : $nat$
			\State $res \gets$ pos.x \Comment $\Theta$(1)
			\EndProcedure
		\end{algorithmic}
		\underline{Complejidad:} $\Theta$(1)
	\end{algorithm}
	
	\begin{algorithm}[H]
		\caption{iY}
		\begin{algorithmic}
			\Procedure{iY}{\texttt{in} pos : \texttt{estr}} $\to res$ : $nat$
			\State $res \gets$ pos.y \Comment $\Theta$(1)
			\EndProcedure
		\end{algorithmic}
		\underline{Complejidad:} $\Theta$(1)
	\end{algorithm}
	
	\begin{algorithm}[H]
		\caption{iSumarPosiciones}
		\begin{algorithmic}
			\Procedure{iSumarPosicion}
			{\texttt{in} pos$_1$ : \texttt{estr}, \texttt{in} pos$_2$ : \texttt{estr}} $\to res$ : estr
			\State var pos : estr
			\State $pos \gets$ iCrearPosicion(pos$_1$.x $+$ pos$_2$.x, pos$_1$.y $+$ pos$_2$.y) \Comment $\Theta$(1)
			\State $res \gets$ pos \Comment $\Theta$(1)
			\EndProcedure
		\end{algorithmic}
		\underline{Complejidad:}$\Theta$(1)
	\end{algorithm}
	
	\begin{algorithm}[H]
		\caption{iRestarPosiciones}
		
		\begin{algorithmic}
			\Procedure{iRestarPosiciones}
			{\texttt{in} pos$_1$ : \texttt{estr}, \texttt{in} pos$_2$ : \texttt{estr}} $\to res$ : $estr$
			\State var pos : estr
			\State $pos \gets$ iCrearPosicion(pos$_1$.x $-$ pos$_2$.x, pos$_1$.y $-$ pos$_2$.y) \Comment $\Theta$(1)
			\State $res \gets$ pos \Comment $\Theta$(1)
			\EndProcedure
		\end{algorithmic}
		\underline{Complejidad:} $\Theta$(1)
	\end{algorithm}
	
	\begin{algorithm}[H]
		\caption{iDistanciaPosiciones}
		\begin{algorithmic}
			\Procedure{iDistanciaPosiciones}
			{\texttt{in} pos$_1$ : \texttt{estr}, \texttt{in} pos$_2$ : \texttt{estr}} $\to res$ : $nat$
			\State var pos : nat
			\State $pos \gets$ Max(pos$_1$.x, pos$_2$.x) $-$ Min(pos$_1$.x, pos$_2$.x) + Max(pos$_1$.y, pos$_2$.y) $-$ Min(pos$_1$.y,pos$_2$.y) \Comment $\Theta$(1)
			\State $res \gets$ pos \Comment $\Theta$(1)
			\EndProcedure
		\end{algorithmic}
		\underline{Complejidad:} $\Theta$(1)
	\end{algorithm}
	
\end{Algoritmos}

\subsection{Matriz($\alpha$)}
%probando
\begin{Interfaz}
  
  \textbf{parámetros formales}\hangindent=2\parindent\\
  \parbox{1.7cm}{\textbf{géneros}} $\alpha$\\
  \parbox[t]{1.7cm}{\textbf{función}}\parbox[t]{\textwidth-2\parindent-1.7cm}{%
    \InterfazFuncion{Copiar}{\In{a}{$\alpha$}}{$\alpha$}
    {$res \igobs a$}
    [$\Theta(copy(a))$]
    [función de copia de $\alpha$'s]
  }

  \textbf{se explica con}: \tadNombre{Matriz$(\alpha)$}.

  \textbf{géneros}: \TipoVariable{mat$(\alpha)$}.
  
  \Titulo{Operaciones básicas de matriz}

  \InterfazFuncion{Crear}{\In{alto}{nat},\In{ancho}{nat},\In{a}{($\alpha$)}}{matriz($\alpha$)}
  {$res \igobs$ Crear(ancho, alto, a)}
  [$\Theta$(ancho*alto)]
  [genera una matriz de ancho*alto que contenga todos a.]

  \InterfazFuncion{$\bullet$[$\bullet$][$\bullet$]}{\Inout{m}{matriz($\alpha$)}, \In{ancho}{nat}, \In{alto}{nat}}{$\alpha$}
  [PosValida(m, ancho, alto)]
  {alias($res \igobs$ Valor(m, ancho, alto))}
  [$\Ogr$(1)]
  [devuelve el elemento en la posicion ancho*alto de la matriz m con alias]
  [el elemento se retorna con aliasing y se podra modificar]
  
  \InterfazFuncion{Alto}{\In{m}{matriz($\alpha$)}}{nat}
  {$res \igobs$ Alto(m)}  
  [$\Ogr$(1)]
  [retorna la altura de la matriz m]
  
  \InterfazFuncion{Ancho}{\In{m}{matriz($\alpha$)}}{nat}
  {$res \igobs$ Ancho(m)}  
  [$\Ogr$(1)]
  [retorna la anchura de la matriz m]
   
\end{Interfaz}


\section{Tads}
\subsection{Posicion}
\begin{tad}{\tadNombre{Posicion}}
\tadGeneros{posicion}
\tadExporta{todo}
\tadUsa{Nat}

\tadIgualdadObservacional{p}{p'}{posicion}{X($p$) $\igobs$ X($p'$) \\
$\land$ Y($p$) $\igobs$ Y($p'$)}

\tadAlinearFunciones{Estudiantes}{placa/p,placa/p, grilla/g}

\tadObservadores
\tadOperacion{X}{posicion}{Nat}{}
\tadOperacion{Y}{posicion}{Nat}{}

\tadGeneradores
\tadOperacion{<$\bullet$, $\bullet$>}{Nat, Nat}{posicion}{}

\tadOtrasOperaciones
\tadOperacion{$\bullet$ + $\bullet$}{posicion,posicion}{posicion}{}
~
\tadOperacion{$\bullet$ - $\bullet$}{posicion/p,posicion/p'}{posicion}
{Y($p'$) $<$ Y($p$) $\land$ X($p'$) $<$ X($p$)}
~
\tadOperacion{Dist}{posicion,posicion}{posicion}{}

\subsection{Axiomas}

\tadAlinearAxiomas{aSal($d$,$y$,$alto$)}

\tadAxioma{X(<$x$, $y$>)}{$x$}
~
\tadAxioma{Y(<$x$, $y$>)}{$y$}
~
\tadAxioma{p $+$ p'}
{ < X(p) $+$ X(p'), Y(p) $+$ Y(p') >}
~
\tadAxioma{p $-$ p'}
{ < X(p) $-$ X(p'), Y(p) $-$ Y(p') >}
~
\tadAxioma{Dist($p$,$p'$)}
{ Max( X(p), X(p') ) $-$ Min( X(p), X(p') )  $+$ \\  Max( Y(p), Y(p') ) $-$ Min( Y(p), Y(p') )}
~
\end{tad}

\subsection{Matriz}
\begin{tad}{\tadNombre{Matriz($\alpha$)}}
\tadIgualdadObservacional{m}{m'}{matriz($\alpha$)}{ancho(m) $\igobs$ ancho(m') $\land$ alto(m) $\igobs$ alto(m') \\
$\yluego$ ( ($\forall col, fila$: nat ) posValida(m, col, fila ) $\impluego$
(valor(m, col, fila) $\igobs$ valor(m', col, fila ) ) )}

\tadParametrosFormales{
    \tadEncabezadoInline{géneros}{$\alpha$}
}

\tadGeneros{matriz($\alpha$)}
\tadExporta{matriz($\alpha$), generadores, observadores, posValida}
\tadUsa{\tadNombre{$\alpha$, Bool, Nat}}

%notar la importancia de agrupar el ultimo parametro, para la coma
\tadAlinearFunciones{posValida}{columa/nat,fila/nat,$\alpha$,{matriz($\alpha$)}/m}

\tadObservadores 
\tadOperacion{ancho}{matriz($\alpha$)}{nat}{} 
\tadOperacion{alto}{matriz($\alpha$)}{nat}{}
\tadOperacion{valor}{{matriz($\alpha$)}/m,columna/nat,fila/nat}{$\alpha$}{posValida(m, columna, fila)}

\tadGeneradores
\tadOperacion{crear}{ancho/nat,alto/nat,$\alpha$}{matriz($\alpha$)}{ancho $>$ 0 $\land$ alto $>$ 0} 
\tadOperacion{definir}{columna/nat,fila/nat,$\alpha$,{matriz($\alpha$)}/m}{matriz($\alpha$)}{posValida(m, columna, fila)}

\tadOtrasOperaciones
\tadOperacion{posValida}{matriz($\alpha$),columna,fila}{bool}{}

\tadAxiomas[\paratodo{matriz($\alpha$)}{m}, \paratodo{nat}{an, al}, \paratodo{$\alpha$}{a}]
\tadAlinearAxiomas{valor(definir($c$, $f$, $a$, $m$), $c'$, $f'$)}

\tadAxioma{ancho(crear($an$, $al$, $a$))}{$an$}
\tadAxioma{ancho(definir($c$, $f$, $a$, $m$))}{ancho($m$)}

\tadAxioma{alto(crear($an$, $al$, $a$))}{$al$}
\tadAxioma{alto(definir($c$, $f$, $a$, $m$))}{alto($m$)}

\tadAxioma{valor(crear($an$, $al$, $a$), $c$, $f$)}{$a$}
\tadAxioma{valor(definir($c$, $f$, $a$, $m$), $c'$, $f'$)}{\IF $f=f' \land c=c'$ THEN $a$ ELSE valor($m$,$c'$,$f'$) FI}

\tadAxioma{posValida($m$, $c$, $f$)}{$f$ $<$ alto($m$) $\land$ $c$ $<$ ancho($m$)}


\end{tad}

\subsection{Diccionario Acotado}
\begin{tad}{\tadNombre{DiccionarioAcotado(clave, significado)}}
\tadIgualdadObservacional{d}{d'}{diccAcot($\kappa,\sigma$)}{tamaño(d) $\igobs$ tamaño(d') $\land$ \\
$(\forall c:\kappa)$(def?($c, d$) $\igobs$ def?($c, d'$) $\yluego$\\ (def?($c, d$) $\impluego$ obtener($c, d$) $\igobs$ obtener($c, d'$)))}

\tadParametrosFormales{
    \tadEncabezadoInline{géneros}{clave, significado}
}

\tadGeneros{diccAcot(clave, significado)}
\tadExporta{diccAcot(clave, significado), generadores, observadores, borrar, claves}
\tadUsa{\tadNombre{Bool, Nat, Conjunto(clave)}}

%notar la importancia de agrupar el ultimo parametro, para la coma
\tadAlinearFunciones{obtener}{clave,significado,{diccAcot(clave, significado)}}

\tadObservadores 
\tadOperacion{def?}{clave,{diccAcot(clave, significado)}}{bool}{} 
\tadOperacion{obtener}{clave/c,{diccAcot(clave, significado)}/d}{significado}{def?($c$, $d$)}
\tadOperacion{tamaño}{diccAcot(clave, significado)}{nat}{}

\tadGeneradores
\tadOperacion{vacío}{nat}{diccAcot(clave, significado)}{}
\tadOperacion{definir}{clave,significado,{diccAcot(clave, significado)}/d}{diccAcot(clave, significado)}{$\#$claves($d$) $<$ tamaño}

\tadOtrasOperaciones
\tadOperacion{borrar}{clave/c,{diccAcot(clave, significado)}/d}{diccAcot(clave, significado)}{def?($c$,$d$)}
\tadOperacion{claves}{{diccAcot(clave, significado)}}{conj(clave)}{}

\tadAxiomas[\paratodo{diccAcot(clave, significado)}{d}, \paratodo{clave}{c, k}, \paratodo{significado}{s}]
\tadAlinearAxiomas{obtener($c$, definir($k$, $s$, $d$))}

\tadAxioma{def?($c$,vacío($n$))}{false}
\tadAxioma{def?($c$, definir($k$, $s$, $d$)}{$c = k$ $\lor$ def?($c$, $d$)}

\tadAxioma{obtener($c$, definir($k$, $s$, $d$))}{\IF $c = k$ THEN $s$ ELSE obtener($c$, $d$) FI}

\tadAxioma{tamaño(vacio($n$))}{n}
\tadAxioma{tamaño(definir($k$, $s$, $d$))}{tamaño(d)}

\tadAxioma{borrar($c$, definir($k$, $s$, $d$))}{\IF $c = k$ THEN {\IF def?($c$,$d$) THEN borrar($c$,$d$) ELSE $d$ FI} ELSE
definir($k$, $s$, borrar($c$, $d$)) FI}

\tadAxioma{claves(vacío($n$))}{$\emptyset$}
\tadAxioma{claves(definir($c$,$s$,$d$))}{Ag($c$, claves($d$))}

\end{tad}

\end{document}
