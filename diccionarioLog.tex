\begin{Interfaz}
	
	\textbf{parámetros formales}\hangindent=2\parindent\\
	\parbox{1.7cm}{\textbf{géneros}} $\sigma$\\
	\parbox[t]{1.7cm}{\textbf{función}}\parbox[t]{\textwidth-2\parindent-1.7cm}{%
		\InterfazFuncion{Copiar}{\In{s}{$\sigma$}}{$\sigma$}
	    	{$res \igobs a$}
		[$\Theta(copy(s))$]
		[función de copia de $\sigma$'s]
	}
	
	\textbf{se explica con}: \tadNombre{diccionarioAcotado(nat,$\sigma$)}
	
	\textbf{usa}: nat, bool
	
	\textbf{géneros}: \TipoVariable{diccLog(nat, $\sigma$)}
	
	\tituloModulo{Operaciones de DiccionarioLog(nat,$\sigma$)}
	
	\InterfazFuncion{Vacio}{\In{tam}{nat}}{diccLog(nat, $\sigma$)}
	{res $\igobs$ vacio(tam)}
	[$\Ogr$(1)]
	[Crea un diccionario vacío con tam como límite de claves]
	
	\InterfazFuncion{Definir}{\Inout{d}{diccLog(nat, $\sigma$)}, \In{c}{nat}, \In{s}{$\sigma$}}{}
	[($\forall$ c' : nat)( c' $\in$ claves(d) $\Rightarrow$ c $>$ c') $\land$ d=$d_0$ $\land$ $\#$claves(d) $<$ tamaño(d)]
	{d $\igobs$ Definir(c,s,$d_0$)}
	[$\Ogr$(1)]
	
	\InterfazFuncion{Definido}{\Inout{d}{diccLog(nat,$\sigma$)}, \In{c}{nat}}{bool}
	{res $\igobs$ def?(c,d)}
	[$\Ogr$(log $\#$claves(d))]
	
	\InterfazFuncion{Obtener}{\Inout{d}{diccLog(nat,$\sigma$)}, \In{c}{nat}}{$\sigma$}
	[def?(c,d) $\land$ d=$d_0$]
	{alias(res $\igobs$ obtener(c,d)) $\land$ ($\forall$ c' : nat)( (c' $\in$ claves(d) $\land$ c $\neq$ c') 
	$\Rightarrow$ obtener(c',d) $\igobs$ obtener(c,d) )}
	[$\Ogr$(log $\#$claves(d))]
	
\end{Interfaz}

\begin{Representacion}

	\tituloModulo{Representación DiccionarioLog(nat, $\sigma$)}
	
	\begin{Estructura}{diccLog(nat, $\sigma$)}
		\begin{Tupla}
			\tupItem{tamaño}{nat}
			\tupItem{valores}{arreglo dimensionable de tupla<$clave$ : nat, $significado$ : $\sigma$>}
			\tupItem{proxADefinir}{nat}
		\end{Tupla}
	\end{Estructura}
	
	\Rep{SinRepetidos(Claves(e.valores)) $\land$ Ordenada(Claves(e.valores))\\
		e.tamaño $\igobs$ tamaño(e.valores) $\land$ e.proxADefinir $\leq$ e.tamaño}
		
~
	\tadOperacion{SinRepetidos}{secu(nat)}{bool}{}
	\tadAxioma{SinRepetidos(s)}{
		\IF vacio?(s)
			THEN true
			ELSE	 $\neg$Esta?(prim(s),fin(s)) $\land$ sinRepetidos(fin(s))
		FI}
	
~	
	val es arreglo dimensionable de tupla<$clave$ : nat, $significado$ : $\sigma$>
	\tadOperacion{Claves}{val}{secu(nat)}{}
	\tadAxioma{Claves(v)}{
		\IF tamaño(v) = 0
			THEN <>
			ELSE Primero(v).clave $\bullet$ Claves(fin(v))
		FI}
	
~

	\tadOperacion{Ordenada}{secu(nat)}{bool}{}
	\tadAxioma{Ordenada(s)}{
		\IF vacia?(s)
			THEN true
			ELSE ($\forall$ n : nat) (n $\in$ fin(s) $\Rightarrow$ n $>$ Primero(s)) $\land$ Ordenada(fin(s))
		FI}
	
~

	\AbsFc{diccLog(nat,$\sigma$)}{d : diccLog(nat,$\sigma$)/\\
	e.tamaño $\igobs$ tamaño(d)\\
	($\forall$ c : nat)( def?(c,d) $\Leftrightarrow$ ($\exists$ i : nat)( 0 $\leq$ i $<$ e.tamaño) $\land$ definido(e.valores,i) $\yluego$ e.valores[i].clave $\igobs$ c )\\
	($\forall$ c : nat)( def(c,d) $\impluego$ Obtener(c,d) $\igobs$ Significado(c,e) )}
	
~

	\tadOperacion{Significado}{nat,estr}{$\sigma$}{def?(c,d)}
	\tadAxioma{Significado(c,e)}{ e.valores[Posicion(c,e.valores)].significado}
	
~

	\tadOperacion{Posicion}{nat,val}{nat}{}	
	\tadAxioma{Posicion(c,v)}{
		\IF Primero(v).clave = c
			THEN 0
			ELSE 1 + Posicion(c,fin(v))
		FI}
		
~

	\begin{algorithm}[H]
		\caption{iVacio}
		\begin{algorithmic}
			\Procedure{iVacio}{\texttt{in} tam : \texttt{nat}} $\to res$ : $estr$
				\State $e.tamaño \gets$ tam \Comment $\Ogr$(1)
				\State $e.valores \gets$ CrearArreglo(tam) \Comment $\Ogr$(1)
				\State $e.proxADefinir \gets$ 0 \Comment $\Ogr$(1)				
			\EndProcedure
		\end{algorithmic}
		\underline{Complejidad:} $\Ogr$(1)
	\end{algorithm}
	
	\begin{algorithm}[H]
		\caption{iDefinir}
		\begin{algorithmic}
			\Procedure{iDefinir}{\texttt{in/out} e : \texttt{estr}, \texttt{in} c : \texttt{nat}, \texttt{in} s : \texttt{$\sigma$}}
				\State $e.valores[e.proxADef] \gets$ tupla<c,s> \Comment $\Ogr$(1)
				\State $e.proxADefinir \gets$ e.proxADefinir $+$ 1 \Comment $\Ogr$(1)
			\EndProcedure
		\end{algorithmic}
		\underline{Complejidad:} $\Ogr$(1)
	\end{algorithm}
	
	\begin{algorithm}[H]
		\caption{iDefinido}
		\begin{algorithmic}
			\Procedure{iDefinido}{\texttt{in} e : \texttt{estr}, \texttt{in} c : \texttt{nat}} $\to res$ : $bool$
				\State $arriba \gets$ e.proxADefinir - 1 \Comment $\Ogr$(1)
				\State $abajo \gets$ 0 \Comment $\Ogr$(1)
				\While{arriba $>$ abajo} \Comment Se repite log e.proxADefinir veces y lo de adentro es $\Ogr$(1)\\
				 \hfill $\Rightarrow$ $\Ogr$(log e.proxADefinir)
					\State $	medio \gets$ (arriba $+$ abajo) $/$ 2 \Comment $\Ogr$(1)
					\If{e.valores[medio].clave $<$ c} \Comment $\Ogr$(1)
						\State $abajo \gets$ medio \Comment $\Ogr$(1)
					\Else
						\State $arriba \gets$ medio \Comment $\Ogr$(1)
					\EndIf
				\EndWhile
				\State $res \gets$ ( e.valores[medio].clave $=$ c ) \Comment $\Ogr$(1)
			\EndProcedure
		\end{algorithmic}
		\underline{Complejidad:} e.proxADefinir $-$ 1 coincide con la cantidad de claves definidas en el diccionario, pues se van agregando adelante. $\Ogr$(e.proxADefinir) $=$ $\Ogr$($\#$claves)
	\end{algorithm}
	
	\begin{algorithm}[H]
		\caption{iObtener}
		\begin{algorithmic}
			\Procedure{iObtener}{\texttt{in} e : \texttt{estr}, \texttt{in} c : \texttt{nat}} $\to res$ : $\sigma$
				\State $arriba \gets$ e.proxADefinir - 1 \Comment $\Ogr$(1)
				\State $abajo \gets$ 0 \Comment $\Ogr$(1)
				\While{arriba $>$ abajo} \Comment Se repite log e.proxADefinir veces y lo de adentro es $\Ogr$(1) $\Rightarrow$ $\Ogr$(log e.proxADefinir)
					\State $	medio \gets$ (arriba $+$ abajo) $/$ 2 \Comment $\Ogr$(1)
					\If{e.valores[medio].clave $<$ c} \Comment $\Ogr$(1)
						\State $abajo \gets$ medio \Comment $\Ogr$(1)
					\Else
						\State $arriba \gets$ medio \Comment $\Ogr$(1)
					\EndIf
				\EndWhile
				\State $res \gets$ e.valores[medio].significado \Comment $\Ogr$(1)
			\EndProcedure
		\end{algorithmic}
		\underline{Complejidad:} $\Ogr$($\#$claves)\\
		\underline{Justificación:} e.proxADefinir $-$ 1 coincide con la cantidad de claves definidas en el diccionario, pues se van agregando adelante. $\Ogr$(e.proxADefinir) $=$ $\Ogr$($\#$claves)
	\end{algorithm}
	
\end{Representacion}